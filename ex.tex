\subsection*{An Example}

\begin{figure}[t]
	\centering
	\lstset{
		numbers=left,
		numberstyle= \tiny,
		keywordstyle= \color{blue!70},
		commentstyle= \color{red!50!green!50!blue!50},
		frame=shadowbox,
		rulesepcolor= \color{red!20!green!20!blue!20} ,
		xleftmargin=1.5em,xrightmargin=0em, aboveskip=1em,
		framexleftmargin=1.5em,
                numbersep= 5pt,
		language=C,
    basicstyle=\scriptsize\ttfamily,
    numberstyle=\scriptsize\ttfamily,
    emphstyle=\bfseries,
                moredelim=**[is][\color{red}]{@}{@},
		escapeinside= {(*@}{@*)}
	}
\begin{lstlisting}[]
private static boolean isFinalSigma(java.lang.String s, int index) {
  if(index <= 0) {
    return false;
  }
  char previous = s.charAt(index - 1);
  if(!(java.lang.Character.isLowerCase(previous) || java.lang.Character.isUpperCase(previous) || java.lang.Character.isTitleCase(previous))) {
    return false;
  }
  if(index + 1 >= java.lang.String.length()) {
    return true;
  }
  ...
}
\end{lstlisting}
        \vspace{-12pt}
        \caption{An Example of Correct FQN Resolution}
        \label{fig:ex1}
\end{figure}

Fig.~\ref{fig:ex1} displays an example that {\tool} correctly resolved
the FQNs for the API elements. For convenience, we listed the code
with all the FQNs of the API elements. As seen, {\tool} is able to
recognize the API elements from the \code{java.lang} of the JDK
library. For example, the formal type \code{String} of \code{s}, the
FQNs of the method calls \code{isLowerCase}, \code{isUpperCase}, and
\code{isTitleCase}, and the FQN of the method call \code{length}.
Note that, \code{length} is a very popular simple name that matches
with the prefixes of several API elements.
