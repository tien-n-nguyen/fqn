\subsection{Key Ideas}
\label{sec:key}
Following Observations 1--2 and the limitations in existing state-of-the-art approaches, we designed {\tool} to identify the FQNs in a code snippet with the following key ideas:

\vspace{2pt}
\noindent {\bf Key Idea 1} [{\em Leveraging Regularity of API Usages}].
We leverage the pattern-capturing abilities of deep learning models to implicitly learn co-occurring API elements in API usages and derive the FQNs. Observation 1 guides the first principle of our solution, which is the basis of regularity of API usages in the training corpus: the API elements (with their FQNs) appearing together more regularly in API usages have a higher impact on deciding the FQNs than the less regular ones. To facilitate this idea and ensure that the FQNs of all API elements in use are known, we utilize complete, compilable code using libraries from a large code corpus.

\vspace{2pt}
\noindent {\bf Key Idea 2} [{\em ``Tell Me Your Friends, I'll Tell You
    Who You Are''}].  We consider FQN resolution as a task of
identifying inter-connected API elements in a given code snippet. As
described in Observation 2, rather than attempting to determine the
FQN of an API element based solely on its individual characteristics
(e.g., local surrounding context) as in the state-of-the-art approach
Huang {\em et al.}~\cite{prompt-ase22}, we adopt a broader 
context that include program dependencies and inter-API
relations. Based on this knowledge, we aim to {\em derive all related
API elements simultaneously}.

%We use a graph representation, called Augmented Usage Graph
%(AUG)~\cite{msr19}, to represent the program dependencies and
%relations among program entities and API elements. We extract the AUGs
%from the complete, compilable source code using the APIs from a large
%code corpus. We then enhance the AUGs with all the FQNs because the
%training code is compilable. From those AUGs,

\vspace{2pt}
\noindent {\bf Key Idea 3} [{\em FQN Resolution via Text Infilling}].
Deriving FQNs in an incomplete code snippet can be postulated as a fill-in-the-blank task, wherein the missing parts of a fully-qualified name are inserted ahead of its corresponding simple name at all relevant type inference locations. For example, in Fig.~\ref{fig:example3}, a blank representing \code{com.google.gwt.event.dom.client} will be inserted ahead of \code{ClickEvent}. To fill in all such inserted blanks, we adopt a Causal Language Model (CLM) capable of generating text and extend its capabilities to predict the missing spans of text in the blanks. Let us call this framework an Infilling Language Model (ILM). The blanks thus constructed are consistent with the preceding and subsequent code, thus enabling the ILM to gain access to the wider context and learn from the dependencies among the API and program elements in the code snippet.
