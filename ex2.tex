\subsubsection{Example 2}

\begin{figure}[t]
	\centering
	\lstset{
		numbers=left,
		numberstyle= \tiny,
		keywordstyle= \color{blue!70},
		commentstyle= \color{red!50!green!50!blue!50},
		frame=shadowbox,
		rulesepcolor= \color{red!20!green!20!blue!20} ,
		xleftmargin=1.5em,xrightmargin=1em, aboveskip=1em,
		framexleftmargin=1.5em,
                numbersep= 5pt,
                firstnumber=0,
		language=Java,
    basicstyle=\tiny\ttfamily,
    numberstyle=\tiny\ttfamily,
    emphstyle=\bfseries,
                moredelim=**[is][\color{red}]{@}{@},
		escapeinside= {(*@}{@*)}
	}
\begin{minipage}{.52\textwidth}
\begin{lstlisting}[label = left5, caption = Project: gwt class 11]
public class gwt_class_11 implements (*@{\color{orange}{<blank>}@*).EntryPoint {
    (*@{\color{orange}{<blank>}@*).HorizontalPanel dictionaryPanel;
    (*@{\color{orange}{<blank>}@*).Label wordLabel;
    public gwt_class_11(){
        dictionaryPanel = new (*@{\color{orange}{<blank>}@*).HorizontalPanel();
        wordLabel = new (*@{\color{orange}{<blank>}@*).Label("Word");
    }
    public void onModuleLoad() {
        dictionaryPanel.add(wordLabel);
        (*@{\color{orange}{<blank>}@*).RootPanel.get("dictionary").add(dictionaryPanel);
    }
}
\end{lstlisting}
\end{minipage}
\hspace{2pt}
\begin{minipage}{.46\textwidth}
\begin{lstlisting}[label = right5, caption = Ground Truth]
Blank 1: "com.google.gwt.core.client.EntryPoint",
Blank 2: "com.google.gwt.user.client.ui.HorizontalPanel", 
Blank 3: "com.google.gwt.user.client.ui.Label", 
Blank 4:   "com.google.gwt.user.client.ui.HorizontalPanel", 
Blank 5: "com.google.gwt.user.client.ui.Label", 
Blank 6: "com.google.gwt.user.client.ui.RootPanel"

\end{lstlisting}
\end{minipage}  
\vspace{-18pt}
\caption{Correct Example from {\tool}}
\label{eval:example6}
\end{figure}


{\ref{eval:example6}} highlights an example that {\tool} correctly
resolved the FQNs for the incomplete code. {\tool} leverages the
philosophy that ``Tell me your friend, I'll tell you who you are''.
It leverages the semantic dependencies among the API elements to derive
all the FQNs (identities) at once. For example, at line 4, the FQN-line-4
has a method \code{HorizonPanel}, which returns a dictionaryPanel. This
dictionaryPanel in turns has a method \code{add} (line 8) and
can be used as argument of the method \code{add} at line 9. Those
connections help {\tool} decide the FQNs.

%Here  in the incomplete code dictionaryPanel and wordLabel objects are type of HorizontalPanel and Label respectively. 
