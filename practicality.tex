\section{Practicality Evaluation}
\label{sec:eval}

In this experiment, we seek to establish \tool's efficacy in a more practical setting, i.e., in resolving FQNs for real-world (incomplete) code snippets from StackOverflow.

\subsection{Experiment Methodology}
Phan {\em et al.}~\cite{icse18} collected a benchmark dataset of 268 code snippets from StackOverflow to represent how developers use Java libraries in practice. To establish the ground-truth FQN annotations for these code snippets, they manually added the required libraries and made them compilable. These code snippets are relevant to the six libraries listed in Section~\ref{sec:effectiveness-data}. For brevity, let us refer to this dataset as StatType-SO. Among these, 
%the code snippets utilizing \code{jdk} and \code{android} libraries reference a wider range of APIs. 
we test the performance on code snippets utilizing \code{gwt}, \code{hibernate-orm}, and \code{xstream} in this experiment.
%Overall, StatType-SO has \_\_ API elements, of which, \_\_ are unique.
%\subsection{Procedure \& Evaluation Metrics}
To retrieve our data instances for this experiment, first, we employ the TILE algorithm (see Section~\ref{sec:tile}) to identify all type inference points across the methods in the StatType-SO dataset. Next, we utilize the constructed $\langle$\blank-code, FQN$\rangle$ pairs to create infilling examples consistent with Section~\ref{sec:effectiveness-eval-proc}. Finally, we leverage the trained \tool model to predict the missing FQNs in these infilling examples.
% \underline{\textit{Baseline}}: The state-of-the-art approach,
% MLM\textsubscript{\textit{FIB}}~\cite{prompt-ase22}, for FQN
% resolution outperforms previous works~\cite{coster-ase19, snr-icse22}
% on the benchmark StatType-SO dataset. Therefore, we chose
% MLM\textsubscript{\textit{FIB}} as the only baseline for this
% experiment. To facilitate the comparison, we construct data instances
% as in Section~\ref{sec:effectiveness-eval-proc}, and leverage the
% trained MLM\textsubscript{\textit{FIB}} to infer the FQNs.
%\underline{\textit{Evaluation Metrics}}:
We adopt the same evaluation metrics as in Section~\ref{sec:effectiveness-eval-proc}, i.e., Accuracy\textsubscript{\textit{EM}}, ROUGE-L, and BLEU-2.

\subsection{Effectiveness on Incomplete Java Code (RQ\textsubscript{4})}
\label{sec:rq4}

In Table~\ref{tab:results-practical}, we report the performance of our model for incomplete Java code snippets. We observed that the fully-qualified names predicted by {\tool} exactly match the ground-truth fully-qualified names for \code{gwt}, \code{hibernate}, and \code{xstream} with an accuracy of 69.23\%, 40.00\%, and 20.83\%, respectively. Among the three tested libraries from the StatType-SO dataset, we achieved the best performance overall, across all metrics in the \code{gwt} library, and \tool faced the most difficulty with \code{xstream}.


% \begin{table}[]
% \centering
% %\scriptsize
% \begin{tabular}{l|c|c|c|l}
% \toprule
% \multicolumn{1}{r|}{\textbf{Baselines} ($\rightarrow$)} & \multicolumn{1}{l|}{\multirow{2}{*}{COSTER}} & \multicolumn{1}{l|}{\multirow{2}{*}{SnR}} & \multicolumn{1}{l|}{\multirow{2}{*}{MLM\textsubscript{FIB}}} & \multirow{2}{*}{\tool} \\
% \multicolumn{1}{l|}{\textbf{Libraries} ($\downarrow$)} & \multicolumn{1}{l|}{}                                 & \multicolumn{1}{l|}{}                              & \multicolumn{1}{l|}{}                              &                                   \\ 
% \hline
% \tabcode{android}                            & 43.3                                                  & 93.6                                               & 90.6                                               &                                   \\
% \tabcode{gwt}                                & 90.8                                                  & 75.8                                               & 95.2                                               &                                   \\
% \tabcode{hibernate}                          & 90.4                                                  & 94.8                                               & 75.9                                               &                                   \\
% \tabcode{jdk}                                & 56.2                                                  & 71.1                                               & 98.9                                               &                                   \\
% \tabcode{joda-time}                          & 57.1                                                  & 89.5                                               & 97.5                                               &                                   \\
% \tabcode{xstream}                            & 88.4                                                  & 100.0                                              & 88.0                                               &                                   \\ \hline
% \textbf{Avg. Accuracy}                          & 71.0                                                  & 87.5                                               & 91.0                                               &      \\
% \bottomrule
% \end{tabular}
% \caption{FQN resolution accuracy (in \%) for incomplete code snippets from StackOverflow (StatType-SO).}
% \label{tab:results-practical}
% \end{table}

% \begin{table}[]
% \centering
% %\scriptsize
% \begin{tabular}{l|c|c|c}
% \toprule
% \multicolumn{1}{r|}{\textbf{Baseline} ($\rightarrow$)} & \multicolumn{3}{c}{\tool} \\ \cline{2-4}
% \textbf{\textbf{Libraries} ($\downarrow$)}             & $Accuracy\textsubscript{\textit{EM}}$     & $ROUGE-L$     & $BLEU-2$    \\
% \hline
% \tabcode{android}                                      &           &           &          \\
% \tabcode{gwt}                                          & 69.23          & 83.39          & 41.53         \\
% \tabcode{hibernate}                                    & 40.00           &  58.61         & 0.27          \\
% \tabcode{jdk}                                          &           &           &          \\
% \tabcode{joda-time}                                    &           &           &          \\
% \tabcode{xstream}                                      & 20.83          &   0.41        & 0.32         \\
% \bottomrule
% \end{tabular}
% \caption{Comparison of model performance for incomplete code snippets from StackOverflow (StatType-SO), based on {\em Exact Match Accuracy} ($M_1$), {\em ROUGE-L} ($M_2$), {\em BLEU-2} ($M_3$)}
% \label{tab:results-practical}
% \end{table}
\begin{table}[]
\begin{tabular}{c|ccc}
\toprule
\multirow{2}{*}{\begin{tabular}[c]{@{}c@{}}\textbf{Evaluation}\\ \textbf{Metrics}\end{tabular}} & \multicolumn{3}{c}{\textbf{Libraries}} \\ \cline{2-4} 
                 & \multicolumn{1}{c|}{\code{gwt}}   & \multicolumn{1}{c|}{\code{hibernate}} & \code{xstream} \\ 
\midrule
\textit{EM}      & \multicolumn{1}{c|}{69.23} & \multicolumn{1}{c|}{40.00}     & 20.83   \\
\textit{ROUGE-L} & \multicolumn{1}{c|}{83.29} & \multicolumn{1}{c|}{58.61}     & 0.41    \\
\textit{BLEU-2}  & \multicolumn{1}{c|}{41.53} & \multicolumn{1}{c|}{0.27}      & 0.32    \\ \bottomrule
\end{tabular}
\caption{Comparison of performance for incomplete Java programs (RQ2)}
\label{tab:practicality}
\end{table}


%Alternatively, the StatType-SO dataset contains an increased number of ex- ternal FQNs (76.4\%) (col.E-Fs) w.r.t COSTER-SO. Such a behaviour is also reflected in the number of unique external FQNs (col.UE-Fs) with only 10 for COSTER-SO and 128 for StatType-SO. To further increase the evaluation setup, we created another dataset (RESICO-SO) that replicates the distribution of FQNs in the StatType-SO dataset and possibly improves the number of unique external FQNs compared to previous datasets. This new dataset represents the third dataset considered to verify the generalisability of the results.

