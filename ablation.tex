\section{Ablation Study (RQ4)}
\label{sec:ablation}

In Section~\ref{sec:key}, we hypothesized that a wider {\em dependency
  context} can provide crucial hints towards overcoming name ambiguity
and identifying the API elements better. Accordingly, we input the
code of the entire method to train {\tool}. Let us denote such a
dataset by $D_w$. In contrast,
MLM\textsubscript{\textit{FIB}}~\cite{prompt-ase22} leverages a
narrower surrounding context (two lines before and after the statement
containing the API element). Let us denote this dataset by $D_n$. The
details for building $D_w$ and $D_n$ were described in
Section~\ref{sec:effectiveness-eval-proc}.

The goal of this experiment is to gauge the impact that the wider
dependency context has on FQN resolution. Thus, we conducted an
ablation on the amount of contextual information, and created two
baselines: (a) \tool w/ dependency context, i.e., when trained on
$D_w$, (b) \tool w/o dependency context, i.e., trained on $D_n$. We
evaluate both ablation baselines against the test set in
Section~\ref{sec:effectiveness-data}.

\subsection*{Results and Analysis (RQ\textsubscript{2})}
\label{sec:rq2}

We can see in Table~\ref{tab:ablation} that building an intrinsic knowledge base across a wider dependency context (i.e., $B_w$) helps \tool significantly outperform the ablation baseline trained on a narrower surrounding context (i.e., $B_n$), achieving a relative improvement of 157.14\%, 47.54\%, and 59.65\% in \textit{Accuracy\textsubscript{EM}}, \textit{ROUGE-L}, and \textit{BLEU-2} scores, respectively. 

Consider the \code{junit.framework.Assert} class in the \textit{JUnit} unit testing framework which contains a set of assertion methods useful for writing tests. More recently, the \code{Assert} class in \code{junit.framework} has been deprecated, and moved to \code{org.junit.Assert}. In our dataset, about 98.89\% API elements that belong to \code{junit.framework.Assert} class are referenced in the \code{gwt} library; and about 97.4\% belonging to \code{org.junit.Assert} are in the \code{hibernate} library. We studied the performance of our ablation baselines ($B_n$ and $B_w$) on these APIs. We noticed that in the absence of dependency context, $B_n$ incorrectly predicts APIs belonging to \code{junit.framework.Assert} 35.53\% of the time, and those in \code{org.junit.Assert} 30.75\% of the time. In contrast, $B_w$ makes such incorrect predictions only 9.14\% and 8.19\% of the time, respectively. The additional contextual knowledge in the form of inter-API dependencies available in $B_w$ likely contributed to it understanding that the same API element, e.g., \code{assertEquals}, when referenced in the \code{gwt} library corresponds to \code{junit.framework.Assert.assertEquals}; and when referenced in the \code{hibernate} library corresponds to \code{org.junit.Assert.assertEquals}. Thus, we can both qualitatively and quantitatively corroborate \textit{Key Idea 2} by demonstrating the role of {\bf dependency context} in FQN resolution.


% junit.framework.Assert appears in android 0 times, gwt 78807 times (98.89%), hibernate 828 times, jdk 0 times, joda-time 0 times, xstream 54 times.
% org.junit.Assert appears in android 652 times, gwt 0 times, hibernate 24916 times (97.4%), jdk 3 times, joda-time 0 times, xstream 9 times.

% junit.framework.Assert appears 2558 times, and org.junit.Assert appears 891 times.

% In our model, juni.. is incorrectly predicted 234 times, i.e., 9.14% times; and orf.. is incorrectly predicted 73 times, i.e., 8.19%.

% In baseline, juni.. is incorrectly predicted 909 times (35.53); and org.. is incorrectly predicted 274 times, i.e., 30.75% of the times.
