\section{Introduction}
\label{sec:intro}

Software libraries play important roles in software development. Their
functionality is provided via the Application Programming Interface
(API) elements including classes, methods, and fields. According to
the usage documentation of the libraries, one needs to use the API
elements in a specific combination and/or order to accomplish certain
programming tasks. Such usages are referred to as {\em API
  usages}. The frequently occurring API usages are called~{\em
  API~usage patterns}. Several approaches have been leveraging API
usages and/or patterns to support software engineering tasks such as
pattern-based code completion~\cite{icse12}, API method
completion~\cite{fse16}, API usage recommendation~\cite{icsme18}, API
code migration~\cite{ase14}, fully-qualified name
suggestion~\cite{icse18}, API documentation
suggestion~\cite{liveapi14}, etc.

To be able to use API elments correctly, developers could rely on API
documentation or online forums, e.g., StackOverflow or GitHub
Gists. Unfortunately, due to the context of the ongoing discussions in
such forums, the code snippets in the answering StackOverflow posts
may contain the ambiguities on the references to the API elements of
the external libraries~\cite{liveapi14}.

Several approaches have been proposed to automatically resolve the
fully qualified names (FQNs) of the API elements for the code
snippets.  The approaches can be classified into different
categories. The first category is {\em program analysis}. Partial
program analysis~\cite{dagenais-oopsla08} derives the types and FQNs
in a best-effort manner. RecoDoc~\cite{dagenais-icse12} utilizes
several heuristics on syntactic constructs to recognize the names.
The second category is {\em information
  retrieval}. Baker~\cite{liveapi14} builds a candidate list for each
name by tracking the scopes of the names and then overlapping the
lists according to the scoping rules for short-listing.
COSTER~\cite{coster-ase19} captures the context of the query API
element and matches that with the FQNs of API elements with three
criteria on likelihood, context similarity, and name similarity.  The
third category is {\em constraint-based}. SnR~\cite{snr-icse22} builds
a knowledge base of APIs, i.e., various facts about the available
APIs. SnR extracts typing constraints from the given snippet, and
solves the constraints against the knowledge base. The key limitation
with the above categories of approaches is that the dictionary of the
APIs must be known a priori, thus, they cannot deal with the FQNs that
a model has not seen before in the training corpus.

To address that, several approaches leverage the advances in {\em
  artificial intelligence (AI) and machine learning
  (ML)}. StatType~\cite{icse18} considers the problem of deriving FQNs
as the {\em phrase-based machine translation} from the code without
FQNs to the one with them. The key issue with StatType is that it does
not capture well the relations of the APIs in the surrounding context,
which is important in helping decide the FQN of a simple name of an
API. Huang {\em et al.}~\cite{prompt-ase22} formulate the problem as a
{\em fill-in-blank task} using a masked language model. The approach
fills in the FQN for each name considering the context consisting of
the few lines before and after the fill-in location. The key issue
with this approach is that the API elements are designed to use in
different client code, making the surrounding contexts different in
different usage code.

We introduce {\tool} to recover the FQNs for all the APIs in any code
snippet. Instead of characterizing the context with a few lines before
and after the location of an API element to derive its FQN, we
leverage the dependencies and relations among the API elements in API
usages to decide the FQNs for all the relevant APIs at once. We
consider the FQNs as the identities of the API elements and we follow
the philosophy {\em ``Tell Me Your Friends, I'll Tell You Who You
  Are''} to derive the FQNs. To determine the ``friends'' of an API
element $A$, we leverage the program entities and other API elements
that depend on and/or have relations with $A$. The software library
designers always have the intents for users to use certain API
elements together in the specific combinations/orders to achieve a
programming task. Such combinations of API elements are often
referred to as {\em API usages}. Therefore, such API elements in API
usages frequently occur together in the code using the libraries,
which are referred to as {\em API usage patterns}.

That {\em basis of regularity of API usages} drives our approach in
two ways. First, the API elements with their FQNs regularly appear
together in API usages have higher impact in deciding the FQNs than
the less regular ones. Thus, a ML model could {\em learn such
  regularly co-appearing API elements from a training corpus to help
  derive their FQNs at once}. We leverage the complete, compilable
code from the open-source code repositories using the libraries in
which the FQNs of all the API elements are available. Second, {\em the
  program dependencies and usage relations among API elements are
  useful in deriving their identities (i.e., FQNs)}. For example, in
the following code: \code{Button b = new Button(``OK'');}
\code{b\-.addClickHandler\-(\-new ClickHandler\-())}. The class name
\code{Button} is very popular across several libraries (e.g.,
\code{Button} appears in both Google Web Toolkit (GWT) and
\code{Android} library). Relying solely on the first statement might
not be sufficiently to derive the FQNs for the API elements
\code{Button}, \code{addClickHandler}, and \code{ClickHandler}, and
the variable \code{b}. However, the \code{def-use} dependency between
two statements via the variable \code{b} could give a model a hint on
the FQNs of all the elements.  If a model learns in the training
corpus that a GWT \code{Button} can be a receiving object of a method
call \code{addClickHandler}, which in turn accepts as the single
argument an object of the type \code{ClickHandler}, it will be able to
derive the FQNs for the elements in both statements at once.

Putting together those ideas, we design {\tool}, ...

We performed several experiments to evaluate {\tool}. ...

The key contributions of this paper include:

1. {\tool}: a neural-network approach ...


2. An extensive evaluation showing {\tool}'s better accuracy in
deriving FQNs than the state-of-the-art approaches.

3. An analysis showing the embeddings for the API elements ...
