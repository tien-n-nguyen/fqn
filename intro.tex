\section{Introduction}
\label{sec:intro}

External software libraries play a crucial role in software development. They provide functionalities to accomplish specific programming tasks via Application Programming Interface (API) elements such as classes, methods, and fields. To learn how to reuse them, developers often turn to API documentation and online forums, e.g., StackOverflow (S/O) or GitHub Gists and search for concrete API usage examples. However, integrating such API elements is not straightforward. First, the documentation can vary in quality and clarity. Thus, the developers might need to spend more time understanding the API usage and how it fits into their codebase.

Next, the code snippets in online forums carrying API usages are interspersed between conversational contexts and are typically incomplete. To work with such partial code while ensuring the correct reference/usage of API elements, the resolution of their fully qualified names (FQNs) is mandatory.
However, due to the lack of import statements and library dependencies, such references might be ambiguous and the information needed for FQN resolution might not be readily available. For example, consider an incomplete code snippet that references \code{ParseException} element to signal that an error has unexpectedly been reached while parsing. This API element can correspond to either \code{java.text.ParseException} in the \code{jdk} library, or \code{android.net.ParseException} in the \code{android} library. To be able to correctly resolve its fully qualified name, one might require a thorough understanding of the API element, the context in which it is supposed to be used, and its potential dependencies in the external code.

%~\cite{liveapi14}.
% fully-qualified name suggestion~\cite{icse18}

Several approaches have been proposed to automatically resolve the fully qualified names of API elements in a code snippet. We can classify them into various categories, the first being program analysis. Partial program analysis (PPA)~\cite{dagenais-oopsla08} infers the types/FQNs by resolving the syntactic ambiguities in incomplete code snippets in a best-effort manner and analyzing the relations among the program elements. RecoDoc~\cite{dagenais-icse12} tries to recover a code-like term's fully-qualified name by linking it with all the inferred types in the system code based on several heuristics. One key issue with such best-effort, heuristic-driven analyses is that they are not comprehensive and do not work in all cases. Besides, RecoDoc requires the code snippet to be a subset of a partial program to carry out such an analysis.

The second category leverages {\em information retrieval} (IR) techniques. Given a code snippet, Baker~\cite{liveapi14} tracks the scope of all API elements to build a candidate list for each, which it then overlaps based on the scoping rules for shortlisting. However, this technique is ineffective as the missing information in incomplete code often leaves each API element with potentially many FQNs.
%Baker extracts constraints from code snippets and uses a naive constraint solving algorithm to infer FQNs.
%Baker~\cite{liveapi14} builds a candidate list for each name by tracking the scopes of the names and then overlapping the lists according to the scoping rules for shorlisting.
COSTER~\cite{coster-ase19} extracts locally specific code elements (i.e., \textit{local context}) and globally related tokens (i.e., \textit{global context}) of the query API element to infer their FQNs based on three criteria: likelihood, context similarity, and name similarity.
The next category leverages {\em constraint-based} techniques. SnR~\cite{snr-icse22} first builds a knowledge base of APIs from the existing libraries by extracting various type-related facts about the available APIs. Then, SnR extracts typing constraints from a given code snippet and uses the knowledge-base to infer a set of APIs that satisfy the constraints.
The key limitation with the IR and constraint-based approaches is that the dictionary of the APIs must be known apriori, thus, they cannot deal with the FQNs  that have not been seen before in the training corpus.

To address this issue of out-of-vocabulary (OOV) failures, some approaches leveraged {\em machine} and {\em deep learning} (ML \& DL) techniques for deriving the FQNs. 
StatType~\cite{icse18} formulates FQN resolution as a {\em phrase-based machine translation} task where an API sequence without FQNs is constructed for a given code snippet and translated to an equivalent one with FQNs.
In contrast, Huang et al.~\cite{prompt-ase22} model FQN resolution as a fill-in-the-blank problem, transforming the given code snippet by: (a) identifying the type inference (TI) locations using regular expressions; (b) fragmenting the code snippet into code blocks, each corresponding to a TI location; (c) inserting \code{[MASK]} tokens in all locations; (d) independently resolving their FQNs by filling in the corresponding \code{[MASK]} tokens with the help of a {\em masked language model} (MLM\textsubscript{\textit{FIB}}).
However, the key issue with both approaches is that their limited amount of context does not account for the dependence of the API elements on other program elements in the code snippet. Let us refer to this as \textbf{dependency context}. Such contextual information can provide crucial hints towards deciding between multiple FQNs corresponding to an API element with the same simple name but from different libraries. 

% To address that, a few approaches have leveraged {\em artificial
%   intelligence (AI) and machine learning (ML)} in deriving
% FQNs. StatType~\cite{icse18} formulates the problem of FQN resolution
% as the {\em phrase-based machine translation} from the code without
    % FQNs to the one with them. The key issue is that with
    % its~short phrases of code tokens, StatType does not capture well the
    % relations of the APIs in the surrounding context, which is useful
    % in deciding the FQNs from the simple names of the API elements.
    % %
% Huang {\em et al.}~\cite{prompt-ase22} formulates this problem as a
% filling-in-a-blank task with an FQN using a masked language model.
% It collects as a training instance a code snippet consisting
% of one simple name for an API element and a couple lines before and
    % after that name. Thus, if those surrounding lines do not contain any
    % API names, the model could not take advantage of the relations among
    % the API elements to derive their FQNs. In other words, while using the
    % context of a few lines, it does not leverage the {\bf dependency
    %   context} among the API elements for FQN prediction. In fact, the
    % dependency context can provide crucial hints for deriving FQNs. Let us
    % take an example in Google Web Toolkit (GWT). 
    \begin{lstlisting}[
frame=shadowbox,
rulesepcolor= \color{red!20!green!20!blue!20} ,
xleftmargin=1.5em,xrightmargin=0em, aboveskip=1em,
framexleftmargin=1.5em,
numbersep= 5pt,
language=Java,
basicstyle=\scriptsize\ttfamily, numberstyle=\scriptsize\ttfamily, emphstyle=\bfseries, escapeinside= {(*@}{@*)}]
{
  StyleElement style=Document.get().createStyleElement();
  cssRes = resources.(*@{\color{blue}css();@*)
  ...
  s = cssRes.(*@{\color{blue}getText();@*)
  ...
  style.(*@{\color{blue}setInnerText(s);@*)
  ...
}                
\end{lstlisting}
%Need a small example
%Take as an example a code snippet (not shown) in our dataset with the
%presence of \code{getText}. This name is popular with a very large
%number of API method candidates. However, considering the relation
%between \code{css} and \code{getText} in the code
%\code{`...css()\-.getText()'},~the number of candidates for
%\code{getText} is only 4. Finally, considering~the return value of
%\code{getText} as an argument of \code{setInnerText(...)} in the code
%\code{`setInnerText(...css()\-.getText())'}, only one candidate is
%remained:
%\code{com\-.google\-.gwt\-.resources\-.client\-.CssResource\-.getText()}.
%Thus, those relations actually help identify the API elements,
For example, consider the above code snippet inspired from the S/O post \#34595450. The \code{getText} method call on line 5 is a very popular API name that occurs in multiple libraries. Nguyen et al.~\cite{icse18} reported that there are about 500 API methods of that name across five different Java libraries. Consider the data dependency between line 3 and line 5 on account of the variable \code{cssRes}. We can see that the return type of \code{css(...)} on line 3 has a member API method with the name \code{getText}. With this knowledge, the number of candidates for \code{getText} can be brought down to 4. Next, consider the data dependency between line 5 and line 7 via the variable \code{s}. We can see that the API method \code{setInnerText} accepts as the first argument a variable of the type compatible with the return type of \code{getText} on line 5. Based on these criteria, the candidates can further be narrowed down to a single one, namely, \code{com\-.google\-.gwt\-.resources\-.client\-.CssResource\-.getText()}. Thus, we can see that the relationships among API names can help identify the corresponding API element. However, the state-of-the-art DL approach, MLM\textsubscript{\textit{FIB}}, fragments the given code snippet into multiple code blocks \textit{CB\textsubscript{i}}, each corresponding to the API elements on lines 3, 5, and 7 to independently derive their FQNs. In the absence of such a dependency context, precisely identifying the FQN of \code{getText} would not be possible.

We introduce {\tool}, an ML-based approach to recover the FQNs for all
the APIs in any code snippet. We leverage the dependencies among the
API elements in API usages to decide the FQNs for all the relevant
APIs at once. We consider the FQNs as the identities of the API
elements and we follow the philosophy {\em ``Tell Me Your Friends,
  I'll Tell You Who You Are''} to derive the FQNs. To determine the
``friends'' of an API element $A$, we leverage the program entities
and other API elements that depend on or have relations with $A$ as in
the above example. The rationale is that the software library's
designers always have the intents for users to use certain API
elements together in the specific combinations to achieve a
programming task. Such combinations of API elements are often referred
to as {\em API usages}, which could appear in multiple code
repositories that use the corresponding libraries.

%Therefore, such API elements in API usages frequently occur together
%in the code using the libraries, which are referred to as {\em API
%  usage patterns}.

That {\em basis of regularity of API usages} drives our approach in
two ways. First, the API elements with their FQNs regularly appear
together in API usages have higher impact in deciding the FQNs than
the less regular ones. Thus, a ML model could {\em learn such
  regularly co-appearing API elements from a training corpus to help
  derive their FQNs at once}. We leverage the complete, compilable
code from the open-source code repositories using the libraries in
which the FQNs of all the API elements are available. Second, {\em the
  program dependencies and relations among API elements also
  frequently happen due to intention of the libraries' designers}.
That dependency context as explained earlier is useful in deriving the
identities (i.e., FQNs) of the API elements.

%use ILM, and show that it is capable of capturing the dependencies
%among API elements

Putting together those ideas, we design {\tool}, ...  {\bf Talking
  about ILM and its capability to capture the
  dependencies/relations...}

We performed several experiments to evaluate {\tool}. ...

The key contributions of this paper include:

1. {\tool}: a neural-network approach ...

2. An extensive evaluation showing {\tool}'s better accuracy in
deriving FQNs than the state-of-the-art approaches.

