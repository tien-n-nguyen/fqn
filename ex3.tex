\subsubsection{Example 3}

\begin{figure}[t]
	\centering
	\lstset{
		numbers=left,
		numberstyle= \tiny,
		keywordstyle= \color{blue!70},
		commentstyle= \color{red!50!green!50!blue!50},
		frame=shadowbox,
		rulesepcolor= \color{red!20!green!20!blue!20} ,
		xleftmargin=1.5em,xrightmargin=1em, aboveskip=1em,
		framexleftmargin=1.5em,
                numbersep= 5pt,
                firstnumber=0,
		language=Java,
    basicstyle=\tiny\ttfamily,
    numberstyle=\tiny\ttfamily,
    emphstyle=\bfseries,
                moredelim=**[is][\color{red}]{@}{@},
		escapeinside= {(*@}{@*)}
	}
\begin{minipage}{.55\textwidth}
\begin{lstlisting}[label = left3, caption = Example 3]
public class Android13 {
  public static void main(String[] args) {
    (*@{\color{orange}{<blank>}@*).Uri imageUri = null;
    (*@{\color{orange}{<blank>}@*).Bitmap bitmap = null;
  try {
    bitmap = (*@{\color{orange}{<blank>}@*).Media.getBitmap(getContentResolver(), imageUri);
  } catch ((*@{\color{orange}{<blank>}@*).FileNotFoundException e) {
     e.printStackTrace();
  } catch (((*@{\color{orange}{<blank>}@*).IOException e) {
     e.printStackTrace();
  }
  (*@{\color{orange}{<blank>}@*).ByteArrayOutputStream bytes = new (*@{\color{orange}{<blank>}@*).ByteArrayOutputStream();
  bitmap.compress((*@{\color{orange}{<blank>}@*).Bitmap.CompressFormat.JPEG, 40, bytes);
  (*@{\color{orange}{<blank>}@*).ByteArrayInputStream fileInputStream = new (*@{\color{orange}{<blank>}@*).ByteArrayInputStream(bytes.toByteArray());
}
\end{lstlisting}
\end{minipage}
\hspace{2pt}
\begin{minipage}{.42\textwidth}
\begin{lstlisting}[label = right2, caption = Fully-Qualified Names]
"blanks": [["android.net", "android.net.Uri"], ["android.graphics", "android.graphics.Bitmap"], ["android.provider.MediaStore.Images", "android.provider.MediaStore.Images.Media"], ["java.io", "java.io.FileNotFoundException"], ["java.io", "java.io.IOException"], ["java.io", "java.io.ByteArrayOutputStream"], ["android.graphics", "android.graphics.Bitmap"], ["android.content", "android.content.ContentResolver"], ["java.io", "java.io.ByteArrayInputStream"]]
\end{lstlisting}
\end{minipage}  
\vspace{-18pt}
\caption{Correct Example from {\tool}}
\label{eval:example3}
\end{figure}

Fig.~\ref{eval:example3} displays an incomplete code snippet in which
{\tool} correctly predicted all the FQNs for all the API elements. The
code snippet contains 10 ``blanks'', that is, 10 locations that need
to be filled in with the FQNs. The correct FQNs are shown in
Listing~\ref{right2}. {\tool} leverages the dependencies among the API
elements in the code snippet to derive their FQNs at the same time.
For example, considering line 5, the FQN-line-5 has a field called
\code{Media} and Media has a method \code{getBitmap}, which returns a
bitmap.  At line 3, that bitmap has the FQN-line-3. From such semantic
dependencies, the FQNs at different lines for different API elements
are interdependent to one another. Thus, {\tool} can detect their FQNs
at once.
