%%
%% This is file `sample-acmsmall-conf.tex',
%% generated with the docstrip utility.
%%
%% The original source files were:
%%
%% samples.dtx  (with options: `acmsmall-conf')
%% 
%% IMPORTANT NOTICE:
%% 
%% For the copyright see the source file.
%% 
%% Any modified versions of this file must be renamed
%% with new filenames distinct from sample-acmsmall-conf.tex.
%% 
%% For distribution of the original source see the terms
%% for copying and modification in the file samples.dtx.
%% 
%% This generated file may be distributed as long as the
%% original source files, as listed above, are part of the
%% same distribution. (The sources need not necessarily be
%% in the same archive or directory.)
%%
%%
%% Commands for TeXCount
%TC:macro \cite [option:text,text]
%TC:macro \citep [option:text,text]
%TC:macro \citet [option:text,text]
%TC:envir table 0 1
%TC:envir table* 0 1
%TC:envir tabular [ignore] word
%TC:envir displaymath 0 word
%TC:envir math 0 word
%TC:envir comment 0 0
%%
%%
%% The first command in your LaTeX source must be the \documentclass
%% command.
%%
%% For submission and review of your manuscript please change the
%% command to \documentclass[manuscript, screen, review]{acmart}.
%%
%% When submitting camera ready or to TAPS, please change the command
%% to \documentclass[sigconf]{acmart} or whichever template is required
%% for your publication.
%%
%%
\documentclass[acmsmall,review,screen]{acmart}

%%
%% \BibTeX command to typeset BibTeX logo in the docs
\AtBeginDocument{%
  \providecommand\BibTeX{{%
    Bib\TeX}}}

%% Rights management information.  This information is sent to you
%% when you complete the rights form.  These commands have SAMPLE
%% values in them; it is your responsibility as an author to replace
%% the commands and values with those provided to you when you
%% complete the rights form.
\setcopyright{acmcopyright}
\copyrightyear{2018}
\acmYear{2018}
\acmDOI{XXXXXXX.XXXXXXX}

%% These commands are for a PROCEEDINGS abstract or paper.
\acmConference[Conference acronym 'XX]{Make sure to enter the correct
  conference title from your rights confirmation emai}{June 03--05,
  2018}{Woodstock, NY}
%%
%%  Uncomment \acmBooktitle if the title of the proceedings is different
%%  from ``Proceedings of ...''!
%%
%%\acmBooktitle{Woodstock '18: ACM Symposium on Neural Gaze Detection,
%%  June 03--05, 2018, Woodstock, NY}
\acmPrice{15.00}
\acmISBN{978-1-4503-XXXX-X/18/06}

\setcopyright{none}
\settopmatter{printfolios=false,printccs=false,printacmref=false}

\settopmatter{printacmref=false} % Removes citation information below abstract
\renewcommand\footnotetextcopyrightpermission[1]{} % removes footnote with conference

\usepackage{cite}
\usepackage{amsmath,amssymb,amsfonts}
\usepackage{algorithmic}
\usepackage{diagbox}
\usepackage{graphicx}
\usepackage{textcomp}
\usepackage{xcolor}
\def\BibTeX{{\rm B\kern-.05em{\sc i\kern-.025em b}\kern-.08em
    T\kern-.1667em\lower.7ex\hbox{E}\kern-.125emX}}

%\usepackage{amsmath,amssymb,amsfonts}
%\usepackage{algorithmic}
%\usepackage{graphicx}
%\usepackage{textcomp}
%\usepackage{xcolor}


%\usepackage{cite}
\usepackage{colortbl}
\usepackage{booktabs}   %% For formal tables:
                        %% http://ctan.org/pkg/booktabs
\usepackage{subcaption} %% For complex figures with subfigures/subcaptions
                        %% http://ctan.org/pkg/subcaption
\usepackage{array}
\usepackage{amsmath,amsfonts}
\usepackage{amssymb}
%\usepackage{algorithm}
%\usepackage[noend]{algpseudocode}
%\usepackage{algorithmic}
%\usepackage{graphicx}
%\usepackage{textcomp}
\usepackage{float}
\usepackage{listings}
\usepackage{xspace}
\usepackage{multirow}
\usepackage{amsthm}
\usepackage{enumerate}
\usepackage{enumitem}

\newtheorem{definition}{Definition}
\usepackage{balance}
\usepackage{printlen}
\usepackage[skins]{tcolorbox}
\usepackage{color, soul}

%\usepackage{xcolor,pifont}
%\newcommand*\colourcheck[1]{%
%	\expandafter\newcommand\csname #1check\endcsname{\textcolor{#1}{\ding{52}}}%
%}
%\colourcheck{blue}
%\colourcheck{green}
%\colourcheck{red}

\newtcolorbox{myframe}[2][]{%
  enhanced,colback=white,colframe=black,coltitle=black,
  sharp corners,
  toprule=1.0pt,
  rightrule=0.3pt,
  leftrule=0pt,
  bottomrule=0pt,
  fonttitle=\itshape\scshape\large,
  left=0pt,right=5pt,top=5pt,bottom=3pt,
  attach boxed title to top right={yshift=-0.3\baselineskip-0.4pt,xshift=-5mm},
  boxed title style={tile,size=minimal,left=0.2mm,right=0.5mm,
    colback=white,before upper=\strut},
  title=#2,#1
}

\newenvironment{nscenter}
 {\parskip=0pt\par\nopagebreak\centering}
 {\par\noindent\ignorespacesafterend}

%\newcommand{\code}[1]{{\footnotesize\textsf{#1}}}

\newcommand{\tool}{\textsc{DeepFQN}\xspace}
\newcommand{\blank}{\begin{footnotesize}\textsf{\textbf{[blank]}\ }\end{footnotesize}}
\newcommand{\tabblank}{\begin{scriptsize}\textsf{\textbf{[blank]}}\end{scriptsize}}
\newcommand{\answer}{\begin{footnotesize}\textsf{\textbf{[a-sep]}\ }\end{footnotesize}}
\newcommand{\sep}{\begin{footnotesize}\textsf{\textbf{[sep]}\ }\end{footnotesize}}

\newtheorem{Definition}{Definition}
\newtheorem{Claim}{Claim}
\newtheorem{Lemma}{Lemma}
\newtheorem{Theorem}{Theorem}

\newcolumntype{L}[1]{>{\raggedright\arraybackslash}p{#1}}
\newtheorem{Observation}{Observation}
\newtheorem{property}{Property}
\newcommand{\code}[1]{{\footnotesize\texttt{#1}}}
\newcommand{\tabcode}[1]{{\scriptsize\texttt{#1}}}
\usepackage{amsthm}
 \definecolor{dkgreen}{rgb}{0,0.6,0}
\definecolor{gray}{rgb}{0.5,0.5,0.5}
\definecolor{mauve}{rgb}{0.58,0,0.82}
\lstset{frame=tb,
  language=Java,
  aboveskip=3mm,
  belowskip=3mm,
  showstringspaces=false,
  columns=flexible,
  basicstyle={\small\ttfamily},
  numbers=left,
  numberstyle=\tiny\color{gray},
  keywordstyle=\color{blue},
  commentstyle=\color{dkgreen},
  stringstyle=\color{mauve},
  breaklines=true,
  breakatwhitespace=true,
  tabsize=4
}


%\usepackage{tikz}
%\usetikzlibrary{shapes.arrows}
%\newcommand{\FancyUpArrow}{\begin{tikzpicture}[baseline=-0.3em]
%		\node[single arrow,draw,rotate=90,single arrow head extend=0.1em,inner
%		ysep=0.1em,transform shape,line width=0.03em,top color=green,bottom color=green!50!black] (X){};
%\end{tikzpicture}}

%\def\BibTeX{{\rm B\kern-.05em{\sc i\kern-.025em b}\kern-.08em
%    T\kern-.1667em\lower.7ex\hbox{E}\kern-.125emX}}


%%
%% Submission ID.
%% Use this when submitting an article to a sponsored event. You'll
%% receive a unique submission ID from the organizers
%% of the event, and this ID should be used as the parameter to this command.
%%\acmSubmissionID{123-A56-BU3}

%%
%% For managing citations, it is recommended to use bibliography
%% files in BibTeX format.
%%
%% You can then either use BibTeX with the ACM-Reference-Format style,
%% or BibLaTeX with the acmnumeric or acmauthoryear sytles, that include
%% support for advanced citation of software artefact from the
%% biblatex-software package, also separately available on CTAN.
%%
%% Look at the sample-*-biblatex.tex files for templates showcasing
%% the biblatex styles.
%%

%%
%% The majority of ACM publications use numbered citations and
%% references.  The command \citestyle{authoryear} switches to the
%% "author year" style.
%%
%% If you are preparing content for an event
%% sponsored by ACM SIGGRAPH, you must use the "author year" style of
%% citations and references.
%% Uncommenting
%% the next command will enable that style.
%%\citestyle{acmauthoryear}


%%
%% end of the preamble, start of the body of the document source.
\begin{document}

\title[Enabling Fully-Qualified Name Resolution]{Enabling Fully-Qualified Name Resolution of API Elements via Generative Text Infilling}

%%
%% The "author" command and its associated commands are used to define
%% the authors and their affiliations.
%% Of note is the shared affiliation of the first two authors, and the
%% "authornote" and "authornotemark" commands
%% used to denote shared contribution to the research.

%\author{Ben Trovato}
%\authornote{Both authors contributed equally to this research.}
%\email{trovato@corporation.com}
%\orcid{1234-5678-9012}
%\author{G.K.M. Tobin}
%\authornotemark[1]
%\email{webmaster@marysville-ohio.com}
%\affiliation{%
%  \institution{Institute for Clarity in Documentation}
%  \streetaddress{P.O. Box 1212}
%  \city{Dublin}
%  \state{Ohio}
%  \country{USA}
%  \postcode{43017-6221}
%}

%\author{Lars Th{\o}rv{\"a}ld}
%\affiliation{%
%  \institution{The Th{\o}rv{\"a}ld Group}
%  \streetaddress{1 Th{\o}rv{\"a}ld Circle}
%  \city{Hekla}
%  \country{Iceland}}
%\email{larst@affiliation.org}

%\author{Valerie B\'eranger}
%\affiliation{%
%  \institution{Inria Paris-Rocquencourt}
%  \city{Rocquencourt}
%  \country{France}
%}

%\author{Aparna Patel}
%\affiliation{%
% \institution{Rajiv Gandhi University}
% \streetaddress{Rono-Hills}
% \city{Doimukh}
% \state{Arunachal Pradesh}
% \country{India}}

%\author{Huifen Chan}
%\affiliation{%
%  \institution{Tsinghua University}
%  \streetaddress{30 Shuangqing Rd}
%  \city{Haidian Qu}
%  \state{Beijing Shi}
%  \country{China}}

%\author{Charles Palmer}
%\affiliation{%
%  \institution{Palmer Research Laboratories}
%  \streetaddress{8600 Datapoint Drive}
%  \city{San Antonio}
%  \state{Texas}
%  \country{USA}
%  \postcode{78229}}
%\email{cpalmer@prl.com}

%\author{John Smith}
%\affiliation{%
%  \institution{The Th{\o}rv{\"a}ld Group}
%  \streetaddress{1 Th{\o}rv{\"a}ld Circle}
%  \city{Hekla}
 % \country{Iceland}}
%\email{jsmith@affiliation.org}

%\author{Julius P. Kumquat}
%\affiliation{%
%  \institution{The Kumquat Consortium}
%  \city{New York}
%  \country{USA}}
%\email{jpkumquat@consortium.net}

%%
%% By default, the full list of authors will be used in the page
%% headers. Often, this list is too long, and will overlap
%% other information printed in the page headers. This command allows
%% the author to define a more concise list
%% of authors' names for this purpose.

\renewcommand{\shortauthors}{Anonymous}

%%
%% The abstract is a short summary of the work to be presented in the
%% article.

\begin{abstract}
Software development relies heavily on the effective use of API elements, which provide pre-built functionalities for specific programming tasks. Online forums, e.g., StackOverflow and GitHub Gists contain vast collections of code snippets that utilize these APIs. However, due to the incomplete nature of these code snippets, determining the correct fully-qualified name (FQN) for API elements is challenging. Several approaches have been proposed to automatically resolve the FQNs, but face one or more of the following issues: (1) are not comprehensive; (2) can not handle out-of-vocabulary API elements; (3) do not consider dependence of the API element on other program elements. In this work, we  present {\tool}, a deep learning-based approach that addresses these issues by formulating FQN resolution as a \textit{fill-in-the-blank} task. For a given code snippet, our tool systematically identifies all API locations, inserts a blank ahead of the simple name corresponding to the API, leverages a causal language model (CLM) to fill the missing blank with the corresponding FQN.
Our empirical evaluation shows a relative improvement over the state-of-the-art FQN resolution approach by 9.1\% on complete Java code. Upon further analysis, we were able to tie the performance improvement to our tool's capabilities in capturing dependencies between the APIs as well as with the other program elements in the code snippet.  

\end{abstract}

\maketitle

\section{Introduction}
\label{sec:intro}
External software libraries play a crucial role in software development. They provide functionalities to accomplish specific programming tasks via Application Programming Interface (API) elements such as classes, methods, and fields. To learn how to reuse them, developers often turn to API documentation and online forums, e.g., StackOverflow (S/O) or GitHub Gists and search for concrete API usage examples. However, integrating API elements in this manner is not straightforward. First, the documentation can vary in quality and clarity. As a result, the developers might need to spend more time understanding the API usage and how it fits into their codebase. Next, the code snippets in online forums carrying API usages are interspersed between conversational contexts and are typically incomplete. Thus, they might include multiple references to unavailable/unresolved API elements, making it inexecutable and further ambiguous to understand.

To work with partial code while ensuring the correct reference/usage of its API elements, the resolution of their fully qualified names (FQNs) is mandatory. However, due to the lack of import statements and library dependencies, such references might be ambiguous and the information needed for FQN resolution might not be readily available. For example, consider an incomplete code snippet that references \code{ParseException} to signal that an error has unexpectedly been reached while parsing. This API element can correspond to either \code{java.text.ParseException} in the \code{jdk} library, or \code{android.net.ParseException} in the \code{android} library.

To correctly resolve the FQN of an API element, one might require a
thorough understanding of its usage, the context in which it is
supposed to be used, and the dependencies among the API elements. For
example, consider the code snippet in Fig.~\ref{fig:excerpt-example1}
inspired from the S/O post \#34595450. The \code{getText} method call
on line 5 is a very popular API name that occurs in multiple
libraries. Nguyen {\em et al.}~\cite{icse18} reported that there are
about 500 API methods of that name across five different Java
libraries. Consider the data dependency between line 3 and line 5 on
the account of the variable \code{cssRes}. We can see that the return type
of \code{css(...)} on line 3 has a member API method with the name
\code{getText}. With this knowledge, the number of candidates for
\code{getText} can be brought down to 4. Next, consider the data
dependency between line 5 and line 7 via the variable \code{s}. We can
see that the API method \code{setInnerText} accepts as the first
argument a variable of the type compatible with the return type of
\code{getText} on line 5. Based on these criteria, the candidates can
further be narrowed down to a single one, namely,
\code{com\-.google\-.gwt\-.resources\-.client\-.CssResource\-.getText()}. Thus,
knowledge of the dependence of the API elements on other program
elements in the code snippet can provide crucial hints towards
deciding among multiple FQNs corresponding to an API element with
the same simple name, but from different libraries. Let us refer to
such contextual information as \textbf{dependency context}.


\begin{figure}
\begin{lstlisting}[
frame=shadowbox,
rulesepcolor= \color{red!20!green!20!blue!20} ,
xleftmargin=1.5em,xrightmargin=0em, aboveskip=1em,
framexleftmargin=1.5em,
numbersep= 5pt,
language=Java,
basicstyle=\scriptsize\ttfamily, numberstyle=\scriptsize\ttfamily, emphstyle=\bfseries, escapeinside= {(*@}{@*)}]
{
  StyleElement style=Document.get().createStyleElement();
  cssRes = resources.(*@{\color{blue}css();@*)
  ...
  s = cssRes.(*@{\color{blue}getText();@*)
  ...
  style.(*@{\color{blue}setInnerText(s);@*)
  ...
}
\end{lstlisting}
\vspace{-12pt}
\caption{Excerpt from StackOverflow Post \#34595450} %on Google Web Toolkit Library}
\label{fig:excerpt-example1}
\end{figure}


%Several approaches have been proposed to automatically resolve the fully qualified names of API elements in a code snippet. We can classify them into various categories, the first being program analysis. Partial program analysis (PPA)~\cite{dagenais-oopsla08} infers the types/FQNs by resolving the syntactic ambiguities in incomplete code snippets in a best-effort manner and analyzing the relations among the program elements. RecoDoc~\cite{dagenais-icse12} tries to recover a code-like term's fully-qualified name by linking it with all the inferred types in the system code based on several heuristics. One key issue with such best-effort, heuristic-driven analyses is that they are \underline{not comprehensive} and do not work in all cases. Besides, RecoDoc requires the code snippet to be a subset of a partial program to carry out such an analysis.

%The second category leverages {\em information retrieval} (IR)
%techniques. Given a code snippet, Baker~\cite{liveapi14} tracks the
%scope of all API elements to build a candidate list for each, which
%it then overlaps based on the scoping rules for
%shortlisting. However, this technique is ineffective as the missing
%information in incomplete code often leaves each API element with
%potentially many FQNs. COSTER~\cite{coster-ase19} extracts locally
%specific code elements (i.e., \textit{local context}) and globally
%related tokens (i.e., \textit{global context}) of the query API
%element to infer their FQNs based on three criteria: likelihood,
%context similarity, and name similarity.  The next category leverages
%{\em constraint-based} techniques. SnR~\cite{snr-icse22} first builds
%a knowledge base of APIs from the existing libraries by extracting
%various type-related facts about the available APIs. Then, SnR
%extracts typing constraints from a given code snippet and uses the
%knowledge-base to infer a set of APIs that satisfy the
%constraints. The key limitation with the IR and constraint-based
%approaches is that the dictionary of the \underline{APIs must be
%known a priori}. Thus, they are ineffective in dealing with FQNs that
%have not been seen before in the training corpus.

Several approaches have been proposed to automatically resolve the
FQNs of API elements in a code snippet. We can
classify them into different categories, the first being {\em program
  analysis}. These program analysis approaches (Partial program
analysis (PPA)~\cite{dagenais-oopsla08},
RecoDoc~\cite{dagenais-icse12}) infers the types/FQNs by resolving the
syntactic ambiguities in partial code in a best-effort,
heuristic-driven fashion. The key issue with such heuristic analyses
is that they are \underline{not comprehensive} and do not work in all
cases. The second category leverages {\em information retrieval} (IR)
techniques (e.g., Baker~\cite{liveapi14}, COSTER~\cite{coster-ase19}).
They represent an API element with the features extracted from
contexts, build a candidate list for each API element via matching,
and infering the FQNs by overlapping the lists using scoping rules.
The third category leverages {\em constraint-based}
techniques. SnR~\cite{snr-icse22} builds a knowledge base of
APIs from the existing libraries by extracting various type-related
facts about the available APIs, and infers the FQNs satisfying
the constraints. The key limitation with the IR and constraint-based
approaches is that the dictionary of the \underline{APIs must be known
  a priori}. Thus, they are ineffective in dealing with FQNs that have
not been seen in training.

The latest category leverages {\em machine} and {\em deep learning}
(ML \& DL) techniques to tackle the issue of out-of-vocabulary (OOV)
while deriving the FQNs.  StatType~\cite{icse18} formulates FQN
resolution as a {\em phrase-based machine translation} task where an
API sequence without FQNs is constructed for a given code snippet and
translated to an equivalent one with FQNs. In contrast, Huang {\em et
  al.}~\cite{prompt-ase22} model FQN resolution as a fill-in-the-blank
problem by employing a masked language model
(MLM\textsubscript{\textit{FIB}}).  However, a key issue with both
approaches is that the limited amount of context they consider
does~not account for the dependency context.
%of the API elements on other program elements in the code snippet.
For example, to derive FQNs in the code snippet in
Fig.~\ref{fig:excerpt-example1}, the state-of-the-art DL approach,
MLM\textsubscript{\textit{FIB}}, fragments the code snippet into
multiple code blocks \textit{CB\textsubscript{i}}, each representative
of the API element on lines 3, 5, and 7. Moreover, each code block
\textit{CB\textsubscript{i}} includes a limited amount of surrounding
context, say, lines 4--6 corresponding to \code{getText} on line 5. As
described earlier, not being aware of \code{css} on line 3 and
\code{setInnerText} on line 7 makes the inferring of the FQN for
\code{getText} challenging. Thus, the \underline{lack of dependency
  context} yields MLM\textsubscript{\textit{FIB}} inadequate in
unambiguously resolving an FQN with the same simple name but belonging
to different libraries.
%Moreover, the splitting of code snippet into code blocks also results in an independent inference of the FQNs in each block, wherein, the correlations between the co-occurring API elements are overlooked.
Moreover, the division of the given code snippet into discrete code blocks disregards the inter-connections between the co-occurring API elements, as a result of which, the API element in each code block undergoes an {\em autonomous FQN inference process}.

In this paper, we introduce {\tool}, a DL-based approach which aims to address the issues highlighted in existing approaches and effectively recover the FQNs for all API elements in a given code snippet. Our approach takes advantage of the inter-dependencies between API elements in API usages to derive the appropriate FQNs for all pertinent API elements simultaneously. In essence, we view an FQN as the identity of an API element, and our approach is centered around the philosophy {\em ``Tell Me Your Friends, I'll Tell You Who You Are''}.
%To determine the ``friends'' of an API element $A$, we leverage the program entities and other API elements that depend on or have relations with $A$ as in the above example.
Our rationale is that the designers of software libraries always intend for users to use specific combinations of API elements together to achieve a given programming task. Such combinations of API elements, also known as {\em API usage patterns}, may appear in multiple code repositories that use the corresponding libraries.

%Therefore, such API elements in API usages frequently occur together
%in the code using the libraries, which are referred to as {\em API
%  usage patterns}.

The {\em basis of regularity} guides our approach in two ways. First, the associations between the more regularly appearing API elements and their corresponding FQNs play a significant role in determining the FQN of the API element in API usage. Thus, an ML/DL-based model could {\em learn from more frequently occurring API element-FQN pairs} in the training corpus. To facilitate this idea, we obtain complete, compilable code from open-source code repositories of the libraries, ensuring that the FQNs of all API elements are readily available. Second, {\em the program dependencies and relations shared more frequently among API elements are a consequence of the design objectives of software libraries}. Such dependency context, as explained earlier, is critical in ascertaining the identities (i.e., FQNs) of the API elements.

%use ILM, and show that it is capable of capturing the dependencies
%among API elements

Putting together these ideas, we break down the task of inferring FQNs for a given code snippet as follows: First, we design a Type Inference Locations Extraction (TILE) algorithm, which, given a complete/incomplete code snippet, systematically identifies all the API locations, and inserts a special \code{[blank]} token ahead of the simple name corresponding to the API element. In Fig.~\ref{fig:approach}, we show an illustration for how TILE transforms a code snippet without FQNs (left), to one with \code{[blank]} tokens (right). Next, we leverage an Infilling Language Model~\cite{donahue-etal-2020-enabling} (ILM), geared to predict the missing FQNs in the place of all \code{[blank]} tokens inserted in the previous step. We do so by enabling a causal language model (CLM), that is traditionally capable of and efficient at generating text in an auto-regressive fashion to predict the missing blanks that are consistent with the preceding and subsequent code.

%{\bf Coordinated Prediction}...

%First, MLM\textsubscript{\textit{FIB}} uses regular expressions to identify the locations of API elements in a code snippet in comparison to our syntax tree-based TILE algorithm. Thus, it is not comprehensive. Next, the masked language model used in MLM\textsubscript{\textit{FIB}} requires the prior knowledge of the length of each blank. However, this is not available a priori. Huang et al. work around this issue by a brute-force approach, i.e., by trying lengths ranging from 3 to 69 for each blank, and picking the one with the highest likelihood. In contrast, our tool, by design, is not reliant on the length of the FQN, and is capable of filling the blanks with variable length spans. Thus, \tool is much more time-efficient in inferring all the FQNs in a code snippet.

Though both \tool and the state-of-the-art
MLM\textsubscript{\textit{FIB}} model FQN resolution as a
fill-in-the-blank task, there are some fundamental departure from our
solution. First, with the aforementioned philosophy, {\tool} takes
advantage of the dependency context for FQN resolution. During
training, we train our model to learn the dependencies among the API
elements, while MLM\textsubscript{\textit{FIB}} aims to learn each FQN
in a few surrounding lines of code. Second, during prediction, with
the knowledge on dependency context, {\tool} resolves the FQNs for all
API elements in the code snippet at the same process.  In contrast,
Huang {\em et al.}~\cite{prompt-ase22} has an additional coordinating
prediction for multiple FQNs (blanks) by brute-force trying different
lengths of an FQN ranging from 3--69 for each blank and picking the
combination of all FQNs with the highest likelihood. Therefore,
without the prior knowledge of the length of each blank, their
approach is much less time-efficient than {\tool}, by design, is not
reliant on the length of the FQN, and is capable of filling the blanks
with variable length spans based on the dependency context. Finally,
MLM\textsubscript{\textit{FIB}} uses regular expressions to identify
the locations of API elements in a code snippet in comparison to our
syntax tree-based TILE algorithm. Thus, it is not comprehensive.



%First, MLM\textsubscript{\textit{FIB}} uses regular expressions to identify the locations of API elements in a code snippet in comparison to our syntax tree-based TILE algorithm. Thus, it is not comprehensive. Next, the masked language model used in MLM\textsubscript{\textit{FIB}} requires the prior knowledge of the length of each blank. However, this is not available a priori. Huang et al. work around this issue by a brute-force approach, i.e., by trying lengths ranging from 3 to 69 for each blank, and picking the one with the highest likelihood. In contrast, our tool, by design, is not reliant on the length of the FQN, and is capable of filling the blanks with variable length spans. Thus, \tool is much more time-efficient in inferring all the FQNs in a code snippet.


We performed several experiments to evaluate {\tool}. ...

The key contributions of this paper include:

1. {\tool}: a neural-network approach ...

2. An extensive evaluation showing {\tool}'s better accuracy in
deriving FQNs than the state-of-the-art approaches.

The model and data is available in our website~\cite{deepFQN-website}.


\section{Motivation}
\label{motiv:sec}

\subsection{Motivating Examples}
\label{examples:sec}

%https://stackoverflow.com/questions/4531396/get-value-of-a-edit-text-field/4531500#4531500: SO post #4531500
\begin{figure}[htbp]
	\centering
	\lstset{
		numbers=left,
		numberstyle= \tiny,
		keywordstyle= \color{blue!70},
		commentstyle= \color{red!50!green!50!blue!50},
		frame=shadowbox,
		rulesepcolor= \color{red!20!green!20!blue!20} ,
		xleftmargin=1.5em,xrightmargin=0em, aboveskip=1em,
		framexleftmargin=1.5em,
                numbersep= 5pt,
		language=C,
    basicstyle=\scriptsize\ttfamily,
    numberstyle=\scriptsize\ttfamily,
    emphstyle=\bfseries,
                moredelim=**[is][\color{red}]{@}{@},
		escapeinside= {(*@}{@*)}
	}
\begin{lstlisting}[]
(*@{\color{blue}{Button}@*)   mButton;
EditText mEdit;

(*@@@*)Override public (*@{\color{black}{void}@*) onCreate(Bundle savedInstanceState) {
    super.onCreate(savedInstanceState);
    setContentView(R.layout.main);

    (*@{\color{blue}mButton = findViewById(R.id.button);@*)
    mEdit   = (EditText)findViewById(R.id.edittext);

    (*@{\color{blue}mButton.setOnClickListener(@*)
        (*@{\color{purple}new View.OnClickListener()@*)
        {
            public (*@{\color{black}{void}@*) onClick(View view)
            {
                Log.v("EditText", mEdit.getText().toString());
            }
        });
}
\end{lstlisting}
        \vspace{-12pt}
        \caption{StackOverflow Post \#4531500 on Android Library}
        \label{fig:example1}
\end{figure}


Let us examine some real-world examples to help motivate our approach. Fig.~\ref{fig:example1} illustrates a code snippet of an answer to S/O question \#4531500 on the Android library. Due to the informal discourse in S/O, code snippets rarely contain the necessary declarations and references to the fully-qualified names (FQNs); and often lack the import statements. Besides, the references to external types are also unqualified since the responder assumes that those FQNs can be implicitly understood from the context of the post. For example, in Fig.~\ref{fig:example1}, the types \code{Button} (line 1), \code{EditText} (line 2), \code{Bundle} (line 4), \code{View} (line 14), or \code{Log} (line 16) are referenced only by their simple names. Thus, the code will not be compilable unless the corresponding import statements are added.

%https://stackoverflow.com/questions/18323473/how-to-implement-gwt-java-button-and-the-clickhandler
\begin{figure}[htbp]
	\centering
	\lstset{
		numbers=left,
		numberstyle= \tiny,
		keywordstyle= \color{blue!70},
		commentstyle= \color{red!50!green!50!blue!50},
		frame=shadowbox,
		rulesepcolor= \color{red!20!green!20!blue!20} ,
		xleftmargin=1.5em,xrightmargin=0em, aboveskip=1em,
		framexleftmargin=1.5em,
                numbersep= 5pt,
		language=C,
    basicstyle=\scriptsize\ttfamily,
    numberstyle=\scriptsize\ttfamily,
    emphstyle=\bfseries,
                moredelim=**[is][\color{red}]{@}{@},
		escapeinside= {(*@}{@*)}
	}
\begin{lstlisting}[]
public class myClass implements EntryPoint {
    final (*@{\color{blue}{Button}@*) myButton = new (*@{\color{blue}{Button}@*)("text");
    (*@{\color{blue}{myButton.addClickHandler(}@*)
        (*@{\color{purple}{new ClickHandler() \{}@*)
            public (*@{\color{black}{void}@*) onClick(ClickEvent event) {
               onClickMyButton(event);
        }
    });
    private (*@{\color{black}{void}@*) onClickMyButton(ClickEvent event) {
            ... 
    }
}
\end{lstlisting}
        \vspace{-12pt}
        \caption{StackOverflow Post \#18323473 on GWT Library}
        \label{fig:example2}
\end{figure}


Moreover, the APIs of external libraries are prone to name ambiguity, meaning that they can be confused with other APIs from different libraries that share the same name and offer a similar functionality. For example, the element \code{Button} on line 1 in Fig.~\ref{fig:example1} is a common unqualified type. Here, it refers to \code{android.widget.Button}. Next, consider Fig.~\ref{fig:example2}, which illustrates a code snippet of an answer to a S/O post on Google Web Toolkit (GWT). Because this code snippet does not contain any import statements and the references to the APIs are also unqualified, the type \code{Button} on line 2 in Fig.~\ref{fig:example2} is ambiguous from that on line 1 in Fig.~\ref{fig:example1}. Such name ambiguity in type names is a common phenomenon, especially among S/O code snippets. For instance, the simple name \code{getId} occurs 27,434 times across various Java libraries~\cite{liveapi14}.

\subsection{Observations}
\label{sec:obs}

We can observe a need for a tool that automatically derives the fully-qualified names of the API elements in code snippets from online forums. This will facilitate the reuse of such incomplete code by enabling the addition of the appropriate import statements. To build one such tool, we draw motivation from the following observations.

\vspace{2pt}
\noindent {\bf Observation 1} [{\em Regularity of API Usages}]. The designers of software libraries intend for developers to use the API elements together (including API classes, method calls, and field accesses) in certain combinations/orders to achieve a task. For example, in the GWT code snippet illustrated in Fig.~\ref{fig:example2}, a variable of the type \code{Button} (FQN: \code{com.google.gwt.user.client.ui.Button}) is instantiated on line 2. Then, on line 3, to set the handler of that GWT button, one needs to invoke the \code{addClickHandler} method (FQN: \code{com.google.gwt.user.client.ui.Button.add\-Click\-Handler})  on the \code{Button} object with an argument of the type \code{ClickHandler} (FQN: \code{com\-.google\-.gwt\-.event\-.dom\-.client\-.ClickHandler}). Thus, they are intended to be used in  such a combination and will appear together frequently.

Next, in Fig.~\ref{fig:example3}, we illustrate a complete code example published on the GWT tutorial website \code{gwtproject.org}. Here, the author provides all necessary \code{import} statements and demonstrates how to use different GUI elements in GWT. Specifically, consider the \code{Button} object declared with the \code{addStockButton} variable name on line 12. It calls the method \code{addClickHandler} on line 23 with an argument of the same type as earlier, \code{ClickHandler}. 
%Though the \code{Button} object is assigned with different variable names in both cases, i.e., \code{myButton} when incomplete, and \code{addStockButton} when complete, we can see that the combination of these API elements can help establish its identity in the form of its FQN. 
Despite being assigned with different variable names in both the complete and incomplete cases, i.e., \code{myButton} and \code{addStockButton} respectively, we can see that the combination of these API elements can help establish \code{Button} object's identity and resolve its FQN.

In brief, the source code in public repositories is a good source for a model to implicitly learn the API usages and derive the FQNs of the API elements in an incomplete snippet.

%https://www.gwtproject.org/doc/latest/tutorial/manageevents
\begin{figure}[htbp]
	\centering
	\lstset{
		numbers=left,
		numberstyle= \tiny,
		keywordstyle= \color{blue!70},
		commentstyle= \color{red!50!green!50!blue!50},
		frame=shadowbox,
		rulesepcolor= \color{red!20!green!20!blue!20} ,
		xleftmargin=1.5em,xrightmargin=0em, aboveskip=1em,
		framexleftmargin=1.5em,
                numbersep= 5pt,
		language=C,
    basicstyle=\scriptsize\ttfamily,
    numberstyle=\scriptsize\ttfamily,
    emphstyle=\bfseries,
                moredelim=**[is][\color{red}]{@}{@},
		escapeinside= {(*@}{@*)}
	}
\begin{lstlisting}[]
import com.google.gwt.core.client.EntryPoint;
import com.google.gwt.event.dom.client.ClickEvent;
import com.google.gwt.event.dom.client.ClickHandler;
import com.google.gwt.user.client.ui.Button;
import com.google.gwt.user.client.ui.HorizontalPanel;
import com.google.gwt.user.client.ui.RootPanel;
import com.google.gwt.user.client.ui.VerticalPanel;
...
public class StockWatcher implements EntryPoint {
  private VerticalPanel mainPanel = new VerticalPanel();
  private HorizontalPanel addPanel = new HorizontalPanel();
  private Button addStockButton = new Button("Add");
  ...
  public void onModuleLoad() {
    ...
    // Assemble Add Stock panel.
    addPanel.add(addStockButton);
    // Assemble Main panel.
    mainPanel.add(addPanel);
    // Associate the Main panel with the HTML host page.
    RootPanel.get("stockList").add(mainPanel);
    // Listen for mouse events on the Add button.
    (*@{\color{blue}{addStockButton.addClickHandler(}@*) (*@{\color{purple}{new ClickHandler() \{}@*)
      public void onClick(ClickEvent event) {
        addStock();
      }
    });
  }
  private void addStock() {
    ...
  }
}
\end{lstlisting}
        \vspace{-12pt}
        \caption{Complete Source Code in gwtproject.org}
        \label{fig:example3}
\end{figure}


\vspace{2pt}
\noindent {\bf Observation 2} [{\em Dependencies/Relations among API Elements in a Usage}].
The API elements used together in an API usage in certain combinations/orders share various program dependencies. These relationships can contribute to identifying the FQNs of the API elements better. For example, in Fig.~\ref{fig:example3}, we can see that to set a handler for a button in GWT, the object of the type \code{Button} needs to be the {\em receiving object of the method call} to \code{addClickHandler}, which in turn needs to accept an object of the type \code{ClickHandler} as an argument. The client code utilizing GWT library for this purpose will demonstrate the relationships shared between these three API elements as well. For example, in Fig.~\ref{fig:example2}, if \code{addClickHandler} on line 3 is determined to be the API element \code{com.\-google.\-gwt.\-user.\-client.\-ui.\-Button.\-add\-Click\-Handler}, the FQN of the element at line 4 must be \code{com\-.google\-.gwt\-.event\-.dom\-.client\-.ClickHandler}. The opposite direction of reasoning is also applicable. In general, if a model can learn the dependencies/relations among API elements in an API usage, it could leverage such knowledge to decide the FQNs of all API elements at the same time.

%the lines 2 and 3 in Fig. 2
As another example, consider the data dependency from the \code{def-use} relationship via the variable \code{myButton} between line~2 and line~3 in Fig.~\ref{fig:example2}. This relationship helps derive the FQNs for the above API elements. For instance, if a model decides the FQN for \code{Button} on line 2 to be \code{com\-.google\-.gwt\-.user\-.client\-.ui\-.Button}, it could consequently derive the FQN of \code{add\-Click\-Handler} on line~3 as \code{com.\-google.\-gwt.\-user.\-client.\-ui.\-Button.\-add\-Click\-Handler}, and vice versa.

\vspace{2pt}
\noindent {\bf State-of-the-Art Approaches.} Several approaches have been proposed to recover the fully-qualified names (FQNs) for the API elements in a code snippet. The {\em program-analysis-based} approaches (e.g., PPA~\cite{dagenais-oopsla08}, RecoDoc~\cite{dagenais-icse12}), {\em information-retrieval-based} approaches (e.g., Baker~\cite{liveapi14}, COSTER~\cite{coster-ase19}), and {\em constraint-based} approaches (e.g., SnR~\cite{snr-icse22}) are not comprehensive, and suffer from out-of-vocabulary failures. %(i.e., cannot derive FQNs that were not seen in the training corpus).

Employing the recent advances in {\em machine} and {\em deep learning} (ML \& DL) for FQN resolution has enabled the generation of new FQNs for API elements. However, the state-of-the-art ML/DL-based approaches (e.g., StatType~\cite{icse18} and Huang {\em et al.}~\cite{prompt-ase22}) {\bf do not yet leverage the regularity in API-usages and the dependencies among the relevant API elements} for FQN prediction. StatType~\cite{icse18} leverages phrase-based statistical machine translation to transform an API type-sequence without FQNs to one with them. However, with only short phrases of lengths between 3-8 tokens, StatType is not capable of capturing long-range dependencies between API elements. For example, in Fig.~\ref{fig:example1}, StatType misses the inter-connections between the statements spanning across 11 lines from \code{Button mButton} on line 1 to \code{mButton}, \code{findViewById}, etc. on line 8; and to \code{mButton}, \code{setOnClickListener} on line 11. Moreover, in some cases, it is possible for two relevant API elements to be even farther in the code snippet.

Huang {\em et al.}~\cite{prompt-ase22} formulated FQN resolution as a fill-in-the-blank problem and leveraged a masked language model (MLM\textsubscript{\textit{FIB}}) for this purpose. They build each training instance by taking the statement at each type inference point (e.g., \code{View} on line 12) and a couple of surrounding lines with unresolved API elements as context. This approach has several limitations. First, the amount of contextual information might not capture all relevant API elements corresponding to the same API-usage. In Fig.~\ref{fig:example1}, \code{mButton} on line 1 is far apart from \code{mButton} on line 8 and \code{mButton} on line 11. Next, individual API elements might be used in a different context in the client code. For example, the code on line 9 in Fig.~\ref{fig:example1} is specific to the method \code{onCreate}. Thus, such limited context might not provide the model with sufficient information for determining the FQN.


%Talk about StatType and fill-in


\subsection{Key Ideas}
\label{sec:key}
Following Observations 1--2 and the limitations in existing state-of-the-art approaches, we designed {\tool} to identify the FQNs in a code snippet with the following key ideas:

\vspace{2pt}
\noindent {\bf Key Idea 1} [{\em Leveraging Regularity of API Usages}].
We leverage the pattern-capturing abilities of deep learning models to implicitly learn co-occurring API elements in API usages and derive the FQNs. Observation 1 guides the first principle of our solution, which is the basis of regularity of API usages in the training corpus: the API elements (with their FQNs) appearing together more regularly in API usages have a higher impact on deciding the FQNs than the less regular ones. To facilitate this idea and ensure that the FQNs of all API elements in use are known, we utilize complete, compilable code using libraries from a large code corpus.

\vspace{2pt}
\noindent {\bf Key Idea 2} [{\em ``Tell Me Your Friends, I'll Tell You Who You Are''}].
We consider FQN resolution as a task of identifying inter-connected API elements in a given code snippet. As described in Observation 2, rather than attempting to determine the FQN of an API element based solely on its individual characteristics (e.g., local surrounding context) as in the state-of-the-art approach Huang {\em et al.}~\cite{prompt-ase22}, we adopt a broader dependency context that include program dependencies and inter-API relationships. Based on this knowledge, we aim to derive all related API elements simultaneously.

%We use a graph representation, called Augmented Usage Graph
%(AUG)~\cite{msr19}, to represent the program dependencies and
%relations among program entities and API elements. We extract the AUGs
%from the complete, compilable source code using the APIs from a large
%code corpus. We then enhance the AUGs with all the FQNs because the
%training code is compilable. From those AUGs,

\vspace{2pt}
\noindent {\bf Key Idea 3} [{\em FQN Resolution via Text Infilling}].
Deriving FQNs in an incomplete code snippet can be postulated as a fill-in-the-blank task, wherein the missing parts of a fully-qualified name are inserted ahead of its corresponding simple name at all relevant type inference locations. For example, in Fig.~\ref{fig:example3}, a blank representing \code{com.google.gwt.event.dom.client} will be inserted ahead of \code{ClickEvent}. To fill in all such inserted blanks, we adopt a Causal Language Model (CLM) capable of generating text and extend its capabilities to predict the missing spans of text in the blanks. Let us call this framework an Infilling Language Model (ILM). The blanks thus constructed are consistent with the preceding and subsequent code, thus enabling the ILM to gain access to the wider dependency context and learn from the interconnections among the API and program elements in the code snippet.


\section{Our Approach}
\label{sec:approach}

\subsection{Type Inference Location Extraction}
\subsubsection{AST Parsing} As a first step towards building \code{<blank>}-sequences, we need to build the AST for a given code snippet. In cases where it is complete, constructing an AST is trivial. In cases where it is incomplete, we can utilize tools such as PPA~\cite{} to build an AST in a best-effort manner.

\subsubsection{Node Transformation}

\subsubsection{AST Unparsing}



Next, we traverse the AST and transform the source code corresponding to each of the type-specific AST nodes as follows:

%\begin{enumerate}[label=\roman*.]
    \item \textbf{Array Creation}

    \textit{Formal Syntax:} \code{new TypeName [ < Type { , Type } > ] [ Expression ] {[ Expression ]} { [ ] }}

    \textit{Syntax:} \code{new TypeName [ < Type { , Type } > ] [ Expression ] {[ Expression ]} { [ ] }}
    
    \textit{Transformation:}    
    %%%%%%%%%%%%%%%%%%%%%%%%%%%%%%%%%%%%%%%%%%%%%%%%%%%%%%%%%    
    \item \textbf{Cast Expression}

%    \textit{Formal Syntax:} \code{( Type ) Expression}
 
    $(\mathcal{N}_{\mathcal{S}}) E \rightsquigarrow (\text{\code{<blank>}. }\mathcal{N}_{\mathcal{S}}) E$    
    %%%%%%%%%%%%%%%%%%%%%%%%%%%%%%%%%%%%%%%%%%%%%%%%%%%%%%%%%  
    \item \textbf{Class Instance Creation}

%    \textit{Formal Syntax:} \code{new [ < Type { , Type } > ] Type ( [ Expression { ,  Expression } ] )}

    $\text{\textbf{\code{new}} }\mathcal{N}_{\mathcal{S}}(E_1, E_2, ..., E_n) \rightsquigarrow\text{ \textbf{\code{new}} \code{<blank>}. }\mathcal{N}_{\mathcal{S}}(E_1, E_2, ..., E_n)$    
    %%%%%%%%%%%%%%%%%%%%%%%%%%%%%%%%%%%%%%%%%%%%%%%%%%%%%%%%%    
    \item \textbf{Instanceof Expression}

%    \textit{Formal Syntax:} \code{Expression instanceof Type}
    
    $E\text{ \textbf{\code{instanceof}} }(\mathcal{N}_{\mathcal{S}}) \rightsquigarrow E\text{ \textbf{\code{instanceof}} }( \text{\code{<blank>}. }\mathcal{N}_{\mathcal{S}})$
    %%%%%%%%%%%%%%%%%%%%%%%%%%%%%%%%%%%%%%%%%%%%%%%%%%%%%%%%%    
%    \item \textbf{Single Variable Declaration}
%
%    \textit{Syntax:} \code{{ ExtendedModifier } Type {Annotation} [ ... ] Identifier { Dimension } [ = Expression ]}
%    
%    \textit{Transformation:}
    %%%%%%%%%%%%%%%%%%%%%%%%%%%%%%%%%%%%%%%%%%%%%%%%%%%%%%%%%    
    \item \textbf{(Super) Constructor Invocation}

%    \textit{Formal Syntax:} \code{< this | super > ( [ Expression { , Expression } ] ) ;}

    $\{\text{\textbf{this}}\vert\text{\textbf{super}\}}(E_1, E_2, ..., E_n) \rightsquigarrow \text{\code{<blank>} }(E_1, E_2, ..., E_n)$    
    %%%%%%%%%%%%%%%%%%%%%%%%%%%%%%%%%%%%%%%%%%%%%%%%%%%%%%%%%    
    \item \textbf{(Super) Field Access}

%    \textit{Syntax:} \code{Expression . Identifier}

%    \textit{Syntax:} \code{super . Identifier}

    $\{E\vert\text{\textbf{super}\}}.I \rightsquigarrow \text{\code{<blank>}. }I$
    %%%%%%%%%%%%%%%%%%%%%%%%%%%%%%%%%%%%%%%%%%%%%%%%%%%%%%%%%    
    \item \textbf{(Super) Method Invocation}

%    \textit{Syntax:} \code{Identifier ( [ Expression { , Expression } ] )}

%    \textit{Syntax:} \code{super . Identifier ( [ Expression { , Expression } ] )}
    
    $\{I\vert\text{\textbf{super}.}I\}(E_1, E_2, ..., E_n) \rightsquigarrow \text{\code{<blank>}. }I(E_1, E_2, ..., E_n)$
    %%%%%%%%%%%%%%%%%%%%%%%%%%%%%%%%%%%%%%%%%%%%%%%%%%%%%%%%%    
    \item \textbf{Throw Statement}

%    \textit{Syntax:} \code{throw Expression ;}
    
    $\text{\textbf{\code{throw }}}\mathcal{N}_{\mathcal{S}} \rightsquigarrow \text{\textbf{\code{throw }}\code{<blank>}. }\mathcal{N}_{\mathcal{S}}$
    %%%%%%%%%%%%%%%%%%%%%%%%%%%%%%%%%%%%%%%%%%%%%%%%%%%%%%%%%    
\end{enumerate}
% Please add the following required packages to your document preamble:
% \usepackage{multirow}
\begin{table*}[]
\centering
\begin{tabular}{l|c}
\toprule
\multicolumn{1}{c|}{\textbf{Node Type}}                               & \textbf{Transformation Rule and Example}                                                                                                                                                              \\ \hline
\multirow{3}{*}{Array Creation}                 & \cellcolor{gray!15} $\text{\tabcode{\textbf{new }}}\mathcal{N}_\mathcal{S}[E] \:\rightsquigarrow\: \text{\tabcode{\textbf{new }<blank>.}}\mathcal{N}_\mathcal{S}[E]$  \\
                                                & \multicolumn{1}{l}{\textit{Example}: \tabcode{new Context[contexts.size()]} $\:\rightsquigarrow\:$ \tabcode{new <blank>.Context[contexts.size()]}} \\ 
                                                & \multicolumn{1}{l}{\textit{FQN}: \tabcode{org.xml.sax.helpers.NamespaceSupport}} \\ \hline
\multirow{3}{*}{Cast Expression}                & \cellcolor{gray!15} $(\mathcal{N}_{\mathcal{S}}) E \:\rightsquigarrow\: (\text{\tabcode{<blank>}. }\mathcal{N}_{\mathcal{S}}) E$                                                                                \\
                                                & \multicolumn{1}{l}{\textit{Example}: \tabcode{(LexicalHandler) value} $\:\rightsquigarrow\:$ \tabcode{(<blank>.LexicalHandler) value}} \\ 
                                                & \multicolumn{1}{l}{\textit{FQN}: \tabcode{org.xml.sax.ext}} \\ \hline
\multirow{2}{*}{Class Instance Creation}        & \cellcolor{gray!15} $\text{\textbf{\tabcode{new}} }\mathcal{N}_{\mathcal{S}}(E_1, E_2, ..., E_n) \:\rightsquigarrow\:\text{ \textbf{\tabcode{new}} \tabcode{<blank>}. }\mathcal{N}_{\mathcal{S}}(E_1, E_2, ..., E_n)$ \\
                                                & \multicolumn{1}{l}{\textit{Example}: \tabcode{new Context()} $\:\rightsquigarrow\:$ \tabcode{new <blank>.Context()}} \\ 
                                                & \multicolumn{1}{l}{\textit{FQN}: \tabcode{org.xml.sax.helpers.NamespaceSupport}} \\ \hline
\multirow{3}{*}{Instanceof Expression}          & \cellcolor{gray!15} $E\text{ \textbf{\tabcode{instanceof}} }\mathcal{N}_{\mathcal{S}} \:\rightsquigarrow\: E\text{ \textbf{\tabcode{instanceof}} } \text{\tabcode{<blank>}. }\mathcal{N}_{\mathcal{S}}$           \\
                                                & \multicolumn{1}{l}{\textit{Example:} \tabcode{addr instanceof Inet6Address} $\:\rightsquigarrow\:$ \tabcode{addr instanceof <blank>.Inet6Address}} \\ 
                                                & \multicolumn{1}{l}{\textit{FQN}: \tabcode{java.net}}\\\hline
\multirow{3}{*}{Single Variable Declaration}    & \cellcolor{gray!15}  $\mathcal{N}_\mathcal{S}\text{ }I\:\{\:= E\} \:\rightsquigarrow\:$ $\text{\tabcode{<blank>}. }\mathcal{N}_\mathcal{S}\text{ }I\:\{\:= E\}$\\
                                                & \multicolumn{1}{l}{\textit{Example}: \tabcode{Node<E> x} $\:\rightsquigarrow\:$ \tabcode{<blank>.Node<E> x}} \\ 
                                                & \multicolumn{1}{l}{\textit{FQN}: \tabcode{java.util.concurrent.LinkedBlockingDeque}} \\ \hline
\multirow{3}{*}{(Super) Constructor Invocation} & \cellcolor{gray!15} $\{\text{\textbf{\tabcode{this}} }\vert\text{ \textbf{\tabcode{super}}\}}(E_1, E_2, ..., E_n) \:\rightsquigarrow\: \text{\tabcode{<blank>} }(E_1, E_2, ..., E_n)$                                                 \\
                                                & \multicolumn{1}{l}{\textit{Example}: \tabcode{this(address, true)} $\:\rightsquigarrow\:$ \tabcode{<blank>(address, true)}} \\ 
                                                & \multicolumn{1}{l}{\textit{FQN}: \tabcode{org.apache.harmony.tests.java.net.DatagramSocketTest.DatagramServer}}\\ \hline
\multirow{3}{*}{(Super) Field Access}           & \cellcolor{gray!15} $\{E\text{ }\vert\text{ \textbf{\tabcode{super}}\}}.I \:\rightsquigarrow\: \text{\tabcode{<blank>}. }I$                                                                                                        \\
                                                & \multicolumn{1}{l}{\textit{Example}: \tabcode{this.changeConfig} $\:\rightsquigarrow\:$ \tabcode{<blank>.changeConfig}}               \\ 
                                                & \multicolumn{1}{l}{\textit{FQN}: \tabcode{android.compat.Compatibility.OverrideCallbacks}} \\ \hline
\multirow{3}{*}{(Super) Method Invocation}      & \cellcolor{gray!15} $\{I\text{ }\vert\text{\textbf{ \tabcode{super}}.}I\}(E_1, E_2, ..., E_n) \:\rightsquigarrow\: \text{\tabcode{<blank>}. }I(E_1, E_2, ..., E_n)$                                                                \\
                                                & \multicolumn{1}{l}{\textit{Example}: \tabcode{connectLocalServer()} $\:\rightsquigarrow\:$ \tabcode{<blank>.connectLocalServer()}}      \\ 
                                                & \multicolumn{1}{l}{\textit{FQN}: \tabcode{org.apache.harmony.tests.java.nio.channels.DatagramChannelTest}}\\
% \multirow{2}{*}{Throw Statement}                & \cellcolor{gray!15} $\text{\textbf{\tabcode{throw }}}E \:\rightsquigarrow\: \text{\textbf{\tabcode{throw }}\tabcode{<blank>}. }\mathcal{N}_{\mathcal{S}}$                                     \\
%                                                 & \multicolumn{1}{l}{\textit{Example}: \tabcode{} $\:\rightsquigarrow\:$ \tabcode{}} \\ 
%                                                 & \multicolumn{1}{l}{\textit{FQN}: \tabcode{}} \\
\bottomrule
\end{tabular}
\caption{AST node-level transformation rules for constructing \tabcode{<build>}-sequences in Type Inference Location Extraction (TILE) algorithm. Here, $\mathcal{N}_\mathcal{S}$ denotes \textit{Simple Name}, $E_i$ denotes \textit{Expression}, and $I$ denotes \textit{Identifier}.}
\end{table*}

\subsection{Type Inference with Infilling Language Model}
\subsubsection{Problem Formulation}

\subsubsection{Training Process}

\subsubsection{Inference}



\section{Empirical Evaluation}
\label{sec:evaluation}

%\subsection{Research Questions}

We conducted several experiments to evaluate {\tool}. We aim to answer the following questions:

\vspace{2pt}
\noindent \textbf{RQ\textsubscript{1} 
  [Effectiveness Evaluation]} {\em How accurate is {\tool} in expanding FQNs for complete code from Java projects?}

\vspace{2pt}
\noindent \textbf{RQ\textsubscript{2}
[API-Usage Relations]} {\em Do the conditioning-driven interactions help the model learn API usages?}

\vspace{2pt}
\noindent \textbf{RQ\textsubscript{3} 
[Ablation Study]}  {\em Is there a benefit to leveraging a wider dependency context in comparison to a narrower surrounding context for predicting fully qualified names?}

%{\em Is our tool capable of capturing the API-usage relations between API elements and other program elements in the code snippet?}

% \vspace{2pt}
% \noindent \textbf{RQ\textsubscript{4} 
% [Practicality Evaluation]}  {\em How accurate is {\tool} in 
% inferring FQNs for API elements in incomplete code snippets from online forums such as StackOverflow?}

%\subsection{Empirical Methodology}

\subsubsection{Datasets}

\subsubsection{RQ1.}

{\em Baselines.}

{\em Procedure.}

{\em Tuning.}

{\em Metrics.}

\subsubsection{RQ2.}

{\em Baselines.}

{\em Procedure.}

{\em Tuning.}

{\em Metrics.}



\section{Effectiveness Evaluation}
\label{sec:effectiveness-eval}

\subsection{Data Collection}\label{sec:effectiveness-data}

% Please add the following required packages to your document preamble:
% \usepackage{multirow}
\begin{table}[t]
\centering
%\scriptsize
\begin{tabular}{c|cc|cc|cc}
\toprule
%\multirow{2}{*}{\diagbox{\textbf{Project} ($\downarrow$)}{\textbf{Split} ($\rightarrow$)}}                 & \multicolumn{2}{c|}{\textbf{Train}}         & \multicolumn{2}{c|}{\textbf{Validation}}    & \multicolumn{2}{c}{\textbf{Test}}           \\ \cline{2-7}
% \multirow{2}{*}{\diagbox{\textbf{Project}}{\textbf{Split}}}                 & \multicolumn{2}{c|}{\textbf{Train}}         & \multicolumn{2}{c|}{\textbf{Validation}}    & \multicolumn{2}{c}{\textbf{Test}}           \\ \cline{2-7}
\multicolumn{1}{r|}{\textbf{Split} ($\rightarrow$)}    & \multicolumn{2}{c|}{\textbf{Train}}         & \multicolumn{2}{c|}{\textbf{Validation}}    & \multicolumn{2}{c}{\textbf{Test}}           \\ \cline{2-7}
\multicolumn{1}{l|}{\textbf{Library} ($\downarrow$)}   & \multicolumn{1}{c|}{$N_x$} & $N$\textsubscript{API} & \multicolumn{1}{c|}{$N_x$} & $N$\textsubscript{API} & \multicolumn{1}{c|}{$N_x$} & $N$\textsubscript{API} \\
\hline
\code{android}   & \multicolumn{1}{c|}{2,389}             &  10,878       & \multicolumn{1}{c|}{298}             & 1,273        & \multicolumn{1}{c|}{300}             &  1,285       \\
\tabcode{gwt}       & \multicolumn{1}{c|}{33,954}             & 191,946        & \multicolumn{1}{c|}{4,244}             &  23,991       & \multicolumn{1}{c|}{4,245}             & 23,937        \\
\code{hibernate} & \multicolumn{1}{c|}{52,910}             & 171,496        & \multicolumn{1}{c|}{6,613}             & 22,173        & \multicolumn{1}{c|}{6,615}             & 22,074        \\
\code{jdk}       & \multicolumn{1}{c|}{3,124}             & 11,994        & \multicolumn{1}{c|}{390}             &  1,339       & \multicolumn{1}{c|}{392}             & 1,398        \\
\code{joda-time} & \multicolumn{1}{c|}{4,976}             &  13,396       & \multicolumn{1}{c|}{622}             &  1,754       & \multicolumn{1}{c|}{623}             & 1,606        \\
\code{xstream}   & \multicolumn{1}{c|}{2,774}             &  7,909       & \multicolumn{1}{c|}{346}             &  1,120       & \multicolumn{1}{c|}{348}             &  995       \\ \hline
\textbf{Total}   & \multicolumn{1}{c|}{100,127}             & 407,619        & \multicolumn{1}{c|}{12,513}             & 51,650        & \multicolumn{1}{c|}{12,523}             & 51,395        \\ 
\bottomrule
\end{tabular}
\caption{Dataset statistics. $N_x$: the number of data instances; $N$\textsubscript{API}: the number of API elements.}
\label{tab:data-stats}
\end{table}

%% Total instances: android - 2987, gwt - 42443, hibernate-orm 66138, jdk 3906, joda-time 6221, xstream 3468
%% Total APIs: android - 13536, gwt - 239874, hibernate-orm 215743, jdk 14931, joda-time 16756, xstream 10024
%% Unique APIs: android - 569, gwt - 5949, hibernate - 12618, jdk - 1480, joda-time - 401, xstream - 643


To enable the effectiveness evaluation, we considered six popular Java libraries that developers typically make use of: \code{android}, \code{gwt}, \code{hibernate}, \code{jdk}, \code{joda-time}, and \code{xstream}. This is consistent with prior works on FQN resolution~\cite{icse18, snr-icse22, prompt-ase22}. Among these, StatType\cite{icse18} and SnR\cite{snr-icse22} leverage projects that utilize these libraries. A limitation with this approach is that not all library API elements are covered in their dataset. In contrast, we download these large-scale repositories from GitHub and use their source code itself to build our dataset.

We used Eclipse JDT to parse each project’s source code and resolve all the FQNs. Next, we traversed the ASTs to collect all the methods. Overall, these libraries contain 198,231 methods including 9,764 from \code{android}, 62,782 from \code{gwt}, 105,734 from \code{hibernate-orm}, 5,235 from \code{jdk}, 9,811 from \code{joda-time}, and 4,905 from \code{xstream}. Next, we randomly split these in a 80\%-10\%-10\% ratio, each for training, validation, and testing, respectively. Finally, we applied the TILE algorithm to identify the type inference locations and build the $\langle$\blank-code, FQN$\rangle$ pairs, discarding the methods with no such pairs. In general, we have 125,163 instances carrying 510,664 FQNs. Among these, 58,543 API elements carry unique FQNs, and 21,660 carry unique FQN prefixes (i.e., unresolved part of the FQN). We report the fine-grained library/split-level data statistics in Table~\ref{tab:data-stats}.

\subsection{Procedure \& Evaluation Metrics}\label{sec:effectiveness-eval-proc}
We design this experiment to test the effectiveness of \tool in resolving FQNs in complete code. For all experiments, we train the same CLM, i.e., GPT-2 “small” (containing 124M parameters) within the ILM framework.

\noindent\underline{\textit{Baselines}}: Due to the formulation of our approach as a fill-in-the-blank task, we can establish a direct comparison with the following: (a) off-the-shelf CodeBERT model (MLM\textsubscript{\textit{base}})~\cite{codebert}, (b) prompt-tuned CodeBERT model (MLM\textsubscript{\textit{FIB}})~\cite{prompt-ase22}. Besides, Huang et al. reported that reserving 30\% of the data for the prompt-tuning phase gives them stable model performance. Accordingly, for MLM\textsubscript{\textit{FIB}}, we reserve 30\% of our training data for prompt-tuning, and remaining 70\% for fine-tuning.

\noindent\underline{\textit{Data Preparation for Baselines}}: We followed the FQN prompt generation steps detailed in \cite{prompt-ase22} to build the training data for both prompt-tuning and fine-tuning phases in MLM\textsubscript{\textit{FIB}}. Due to the code fragmentation step in MLM\textsubscript{\textit{FIB}}, each training instance with $k$ type inference points in our dataset (i.e., at the method-level) corresponds to $k$ prompts/training instances for this baseline. Note that whole FQNs in a prompt are masked during the prompt-tuning phase, while the FQN prefixes are masked in the fine-tuning phase.

\noindent\underline{\textit{FQN Resolution with Baselines}}: In MLM\textsubscript{\textit{base}}, there is no additional training required since we use CodeBERT off-the-shelf. In MLM\textsubscript{\textit{FIB}}, CodeBERT is first prompt-tuned, then fine-tuned for enabling FQN resolution. An MLM is not designed to handle variable lengths of missing spans, and needs to know the length of the FQN prefix (i.e., the number of sub-tokens it corresponds to) beforehand. However, this is presumably not available during the inference phase. Huang et al. address this issue by producing multiple prompts with the lengths of FQNs ranging from the minimum mask span (i.e., 2) and the maximum mask span (i.e., 52), and select the FQN prefix prediction with the highest average probability. We followed suit to establish the inference process with both the baselines.

\noindent\underline{\textit{Evaluation Metrics}}: We use the following metrics to assess the model performance: (a) \textit{Exact Match Accuracy} (Accuracy\textsubscript{EM}), a stricter version of accuracy which ensures that the predicted FQN is exactly the same as the ground-truth FQN; (b) \textit{ROUGE-L}\footnote{Given ground-truth FQN $X$ and predicted FQN $Y$ of lengths $x$ and $y$, respectively: Recall, $R_{LCS}=\frac{LCS(X, Y)}{x}$, Precision, $P_{LCS}=\frac{LCS(X, Y)}{y}$, and F-measure, $F_{LCS}=\frac{2.R_{LCS}.P_{LCS}}{R_{LCS}+P_{LCS}}$.}, which compares the similarity between the predicted FQN and the ground-truth FQN based on the longest common subsequence (LCS); and (c) \textit{BLEU-2}, which compares the similarity between the predicted FQN and the ground-truth FQN based on bigrams in the subword tokens.

\subsection{RQ1... }
\label{sec:rq1}



\section{Investigating API-Usage Relation Learning }
\label{sec:eval}

% \subsection{Background, and Data Collection}
% \subsection{Experiment Methodology}
% \subsection{Evaluation Metrics}
%\subsection{Experiment Results (RQ\textsubscript{3})}
\label{sec:rq3}

% Please add the following required packages to your document preamble:
% \usepackage{multirow}
\begin{table}[]
\centering
\begin{tabular}{l|ccc}
\toprule
\multirow{2}{*}{\textbf{Approach}} & \multicolumn{3}{c}{\textbf{Metrics}}                                  \\ \cline{2-4} 
                                   & \multicolumn{1}{c|}{\textit{Accuracy\textsubscript{EM}}} & \multicolumn{1}{c|}{\textit{ROUGE-L}} & \textit{BLEU-2} \\ \hline
\multicolumn{1}{c|}{\textit{w/o dep. context}}      & \multicolumn{1}{c|}{0.24}         & \multicolumn{1}{c|}{0.58}        &  0.56 \\
\multicolumn{1}{c|}{\textit{w/ dep. context}}       & \multicolumn{1}{c|}{0.72}         & \multicolumn{1}{c|}{0.90}        &  0.91 \\ \bottomrule
\end{tabular}
\caption{Ablation Study.}
\end{table}

Within the infilling language modeling framework, {\tool} uses GPT-2, a causal language model (CLM) behind the scenes. While the main goal of an CLM is to predict the next-token, another mechanism they take advantage of to learn the various interactions between the inputs is {\em conditioning}. In our task, the CLM is conditioned on all the program elements in the input, as well as the \blank tokens representing the API elements. During the prediction, every subsequent prediction is also conditioned on the previously predicted FQN. We hypothesize that these interactions in conjunction with those with the program elements help GPT-2 learn API-usages.

In this regard, we stratified the test set in Section~\ref{sec:effectiveness-eval} based on the number of API elements ($N$) in the input for which FQNs are to be predicted. In Table~\ref{tab:strat-eval}, we list these strata. For comparing the model performance on these data subsets, we consider the same evaluation metrics as in Section~\ref{sec:effectiveness-eval-proc}.

In Table~\ref{tab:strat-eval}, we can see that the model achieves the worst \textit{Accuracy\textsubscript{EM}} for $N=1$. This could possibly be a consequence of the missing API-API interactions in these examples in spite of the presence of the contextual information. Next, we can see that we achieve the highest performance for $N=3$ and $N=4$, following which the performance slightly drops. This could be the effect of the greedy beam search algorithm utilized for this purpose, which tends to worsen the quality of the prediction as the length increases. Based on these observations, we can see that the conditioning in CLM prediction directly influences the API-usages learned by our model.

% Please add the following required packages to your document preamble:
% \usepackage{multirow}
\begin{table}[]
\centering
\begin{tabular}{l|ccc}
\toprule
\multirow{2}{*}{\textbf{Approach}} & \multicolumn{3}{c}{\textbf{Metrics}}                                  \\ \cline{2-4} 
                                   & \multicolumn{1}{c|}{\textit{Accuracy\textsubscript{EM}}} & \multicolumn{1}{c|}{\textit{ROUGE-L}} & \textit{BLEU-2} \\ \hline
\multicolumn{1}{c|}{\textit{w/o dep. context}}      & \multicolumn{1}{c|}{0.24}         & \multicolumn{1}{c|}{0.58}        &  0.56 \\
\multicolumn{1}{c|}{\textit{w/ dep. context}}       & \multicolumn{1}{c|}{0.72}         & \multicolumn{1}{c|}{0.90}        &  0.91 \\ \bottomrule
\end{tabular}
\caption{Ablation Study.}
\end{table}

\subsection{Case Studies}

\subsubsection{Example 1}

\begin{figure}[t]
	\centering
	\lstset{
		numbers=left,
		numberstyle= \tiny,
		keywordstyle= \color{blue!70},
		commentstyle= \color{red!50!green!50!blue!50},
		frame=shadowbox,
		rulesepcolor= \color{red!20!green!20!blue!20} ,
		xleftmargin=1.5em,xrightmargin=0em, aboveskip=1em,
		framexleftmargin=1.5em,
                numbersep= 5pt,
		language=C,
    basicstyle=\scriptsize\ttfamily,
    numberstyle=\scriptsize\ttfamily,
    emphstyle=\bfseries,
                moredelim=**[is][\color{red}]{@}{@},
		escapeinside= {(*@}{@*)}
	}
\begin{lstlisting}[]
private static boolean isFinalSigma(java.lang.String s, int index) {
  if(index <= 0) {
    return false;
  }
  char previous = s.charAt(index - 1);
  if(!(java.lang.Character.isLowerCase(previous) || java.lang.Character.isUpperCase(previous) || java.lang.Character.isTitleCase(previous))) {
    return false;
  }
  if(index + 1 >= java.lang.String.length()) {
    return true;
  }
  ...
}
\end{lstlisting}
        \vspace{-12pt}
        \caption{An Example of Correct FQN Resolution}
        \label{fig:ex1}
\end{figure}

Fig.~\ref{fig:ex1} displays an example that {\tool} correctly resolved
the FQNs for the API elements. For convenience, we listed the code
with all the FQNs of the API elements. As seen, {\tool} is able to
recognize the API elements from the \code{java.lang} of the JDK
library. For example, the formal type \code{String} of \code{s}, the
FQNs of the method calls \code{isLowerCase}, \code{isUpperCase}, and
\code{isTitleCase}, and the FQN of the method call \code{length}.
Note that, \code{length} is a very popular simple name that matches
with the prefixes of several API elements.

\subsubsection{Example 2}





\section{Related Work}
\label{sec:related}

The approaches to recover the FQNs for the API elements in incomplete
code snippets can be broadly classified into the following categories.
The first category is {\bf program analysis}. Partial Program Analysis
(PPA)~\cite{dagenais-oopsla08} resolves the types by applying a set of
heuristics on the syntactic constructs to infer the declared
types. RecoDoc~\cite{dagenais-icse12} uses PPA to infer links between
APIs and documentation. It requires all the target libraries to be
known. The second category of approaches leverages {\bf
  information retrieval}. Baker~\cite{liveapi14} tracks the scopes of
the names and the candidate lists are combined according to the
scoping rules and get smaller. COSTER~\cite{coster-ase19} formulates
the FQN problem as searching. It aims to match the context of the
query API element with the FQNs in the database regarding three
criteria on likelihood, context similarity, and name similarity.  The
third category is {\bf constraint-based}. Differing from COSTER,
SnR~\cite{snr-icse22} builds the knowledge of the constraints on API
elements and extracts such constraints in the code snippet to match
them against the knowledge base, and then solve them to derive the
FQNs.

Advances in machine learning (ML) enables the approaches to learn to
derive FQNs. StatType~\cite{icse18} learns to translate the code
without FQNs into the one having them. It treats source code as texts
and uses phrase-based statistical machine translation. Thus, it does
not consider program dependencies in deciding FQNs.

The closely related work to {\tool} is from Huang {\em et
  al.}~\cite{prompt-ase22}, which fine-tunes a masked language model
(MLM) as a neural knowledge base of program elements with a
``pre-train, prompt and predict'' paradigm from source code. In
comparison, there are key advances from {\tool}. First, Huang {\em et
  al.}~\cite{prompt-ase22} leverages the code before and after the
prediction point in the prompt-based paradigm to identify the API
element and decide the FQN. In contrast, we rely on the program
dependencies and relations among the API elements to derive their
FQNs. Second, while {\tool} uses the MLM to derive the FQNs for all
the API elements all at once, their approach works at each API element
one at a time, thus, do not leverage the constraints/relations among
them. Finally, their approach performs a prediction by attempting the
FQN with different lengths from the smallest to the largest in the
corpus. In practice, those values are not available. {\tool}
addresses that by an advanced technique.




\section{Conclusion}
\label{sec:conclusion}

In this work, we present {\tool}, a deep learning-based approach that
addresses these issues by formulating FQN resolution as a
\textit{fill-in-the-blank} task. For a given code snippet, our tool
systematically identifies all API locations, inserts a blank ahead of
the simple name corresponding to the API, leverages a causal language
model (CLM) to fill the missing blank with the corresponding FQN.  Our
empirical evaluation shows relatively improves over the
state-of-the-art FQN resolution approaches by 9.1\% on real-world code
snippets. Upon further analysis, we were able to tie the performance
improvement to our tool's capabilities to capture dependencies between
the API and other program elements in the code snippet.


\section{Data Availability}

The model and data is available in our website~\cite{deepfqn}.


In this work, we present {\tool}, a deep learning-based approach that
addresses these issues by formulating FQN resolution as a
\textit{fill-in-the-blank} task. For a given code snippet, our tool
systematically identifies all API locations, inserts a blank ahead of
the simple name corresponding to the API, leverages a causal language
model (CLM) to fill the missing blank with the corresponding FQN.  Our
empirical evaluation shows relatively improves over the
state-of-the-art FQN resolution approaches by 9.1\% on real-world code
snippets. Upon further analysis, we were able to tie the performance
improvement to our tool's capabilities to capture dependencies between
the API and other program elements in the code snippet.

%\section*{Acknowledgments}
%This work was supported in part by the US National Science Foundation
%(NSF) grants CNS-2120386.

%\newpage
\balance

\bibliographystyle{IEEEtran}

\bibliography{references, References-fse23}


\end{document}
\endinput
%%
%% End of file `sample-acmsmall-conf.tex'.
