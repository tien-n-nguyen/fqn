\documentclass[conference]{IEEEtran}
\IEEEoverridecommandlockouts
% The preceding line is only needed to identify funding in the first footnote. If that is unneeded, please comment it out.
\usepackage{cite}
\usepackage{amsmath,amssymb,amsfonts}
\usepackage{algorithmic}
\usepackage{graphicx}
\usepackage{textcomp}
\usepackage{xcolor}
\def\BibTeX{{\rm B\kern-.05em{\sc i\kern-.025em b}\kern-.08em
    T\kern-.1667em\lower.7ex\hbox{E}\kern-.125emX}}

%\usepackage{amsmath,amssymb,amsfonts}
%\usepackage{algorithmic}
%\usepackage{graphicx}
%\usepackage{textcomp}
%\usepackage{xcolor}


%\usepackage{cite}
\usepackage{colortbl}
\usepackage{booktabs}   %% For formal tables:
                        %% http://ctan.org/pkg/booktabs
\usepackage{subcaption} %% For complex figures with subfigures/subcaptions
                        %% http://ctan.org/pkg/subcaption
\usepackage{array}
\usepackage{amsmath,amsfonts}
\usepackage{amssymb}
%\usepackage{algorithm}
%\usepackage[noend]{algpseudocode}
%\usepackage{algorithmic}
%\usepackage{graphicx}
%\usepackage{textcomp}
\usepackage{float}
\usepackage{listings}
\usepackage{xspace}
\usepackage{multirow}
\usepackage{amsthm}
\usepackage{enumerate}
\usepackage{enumitem}

\newtheorem{definition}{Definition}
\usepackage{balance}
\usepackage{printlen}
\usepackage[skins]{tcolorbox}
\usepackage{color, soul}

%\usepackage{xcolor,pifont}
%\newcommand*\colourcheck[1]{%
%	\expandafter\newcommand\csname #1check\endcsname{\textcolor{#1}{\ding{52}}}%
%}
%\colourcheck{blue}
%\colourcheck{green}
%\colourcheck{red}

\newtcolorbox{myframe}[2][]{%
  enhanced,colback=white,colframe=black,coltitle=black,
  sharp corners,
  toprule=1.0pt,
  rightrule=0.3pt,
  leftrule=0pt,
  bottomrule=0pt,
  fonttitle=\itshape\scshape\large,
  left=0pt,right=5pt,top=5pt,bottom=3pt,
  attach boxed title to top right={yshift=-0.3\baselineskip-0.4pt,xshift=-5mm},
  boxed title style={tile,size=minimal,left=0.2mm,right=0.5mm,
    colback=white,before upper=\strut},
  title=#2,#1
}

\newenvironment{nscenter}
 {\parskip=0pt\par\nopagebreak\centering}
 {\par\noindent\ignorespacesafterend}

%\newcommand{\code}[1]{{\footnotesize\textsf{#1}}}

\newcommand{\tool}{\textsc{DeepFQN}\xspace}
\newcommand{\blank}{\begin{footnotesize}\textsf{ \textbf{[blank]}\ }\end{footnotesize}}

\newtheorem{Definition}{Definition}
\newtheorem{Claim}{Claim}
\newtheorem{Lemma}{Lemma}
\newtheorem{Theorem}{Theorem}

\newcolumntype{L}[1]{>{\raggedright\arraybackslash}p{#1}}
\newtheorem{Observation}{Observation}
\newtheorem{property}{Property}
\newcommand{\code}[1]{{\footnotesize\texttt{#1}}}
\newcommand{\tabcode}[1]{{\scriptsize\texttt{#1}}}
\usepackage{amsthm}
 \definecolor{dkgreen}{rgb}{0,0.6,0}
\definecolor{gray}{rgb}{0.5,0.5,0.5}
\definecolor{mauve}{rgb}{0.58,0,0.82}
\lstset{frame=tb,
  language=Java,
  aboveskip=3mm,
  belowskip=3mm,
  showstringspaces=false,
  columns=flexible,
  basicstyle={\small\ttfamily},
  numbers=left,
  numberstyle=\tiny\color{gray},
  keywordstyle=\color{blue},
  commentstyle=\color{dkgreen},
  stringstyle=\color{mauve},
  breaklines=true,
  breakatwhitespace=true,
  tabsize=4
}


%\usepackage{tikz}
%\usetikzlibrary{shapes.arrows}
%\newcommand{\FancyUpArrow}{\begin{tikzpicture}[baseline=-0.3em]
%		\node[single arrow,draw,rotate=90,single arrow head extend=0.1em,inner
%		ysep=0.1em,transform shape,line width=0.03em,top color=green,bottom color=green!50!black] (X){};
%\end{tikzpicture}}

%\def\BibTeX{{\rm B\kern-.05em{\sc i\kern-.025em b}\kern-.08em
%    T\kern-.1667em\lower.7ex\hbox{E}\kern-.125emX}}


\begin{document}

%\title{Conference Paper Title*\\
%{\footnotesize \textsuperscript{*}Note: Sub-titles are not captured in Xplore and
%should not be used}
%\thanks{Identify applicable funding agency here. If none, delete this.}
%}

%\makeatletter
%\newcommand{\linebreakand}{%
%\end{@IEEEauthorhalign}
%\hfill\mbox{}\par
%\mbox{}\hfill\begin{@IEEEauthorhalign}
%}
%\makeatother

%\title{Neural Fully-Qualified Name Resolution}
\title{Enabling Fully-Qualified Name Resolution of API Elements via Generative Text Infilling}

%\author{\IEEEauthorblockN{Aashish Yadavally and Tien N. Nguyen}
%\IEEEauthorblockA{\textit{Computer Science Department} \\
%\textit{The University of Texas at Dallas}\\
%Texas, USA \\
%\{aashish.yadavally, tien.n.nguyen\}@utdallas.edu}
%\and
%\IEEEauthorblockN{Wenbo Wang and Shaohua Wang}
%\IEEEauthorblockA{\textit{Department of Informatics} \\
%\textit{New Jersey Institute of Technology}\\
%New Jersey, USA \\
%ww6@njit.edu, davidshwang@ieee.org}
%}

\maketitle

\begin{abstract}

Software development relies heavily on the effective use of API elements, which provide pre-built functionalities for specific programming tasks. Online forums, e.g., StackOverflow and GitHub Gists contain vast collections of code snippets that utilize these APIs. However, due to the incomplete nature of these code snippets, determining the correct fully-qualified name (FQN) for API elements is challenging. Several approaches have been proposed to automatically resolve the FQNs, but face one or more of the following issues: (1) are not comprehensive; (2) can not handle out-of-vocabulary API elements; (3) do not consider dependence of the API element on other program elements. In this work, we  present {\tool}, a deep learning-based approach that addresses these issues by formulating FQN resolution as a \textit{fill-in-the-blank} task. For a given code snippet, our tool systematically identifies all API locations, inserts a blank ahead of the simple name corresponding to the API, leverages a causal language model (CLM) to fill the missing blank with the corresponding FQN.
Our empirical evaluation shows relatively improves over the state-of-the-art FQN resolution approaches by x\% on complete, and y\% on real-world code snippets. Upon further analysis, we were able to tie the performance improvement to our tool's capabilities to capture dependencies between the API and other program elements in the code snippet.  

\end{abstract}

%\begin{IEEEkeywords}
%neural partial program analysis, neural program dependence analysis, neural networks; deep learning
%\end{IEEEkeywords}

\section{Introduction}
\label{sec:intro}
External software libraries play a crucial role in software development. They provide functionalities to accomplish specific programming tasks via Application Programming Interface (API) elements such as classes, methods, and fields. To learn how to reuse them, developers often turn to API documentation and online forums, e.g., StackOverflow (S/O) or GitHub Gists and search for concrete API usage examples. However, integrating API elements in this manner is not straightforward. First, the documentation can vary in quality and clarity. As a result, the developers might need to spend more time understanding the API usage and how it fits into their codebase. Next, the code snippets in online forums carrying API usages are interspersed between conversational contexts and are typically incomplete. Thus, they might include multiple references to unavailable/unresolved API elements, making it unexecutable and further ambiguous to understand.

To work with partial code while ensuring the correct reference/usage of its API elements, the resolution of their fully qualified names (FQNs) is mandatory. However, due to the lack of import statements and library dependencies, such references might be ambiguous and the information needed for FQN resolution might not be readily available. For example, consider an incomplete code snippet that references \code{ParseException} to signal that an error has unexpectedly been reached while parsing. This API element can correspond to either \code{java.text.ParseException} in the \code{jdk} library, or \code{android.net.ParseException} in the \code{android} library. 

To correctly resolve the FQN of an API element, one might require a thorough understanding of its usage, the context in which it is supposed to be used, and its potential dependencies in the external code. For example, consider the code snippet in Figure~\ref{fig:excerpt-example1} inspired from the S/O post \#34595450. The \code{getText} method call on line 5 is a very popular API name that occurs in multiple libraries. Nguyen et al.~\cite{icse18} reported that there are about 500 API methods of that name across five different Java libraries. Consider the data dependency between line 3 and line 5 on account of the variable \code{cssRes}. We can see that the return type of \code{css(...)} on line 3 has a member API method with the name \code{getText}. With this knowledge, the number of candidates for \code{getText} can be brought down to 4. Next, consider the data dependency between line 5 and line 7 via the variable \code{s}. We can see that the API method \code{setInnerText} accepts as the first argument a variable of the type compatible with the return type of \code{getText} on line 5. Based on these criteria, the candidates can further be narrowed down to a single one, namely, \code{com\-.google\-.gwt\-.resources\-.client\-.CssResource\-.getText()}. Thus, knowledge of the dependence of the API elements on other program elements in the code snippet can provide crucial hints towards deciding between multiple FQNs corresponding to an API element with the same simple name, but from different libraries. Let us refer to such contextual information as \textbf{dependency context}.


\begin{figure}
\begin{lstlisting}[
frame=shadowbox,
rulesepcolor= \color{red!20!green!20!blue!20} ,
xleftmargin=1.5em,xrightmargin=0em, aboveskip=1em,
framexleftmargin=1.5em,
numbersep= 5pt,
language=Java,
basicstyle=\scriptsize\ttfamily, numberstyle=\scriptsize\ttfamily, emphstyle=\bfseries, escapeinside= {(*@}{@*)}]
{
  StyleElement style=Document.get().createStyleElement();
  cssRes = resources.(*@{\color{blue}css();@*)
  ...
  s = cssRes.(*@{\color{blue}getText();@*)
  ...
  style.(*@{\color{blue}setInnerText(s);@*)
  ...
}                
\end{lstlisting}
\vspace{-12pt}
\caption{Excerpt from StackOverflow Post \#34595450} %on Google Web Toolkit Library}
\label{fig:excerpt-example1}
\end{figure}


Several approaches have been proposed to automatically resolve the fully qualified names of API elements in a code snippet. We can classify them into various categories, the first being program analysis. Partial program analysis (PPA)~\cite{dagenais-oopsla08} infers the types/FQNs by resolving the syntactic ambiguities in incomplete code snippets in a best-effort manner and analyzing the relations among the program elements. RecoDoc~\cite{dagenais-icse12} tries to recover a code-like term's fully-qualified name by linking it with all the inferred types in the system code based on several heuristics. One key issue with such best-effort, heuristic-driven analyses is that they are \underline{not comprehensive} and do not work in all cases. Besides, RecoDoc requires the code snippet to be a subset of a partial program to carry out such an analysis.

The second category leverages {\em information retrieval} (IR) techniques. Given a code snippet, Baker~\cite{liveapi14} tracks the scope of all API elements to build a candidate list for each, which it then overlaps based on the scoping rules for shortlisting. However, this technique is ineffective as the missing information in incomplete code often leaves each API element with potentially many FQNs.
%Baker extracts constraints from code snippets and uses a naive constraint solving algorithm to infer FQNs.
%Baker~\cite{liveapi14} builds a candidate list for each name by tracking the scopes of the names and then overlapping the lists according to the scoping rules for shorlisting.
COSTER~\cite{coster-ase19} extracts locally specific code elements (i.e., \textit{local context}) and globally related tokens (i.e., \textit{global context}) of the query API element to infer their FQNs based on three criteria: likelihood, context similarity, and name similarity.
The next category leverages {\em constraint-based} techniques. SnR~\cite{snr-icse22} first builds a knowledge base of APIs from the existing libraries by extracting various type-related facts about the available APIs. Then, SnR extracts typing constraints from a given code snippet and uses the knowledge-base to infer a set of APIs that satisfy the constraints.
The key limitation with the IR and constraint-based approaches is that the dictionary of the \underline{APIs must be known a priori}. Thus, they are ineffective in dealing with FQNs that have not been seen before in the training corpus.

The next category leverages {\em machine} and {\em deep learning} (ML \& DL) techniques to tackle the issue of out-of-vocabulary (OOV) failures while deriving the FQNs.
StatType~\cite{icse18} formulates FQN resolution as a {\em phrase-based machine translation} task where an API sequence without FQNs is constructed for a given code snippet and translated to an equivalent one with FQNs.
%In contrast, Huang et al.~\cite{prompt-ase22} model FQN resolution as a fill-in-the-blank problem, transforming the given code snippet by: (a) identifying the type inference (TI) locations using regular expressions; (b) fragmenting the code snippet into code blocks, each corresponding to a TI location; (c) inserting \code{[MASK]} tokens in all locations; (d) independently resolving their FQNs by filling in the corresponding \code{[MASK]} tokens with the help of a {\em masked language model} (MLM\textsubscript{\textit{FIB}}).
In contrast, Huang et al.~\cite{prompt-ase22} model FQN resolution as a fill-in-the-blank problem by employing a masked language model (MLM\textsubscript{\textit{FIB}}).
However, a key issue with both approaches is that the limited amount of context they consider does not account for the dependence of the API elements on other program elements in the code snippet. For example, to derive FQNs in the code snippet in Figure~\ref{fig:excerpt-example1}, the state-of-the-art DL approach, MLM\textsubscript{\textit{FIB}}, fragments the code snippet into multiple code blocks \textit{CB\textsubscript{i}}, each representative of the API element on lines 3, 5, and 7. Moreover, each code block \textit{CB\textsubscript{i}} includes a limited amount of surrounding context, say, lines 4--6 corresponding to \code{getText} on line 5. As described earlier, not being aware of \code{css} on line 3 and \code{setInnerText} on line 7 makes the determination of the FQN for \code{getText} challenging. Thus, the \underline{lack of dependency context} yields MLM\textsubscript{\textit{FIB}} inadequate in unambiguously resolving an FQN with the same simple name but defined across multiple libraries. 
%Moreover, the splitting of code snippet into code blocks also results in an independent inference of the FQNs in each block, wherein, the correlations between the co-occurring API elements are overlooked.
Moreover, the division of the given code snippet into discrete code blocks disregards the inter-connections between the co-occurring API elements, as a result of which, the API element in each code block undergoes an autonomous inference process for the FQN resolution.

In this paper, we introduce {\tool}, a DL-based tool which aims to address the issues highlighted in existing approaches and effectively recover the FQNs for all API elements in a given code snippet. Our approach takes advantage of the inter-dependencies between API elements in API usages to derive the appropriate FQNs for all pertinent API elements simultaneously. In essence, we view an FQN as the identity of an API element, and our approach is centered around the philosophy {\em ``Tell Me Your Friends, I'll Tell You Who You Are''}. 
%To determine the ``friends'' of an API element $A$, we leverage the program entities and other API elements that depend on or have relations with $A$ as in the above example. 
Our rationale is that the designers of software libraries always intend for users to use specific combinations of API elements together to achieve a given programming task. Such combinations of API elements, also known as {\em API usage patterns}, may appear in multiple code repositories that use the corresponding libraries.

%Therefore, such API elements in API usages frequently occur together
%in the code using the libraries, which are referred to as {\em API
%  usage patterns}.

The {\em basis of regularity} guides our approach in two ways. First, the associations between the more regularly appearing API elements and their corresponding FQNs play a significant role in determining the FQN of the API element in API usage. Thus, an ML/DL-based model could {\em learn from more frequently occurring API element-FQN pairs} in the training corpus. To facilitate this idea, we obtain complete, compilable code from open-source code repositories of the libraries, ensuring that the FQNs of all API elements are readily available. Second, {\em the program dependencies and relations shared more frequently among API elements are a consequence of the design objectives of software libraries}. Such dependency context, as explained earlier, is critical in ascertaining the identities (i.e., FQNs) of the API elements.

%use ILM, and show that it is capable of capturing the dependencies
%among API elements

Putting together these ideas, we break down the task of inferring FQNs for a given code snippet as follows: First, we design a Type Inference Locations Extraction (TILE) algorithm, which, given a complete/incomplete code snippet, systematically identifies all API locations, and inserts a special \code{[blank]} token ahead of the simple name corresponding to the API element. In Figure~\ref{fig:unknown}, we show an illustration for how TILE transforms a code snippet without FQNs (left), to one with \code{[blank]} tokens (right). Next, we leverage an Infilling Language Model~\cite{} (ILM), geared to predict the missing FQNs in the place of all \code{[blank]} tokens inserted in the previous step. We do so by enabling a causal language model (CLM), that is traditionally capable of and efficient at generating text in an auto-regressive fashion to predict the missing blanks that are consistent with the preceding and subsequent code.

Though both \tool and the state-of-the-art MLM\textsubscript{\textit{FIB}} model FQN resolution as a fill-in-the-blank task, there are some fundamental differences between their design. First, MLM\textsubscript{\textit{FIB}} uses regular expressions to identify the locations of API elements in a code snippet in comparison to our syntax tree-based TILE algorithm. Thus, it is not comprehensive. Next, the masked language model used in MLM\textsubscript{\textit{FIB}} requires the prior knowledge of the length of each blank. However, this is not available a priori. Huang et al. work around this issue by a brute-force approach, i.e., by trying lengths ranging from 3 to 69 for each blank, and picking the one with the highest likelihood. In contrast, our tool, by design, is not reliant on the length of the FQN, and is capable of filling the blanks with variable length spans. Thus, \tool is much more time-efficient in inferring all the FQNs in a code snippet.


We performed several experiments to evaluate {\tool}. ...

The key contributions of this paper include:

1. {\tool}: a neural-network approach ...

2. An extensive evaluation showing {\tool}'s better accuracy in
deriving FQNs than the state-of-the-art approaches.


\section{Motivation}
\label{motiv:sec}

\subsection{Motivating Examples}
\label{examples:sec}

%https://stackoverflow.com/questions/4531396/get-value-of-a-edit-text-field/4531500#4531500: SO post #4531500
\begin{figure}[t]
	\centering
	\lstset{
		numbers=left,
		numberstyle= \tiny,
		keywordstyle= \color{blue!70},
		commentstyle= \color{red!50!green!50!blue!50},
		frame=shadowbox,
		rulesepcolor= \color{red!20!green!20!blue!20} ,
		xleftmargin=1.5em,xrightmargin=0em, aboveskip=1em,
		framexleftmargin=1.5em,
                numbersep= 5pt,
		language=C,
    basicstyle=\scriptsize\ttfamily,
    numberstyle=\scriptsize\ttfamily,
    emphstyle=\bfseries,
                moredelim=**[is][\color{red}]{@}{@},
		escapeinside= {(*@}{@*)}
	}
\begin{lstlisting}[]
(*@{\color{blue}{Button}@*)   mButton;
EditText mEdit;

public (*@{\color{black}{void}@*) onCreate(Bundle savedInstanceState){
  super.onCreate(savedInstanceState);
  setContentView(R.layout.main);

  (*@{\color{blue}mButton = findViewById(R.id.button);@*)
  mEdit   = (EditText)findViewById(R.id.edittext);

  (*@{\color{blue}mButton.setOnClickListener(@*)
    (*@{\color{purple}new View.OnClickListener()@*)
    {
      public (*@{\color{black}{void}@*) onClick(View view)
       {
         Log.v("EditText", mEdit.getText().toString());
       }
    });
}
\end{lstlisting}
        \vspace{-18pt}
        \caption{StackOverflow Post \#4531500 on Android Library}
        \label{fig:example1}
\end{figure}

%(*@@@*)Override


Let us use a few real-world examples to motivate our approach.
Figure~\ref{fig:example1} displays a code snippet of an answer to an
SO question on Android library. Due to the discussion context in SO,
and its informal nature, code snippets rarely contain sufficient
declarations and references to the fully-qualified names (FQNs). The
code snippets are often missing necessary import statements and the
references to the external types are also unqualified (without
FQNs) since the responder assumes that those FQNs could be implicitly
understood in the context of the post. For example, in
Figure~\ref{fig:example1}, the types \code{Button} (line 1),
\code{EditText} (line 2), \code{Bundle} (line 4), \code{View} (line
14), and \code{Log} (line 16) are referenced by simple names
only. Thus, the code will not be compilable if the corresponding
import statements are not added.

%https://stackoverflow.com/questions/18323473/how-to-implement-gwt-java-button-and-the-clickhandler
\begin{figure}[htbp]
	\centering
	\lstset{
		numbers=left,
		numberstyle= \tiny,
		keywordstyle= \color{blue!70},
		commentstyle= \color{red!50!green!50!blue!50},
		frame=shadowbox,
		rulesepcolor= \color{red!20!green!20!blue!20} ,
		xleftmargin=1.5em,xrightmargin=0em, aboveskip=1em,
		framexleftmargin=1.5em,
                numbersep= 5pt,
		language=C,
    basicstyle=\scriptsize\ttfamily,
    numberstyle=\scriptsize\ttfamily,
    emphstyle=\bfseries,
                moredelim=**[is][\color{red}]{@}{@},
		escapeinside= {(*@}{@*)}
	}
\begin{lstlisting}[]
public class myClass implements EntryPoint {
    final (*@{\color{blue}{Button}@*) myButton = new (*@{\color{blue}{Button}@*)("text");
    (*@{\color{blue}{myButton.addClickHandler(}@*)
        (*@{\color{purple}{new ClickHandler() \{}@*)
            public (*@{\color{black}{void}@*) onClick(ClickEvent event) {
               onClickMyButton(event);
        }
    });
    private (*@{\color{black}{void}@*) onClickMyButton(ClickEvent event) {
            ... 
    }
}
\end{lstlisting}
        \vspace{-12pt}
        \caption{StackOverflow Post \#18323473 on GWT Library}
        \label{fig:example2}
\end{figure}


Importantly, the name ambiguity occurs in the references to the APIs
of the external libraries. For example, the type \code{Button} at line
1 of Figure~\ref{fig:example1} is a common unqualified type name. In
this snippet, it refers to \code{android.widget.Button}. However, it
is ambiguous with other APIs in different libraries that
contain the same concept. For example, Figure~\ref{fig:example2} shows
a code snippet of an answer for an SO post on Google Web Toolkit
(GWT). Because the code snippet contains no import statement, and the
references to the APIs are unqualified, the type \code{Button} at line
2 of Figure~\ref{fig:example2} is ambiguous with the type of the same
name at line 1 of Figure~\ref{fig:example1} on Android library. The
ambiguity in type names is popular in code snippets in the forums,
e.g., the simple name \code{getId} occurs 27,434 in various Java
libraries~\cite{liveapi14}.

\subsection{Observations}
\label{sec:obs}

To facilitate the reuse of code snippets in a forum, an automated tool
is needed to derive the fully-qualified names of the API elements in
the snippets so that the proper import statements are added in the
code. To build such a tool, we draw the motivation from the following
observations.

\vspace{2pt}
\noindent {\bf Observation 1} [{\em Regularity of API Usages}]. The
designers of software libraries have the intents for developers to use
certain API elements together (including API classes, method calls,
field accesses) in certain combinations and orders to achieve a
programming task. Those API elements do not occur randomly. For
example, in Figure~\ref{fig:example2} at line 2, in GWT, a variable of
the type \code{Button} (FQN:
\code{com.google.gwt.user.client.ui.Button}) is instantiated. Then, at
line 3, to set the handler of that GWT button, one needs to have a
method call to \code{addClickHandler} (FQN:
\code{com.google.gwt.user.client.ui.Button.add\-Click\-Handler}) on
the \code{Button} object with an argument of the type
\code{ClickHandler} (FQN:
\code{com\-.google\-.gwt\-.event\-.dom\-.client\-.ClickHandler}).  The
API elements are provided and intented to be used in such a code,
called {\em an API usage}. Thus, those API elements of API usages
frequently appear together in the client code using the
library. Figure~\ref{fig:example3} shows a complete example published
in the GWT tutorial website \code{gwtproject.org}. Providing all the
proper \code{import} statements, the author shows how to use the GUI
elements in GWT including \code{Button}. Specifically, at line 23,
despite using a different variable name \code{addStockButton}, the
method \code{addClickHandler} is called on a Button object (declared
at line 12) with the argument of the same type \code{ClickHandler}. In
brief, the source code in the public repositories could be a good
source for a model to implicitly learn the API usages to derive the
FQNs of the API elements in an incomplete snippet.

%https://www.gwtproject.org/doc/latest/tutorial/manageevents
\begin{figure}[htbp]
	\centering
	\lstset{
		numbers=left,
		numberstyle= \tiny,
		keywordstyle= \color{blue!70},
		commentstyle= \color{red!50!green!50!blue!50},
		frame=shadowbox,
		rulesepcolor= \color{red!20!green!20!blue!20} ,
		xleftmargin=1.5em,xrightmargin=0em, aboveskip=1em,
		framexleftmargin=1.5em,
                numbersep= 5pt,
		language=C,
    basicstyle=\scriptsize\ttfamily,
    numberstyle=\scriptsize\ttfamily,
    emphstyle=\bfseries,
                moredelim=**[is][\color{red}]{@}{@},
		escapeinside= {(*@}{@*)}
	}
\begin{lstlisting}[]
import com.google.gwt.core.client.EntryPoint;
import com.google.gwt.event.dom.client.ClickEvent;
import com.google.gwt.event.dom.client.ClickHandler;
import com.google.gwt.user.client.ui.Button;
import com.google.gwt.user.client.ui.HorizontalPanel;
import com.google.gwt.user.client.ui.RootPanel;
import com.google.gwt.user.client.ui.VerticalPanel;
...
public class StockWatcher implements EntryPoint {
  private VerticalPanel mainPanel = new VerticalPanel();
  private HorizontalPanel addPanel = new HorizontalPanel();
  private Button addStockButton = new Button("Add");
  ...
  public void onModuleLoad() {
    ...
    // Assemble Add Stock panel.
    addPanel.add(addStockButton);
    // Assemble Main panel.
    mainPanel.add(addPanel);
    // Associate the Main panel with the HTML host page.
    RootPanel.get("stockList").add(mainPanel);
    // Listen for mouse events on the Add button.
    (*@{\color{blue}{addStockButton.addClickHandler(}@*) (*@{\color{purple}{new ClickHandler() \{}@*)
      public void onClick(ClickEvent event) {
        addStock();
      }
    });
  }
  private void addStock() {
    ...
  }
}
\end{lstlisting}
        \vspace{-12pt}
        \caption{API Usage as Complete Code in gwtproject.org}
        \label{fig:example3}
\end{figure}


\vspace{2pt}
\noindent {\bf Observation 2} [{\em Dependencies/Relations among API
    Elements in a Usage}]. The dependencies/relations among the API
elements in a API usage can help a model better identify the FQNs of
the elements.  In Figure~\ref{fig:example3}, the API elements
\code{Button}, \code{addClickHandler}, and \code{ClickHandler} in GWT
have the program dependencies/relations among them. For example, in
GWT, to set a handler for a button, the object of the type
\code{Button} needs to be the {\em receiving object of the method
  call} to \code{addClickHandler}, which in turn needs to accept an
object of the type \code{ClickHandler} as an argument. These relations
exhibit in the client code using the GWT library, and are useful in
deciding the FQNs of its API elements. For example, in
Figure~\ref{fig:example2},~at line 3, if \code{addClickHandler} is
determined to be the API element
\code{com.\-google.\-gwt.\-user.\-client.\-ui.\-Button.\-add\-Click\-Handler}, the
FQN of the element at line 4 must be
\code{com\-.google\-.gwt\-.event\-.dom\-.client\-.ClickHandler}.  The
other direction of reasoning is applicable as well. In general, if a
model can learn the dependencies/relations among API elements, it
could leverage such knowledge to decide the FQNs of all those APIs at once.


%the lines 2 and 3 in Figure 2
As another example, the data dependency from the \code{def-use}
relation via the variable \code{myButton} between line 2 and line 3 in
Figure~\ref{fig:example2} is useful in deriving the FQNs of the
above API elements. If a model decides the FQN for \code{Button} at
line 2 as \code{com\-.google\-.gwt\-.user\-.client\-.ui\-.Button}, it
could derive the FQN of \code{add\-Click\-Handler} at line~3 as
\code{com.\-google.\-gwt.\-user.\-client.\-ui.\-Button.\-add\-Click\-Handler},
and vice versa.

\vspace{2pt}
\noindent {\bf State-of-the-Art Approaches.} Several approaches have
been proposed to automatically recover the fully-qualified names
(FQNs) for the API elements in a code snippet. The {\em
  program-analysis-based} approaches (e.g.,
PPA~\cite{dagenais-oopsla08}, RecoDoc~\cite{dagenais-icse12}), {\em
  information-retrieval-based} approaches (e.g.,
Baker~\cite{liveapi14}, COSTER~\cite{coster-ase19}), and {\em
  constraint-based} approaches (e.g., SnR~\cite{snr-icse22}) suffer
the out-of-vocabulary issue (i.e., could not derive the FQNs that were
not seen in the training corpus).

The advances in {\em Artificial Intelligence (AI)} and {\em Machine
  Learning (ML)} have enabled the generation of the new FQNs for the
APIs. However, those ML-based approaches (e.g., StatType~\cite{icse18}
and Huang {\em et al.}~\cite{prompt-ase22}) still do not leverage the
regularity of API usages and the dependencies/relations among relevent
API elements for FQN recovery. StatType~\cite{icse18} uses
phrase-based statistical machine translation from the code without
FQNs to the one with them.  With short phrases of the lengths of 3-8
tokens, it cannot capture the relevant API elements yet far apart. For
example, in Figure~\ref{fig:example1}, such short phrases in StatType
cannot make the connections between the API elements \code{mButton}
and \code{setOnClickListener} at line 11 to the relevant API elements
\code{mButton}, \code{findViewById}, etc. at line 8. In other cases,
the two relevant statements that could help the FQN recovery might be
even farther in the code. In contrast, Huang {\em et
  al.}~\cite{prompt-ase22} uses the context of a few lines surrounding
the prediction point (e.g., line 11) for their filling-in approach
with a masked language model. First, a few lines might not capture the
relevant API elements in the same usage. In Figure~\ref{fig:example1},
\code{mButton} at line 1 is far apart from \code{mButton} at line 8
and \code{mButton} at line 11. Second, each API element might be used
in a different context in the client code. For example, the code at
line 9 in Figure~\ref{fig:example1} is specific to the method
\code{onCreate} at line 4. Thus, this type of context might not help a
model learn the FQN of an API.


%Talk about StatType and fill-in 

\subsection{Key Ideas}
\label{sec:key}

We propose {\tool}, a ML-based approach to identify the FQNs for the
simple names of variables, API classes, method calls, and field
accesses in a code snippet. From the observations, we have drawn the
following key ideas:

\vspace{2pt}
\noindent {\bf Key Idea 1} [{\em Leveraging Regularity of API
    Usages}].  We leverage Machine Learning to implicitly learn
co-occurring API elements in API usages to derive FQNs. Observation 1
inspires us in the first principle in our solution, which is the basis
of regularity of API usages in a large training corpus: the API
elements (with their FQNs) regularly appearing together in API usages
have higher impact in deciding the FQNs than the less regular ones. We
leverage the complete, compilable code using the libraries from large
code corpus, in which the FQNs of all the API elements in use are
known. We use the code to train a Masked Language Model (MLM) in which
the FQNs of the API elements are masked to learn the
dependencies/relations among them. Then, the model will be fine-tuned
to predict the FQNs for the API elements in a given code snippet.

\vspace{2pt}
\noindent {\bf Key Idea 2} [{\em ``Tell Me Your Friends, I'll Tell You
    Who You Are''}]. We consider the problem of deriving the FQNs as
the identifications of the API elements in a given code snippet.
Instead of trying to identify the FQN of an API element based on its
characteristics, we aim to derive the FQNs of related API elements at
the same time by leveraging the dependencies/relations among them.  We
use a graph representation, called Augmented Usage Graph
(AUG)~\cite{msr19}, to represent the program dependencies and
relations among program entities and API elements. We enhance the AUG
with all the FQNs because the training code is compilable. From that
AU, we mask the FQNs of all the API elements to train the Masked
Language Model (MLM).

\vspace{2pt}
\noindent {\bf Key Idea 3} [{\em Span-based Masked Language Model}]. ...

%Span-based + Length, Embedding



\subsection{Key Ideas}
\label{sec:key}

We propose {\tool}, a ML-based approach to identify the FQNs for the
simple names of variables, API classes, method calls, and field
accesses in a code snippet. From the observations, we have drawn the
following key ideas:

\vspace{2pt}
\noindent {\bf Key Idea 1} [{\em Leveraging Regularity of API
    Usages}].  We leverage Machine Learning to implicitly learn
co-occurring API elements in API usages to derive FQNs. Observation 1
inspires us in the first principle in our solution, which is the basis
of regularity of API usages in a large training corpus: the API
elements (with their FQNs) regularly appearing together in API usages
have higher impact in deciding the FQNs than the less regular ones. We
leverage the complete, compilable code using the libraries from a large
code corpus, in which the FQNs of all the API elements in use are
known.

\vspace{2pt}
\noindent {\bf Key Idea 2} [{\em ``Tell Me Your Friends, I'll Tell You
    Who You Are''}]. We consider the problem of deriving the FQNs as
the identifications of the API elements in a given code snippet.
Instead of trying to identify the FQN of an API element based on its
characteristics, we aim to derive the FQNs of related API elements at
the same time by leveraging the dependencies/relations among them.  We
use a graph representation, called Augmented Usage Graph
(AUG)~\cite{msr19}, to represent the program dependencies and
relations among program entities and API elements. We extract the AUGs
from the complete, compilable source code using the APIs from a large
code corpus. We then enhance the AUGs with all the FQNs because the
training code is compilable. From those AUGs, we mask the FQNs of all
the API elements to train a Masked Language Model (MLM) to learn the
dependencies/relations among them. Finally, the model will be
fine-tuned to predict the FQNs for the APIs in a given code snippet.

\vspace{2pt}
\noindent {\bf Key Idea 3} [{\em Span-based Masked Language Model}]. ...

%Span-based + Length, Embedding


%\section{Important Concepts}
\label{sec:concepts}

This section describes the definitions of the important concepts
regarding our representations on API usages.

%\begin{figure}[t] %[!htp]
%	\centering
%	\includegraphics[width=0.9\linewidth]{aug-example}
%        \vspace{-3pt}
%	\caption{An API Usage and its API-Usage Graph}
%	\label{fig:aug}
%\end{figure}

\begin{Definition}[API elements]
  An API element is either a class, a method, or a field that is
  provided in a library to enable the accesses to the library's
  functions via a variable declaration with a certain class, a method
  call to an API method, or a field access to an API field.
\end{Definition}

For example, in Fig.~\ref{fig:example3}, line 12, \code{Button} is
an API class, which is a declared type for the variable
\code{addStockButton}. At line 23, \code{addClickHandler} is an API
method, which is called on the variable \code{addStockButton}.

\begin{Definition}[API Usage]
An API usage consists of a set of API elements and control structures
(i.e., conditions and repetitions), together with other program
elements (e.g., variables, parameters, etc.) in specific combinations
and orders to perform a programming task.
\end{Definition}

In Fig.~\ref{fig:example2}, lines 2-4 show an API usage consisting
of 1) a variable \code{myButton}, 2) its declared class \code{Button},
3) a method call to \code{addClickHandler} on the variable
\code{myButton}, 4) the method call \code{add\-Click\-Handler} accepting an
argument of the type \code{ClickHandler}, etc.

\begin{Definition}[API Usage Relation]
  In an API usage, there exist the API usage relations among the API
  elements and relevant program elements. The API usage relations
  include the following ones: {\bf receiver, parameter, definition,
    order, condition, synchronize, throw, handle}, and {\bf data and
    control dependencies} among the API and program elements (will be
  detailed next).
\end{Definition}

Let us use the term {\em action nodes} to refer to method calls, field
accesses, or operators, and the term {\em data nodes} to represent
objects, values, and literals that appear in API usages. A {\em
  receiver} relation exists between a variable and a method call. In
Fig.~\ref{fig:example2}, at line 3, there exists a receiver relation
between the variable \code{myButton} and the method
\code{addClickHandler}. A {\em parameter} relation connects an
argument to be used as a parameter of an action. A {\em definition}
relation exists between a constructor or method call that creates or
returns a value or object to the respective variable. An {\em order}
relation connects two actions on operating on the same receiver or
parameter. A {\em condition} relation connects an action whose result
controls branching to an action being controlled. A {\em synchronize}
relation connects a variable that the program obtains a lock on to an
action executed under that lock. A {\em throw} relation connects an
action that may throw an exception to a data node representing that
exception object. A {\em handle} relation connects from a \code{catch}
action to an action in a respective exception handling block.

We expect to leverage those API usage relations among the API elements
and relevant program entities to learn the FQNs.

%Toward that goal, we adopt a graph-based representation for API
%usages, called {\em API-Usage Graphs (AUGs)}~\cite{msr19}.

%\begin{Definition}[API Usage Graph (AUG)~\cite{msr19}]
%AUG is a directed, connected graph with labelled nodes and
%edges. Nodes represent data entities (variables, values), and actions,
%(e.g., method calls or operators). Edges represent the API usage
%relations among the entities and actions represented by nodes.
%\end{Definition}

%Fig.~\ref{fig:aug} shows an example of an API usage and its AUG.
%The action nodes are displayed in the rectangles and the data nodes in
%the oval shapes. The action nodes represent constructor calls
%(\code{init}), method calls, field accesses, and operators. If the
%types are available, they will be resolved. However, in the figure,
%only the simple name is shown for clarity. The relational operators
%are also encoded as actions to capture conditions. The data nodes
%represent objects, values, and literals in an API usage. AUG encodes
%data entities as nodes to make explicit the data dependencies between
%actions, such as multiple calls on the same object to ensure we have a
%connected subgraph with all data-dependent parts of a usage. The usage
%relations are shown with their labels. {\em Order} edges are not
%shown for clarity. The AUG building algorithm is explained in~\cite{msr19}.


%\section{Approach Overview}
\label{sec:overview}



\section{Our Approach}
\label{sec:approach}

\subsection{Type Inference Location Extraction}
% Please add the following required packages to your document preamble:
% \usepackage{multirow}
\begin{table*}[]
  \centering
  \small
  \tabcolsep 3pt
\begin{tabular}{l|c}
\toprule
\multicolumn{1}{c|}{\textbf{Node Type}}                               & \textbf{Transformation Rule and Example}                                                                                                                                                              \\ \hline
\multirow{3}{*}{Array Creation}                 & \cellcolor{gray!15} $\mathcal{T}(\text{\tabcode{\textbf{new }}}\mathcal{N}_\mathcal{S}[E]) \:=\: \text{\tabcode{\textbf{new }{\tabblank}.}}\mathcal{N}_\mathcal{S}[\mathcal{T}(E)]$  \\
                                                & \multicolumn{1}{l}{\textit{Example}: \tabcode{new Context[contexts.size()]} $\:=\:$ \tabcode{new \textcolor{blue}{[blank]}.Context[contexts.size()]}} \\ 
                                                & \multicolumn{1}{l}{\textit{FQN}: \tabcode{org.xml.sax.helpers.NamespaceSupport}} \\ \hline
\multirow{3}{*}{Cast Expression}                & \cellcolor{gray!15} $\mathcal{T}((\mathcal{N}_{\mathcal{S}}) E) \:=\: (\text{\tabcode{{\tabblank}}. }\mathcal{N}_{\mathcal{S}}) \mathcal{T}(E)$                                                                                \\
                                                & \multicolumn{1}{l}{\textit{Example}: \tabcode{(LexicalHandler) value} $\:=\:$ \tabcode{(\textcolor{blue}{[blank]}.LexicalHandler) value}} \\ 
                                                & \multicolumn{1}{l}{\textit{FQN}: \tabcode{org.xml.sax.ext}} \\ \hline
\multirow{2}{*}{Class Instance Creation}        & \cellcolor{gray!15} $\mathcal{T}(\text{\textbf{\tabcode{new}} }\mathcal{N}_{\mathcal{S}}(E_1, E_2, ..., E_n)) \:=\:\text{ \textbf{\tabcode{new}} \tabcode{{\tabblank}}. }\mathcal{N}_{\mathcal{S}}(\mathcal{T}(E_1), \mathcal{T}(E_2), ..., \mathcal{T}(E_n))$ \\
                                                & \multicolumn{1}{l}{\textit{Example}: \tabcode{new Context()} $\:=\:$ \tabcode{new \textcolor{blue}{[blank]}.Context()}} \\ 
                                                & \multicolumn{1}{l}{\textit{FQN}: \tabcode{org.xml.sax.helpers.NamespaceSupport}} \\ \hline
\multirow{3}{*}{Instanceof Expression}          & \cellcolor{gray!15} $\mathcal{T}(E\text{ \textbf{\tabcode{instanceof}} }\mathcal{N}_{\mathcal{S}}) \:=\: \mathcal{T}(E)\text{ \textbf{\tabcode{instanceof}} } \text{\tabcode{{\tabblank}}. }\mathcal{N}_{\mathcal{S}}$           \\
                                                & \multicolumn{1}{l}{\textit{Example:} \tabcode{addr instanceof Inet6Address} $\:=\:$ \tabcode{addr instanceof \textcolor{blue}{[blank]}.Inet6Address}} \\ 
                                                & \multicolumn{1}{l}{\textit{FQN}: \tabcode{java.net}}\\\hline
\multirow{3}{*}{Single Variable Declaration}    & \cellcolor{gray!15}  $\mathcal{T}(\mathcal{N}_\mathcal{S}\text{ }I\:\{\:= E\}) \:=\:$ $\text{\tabcode{{\tabblank}}. }\mathcal{N}_\mathcal{S}\text{ }I\:\{\:= \mathcal{T}(E)\}$\\
                                                & \multicolumn{1}{l}{\textit{Example}: \tabcode{Node<E> x} $\:=\:$ \tabcode{\textcolor{blue}{[blank]}.Node<E> x}} \\ 
                                                & \multicolumn{1}{l}{\textit{FQN}: \tabcode{java.util.concurrent.LinkedBlockingDeque}} \\ \hline
\multirow{3}{*}{(Super) Constructor Invocation} & \cellcolor{gray!15} $\mathcal{T}(\{\text{\textbf{\tabcode{this}} }\vert\text{ \textbf{\tabcode{super}}\}}(E_1, E_2, ..., E_n)) \:=\: \text{\tabcode{{\tabblank}} }(\mathcal{T}(E_1), \mathcal{T}(E_2), ..., \mathcal{T}(E_n))$                                                 \\
                                                & \multicolumn{1}{l}{\textit{Example}: \tabcode{this(address, true)} $\:=\:$ \tabcode{\textcolor{blue}{[blank]}(address, true)}} \\ 
                                                & \multicolumn{1}{l}{\textit{FQN}: \tabcode{org.apache.harmony.tests.java.net.DatagramSocketTest.DatagramServer}}\\ \hline
\multirow{3}{*}{(Super) Field Access}           & \cellcolor{gray!15} $\mathcal{T}(\{E\text{ }\vert\text{ \textbf{\tabcode{super}}\}}.I) \:=\: \text{\tabcode{\tabblank}. }I$                                                                                                        \\
                                                & \multicolumn{1}{l}{\textit{Example}: \tabcode{this.changeConfig} $\:=\:$ \tabcode{\textcolor{blue}{[blank]}.changeConfig}}               \\ 
                                                & \multicolumn{1}{l}{\textit{FQN}: \tabcode{android.compat.Compatibility.OverrideCallbacks}} \\ \hline
\multirow{3}{*}{(Super) Method Invocation}      & \cellcolor{gray!15} $\mathcal{T}(\{I\text{ }\vert\text{\textbf{ \tabcode{super}}.}I\}(E_1, E_2, ..., E_n)) \:=\: \text{\tabcode{{\tabblank}}. }I(\mathcal{T}(E_1), \mathcal{T}(E_2), ..., \mathcal{T}(E_n))$                                                                \\
                                                & \multicolumn{1}{l}{\textit{Example}: \tabcode{connectLocalServer()} $\:=\:$ \tabcode{\textcolor{blue}{[blank]}.connectLocalServer()}}      \\ 
                                                & \multicolumn{1}{l}{\textit{FQN}: \tabcode{org.apache.harmony.tests.java.nio.channels.DatagramChannelTest}}\\
% \multirow{2}{*}{Throw Statement}                & \cellcolor{gray!15} $\text{\textbf{\tabcode{throw }}}E \:=\: \text{\textbf{\tabcode{throw }}\tabcode{<blank>}. }\mathcal{N}_{\mathcal{S}}$                                     \\
%                                                 & \multicolumn{1}{l}{\textit{Example}: \tabcode{} $\:=\:$ \tabcode{}} \\ 
%                                                 & \multicolumn{1}{l}{\textit{FQN}: \tabcode{}} \\
\bottomrule
\end{tabular}
%\caption{AST node-level transformation rules for building \tabblank-annotated sequences in Type Inference Location Extraction (TILE). Here, $\mathcal{T}$ denotes \textit{Transformation Function}, $\mathcal{N}_\mathcal{S}$ denotes \textit{Simple Name}, $E_i$ denotes \textit{Expression}, and $I$ denotes \textit{Identifier}.}
\caption{AST node-level transformation rules for building \tabblank-annotated sequences in Type Inference Location Extraction (TILE). Here, $\mathcal{T}$, $\mathcal{N}_\mathcal{S}$, $E_i$, and $I$ denote \textit{Transformation Function}, \textit{Simple Name}, \textit{Expression}, and \textit{Identifier}, respectively.}
\label{tab:tile}
\end{table*}


%While previous works~\cite{prompt-ase22} leveraged abstract syntax trees (ASTs) and complete code to expand FQNs and build a training corpus, they employed regular expressions to identify the type inference points in incomplete code snippets. This approach is not comprehensive and leads to several {\em misses}.

Instead of using regular expression matching to identify the locations
of FQNs in source code, which could cause imprecision, we design
a new component called Type Inference Location Extraction (TILE).
%
Given a code snippet, the primary goal is to identify all type inference points, i.e., the locations where an API element needs to be expanded. Such an identification necessitates a more in-depth analysis of its program structure. Thus, we center our Type Inference Location Extraction (TILE) algorithm around analyzing abstract syntax trees (ASTs). Accordingly, TILE has three steps:

\begin{enumerate}
    \item \textit{AST Parsing}: In cases where the given code is complete, such as when building the training corpus, constructing an AST is trivial. In cases where the given code is incomplete, we can utilize tools such as PPA~\cite{dagenais-oopsla08} to build the AST in a best-effort manner. 

    \item \textit{Node Transformations}: An AST has different node
    types, each representing a source code construct such as a name,
    type, expression, statement, or declaration. From these, we
    identified the AST node types listed in Table~\ref{tab:tile} as to
    be API-specific, i.e., possess a type element in its definition
    that can map to an FQN. In this step, we traverse the AST of the
    given code snippet and examine the source code fragments
    associated with these node types. Finally, based on the listed
    transformation rules in Table~\ref{tab:tile}, we add a \blank
    token preceding the simple name of the relevant API element.

%Let us refer to the resulting \blank-inserted code fragments as \blank-sequences.

    \vspace{2pt}
    \par This recursive, AST node-driven approach for identifying the type inference points is comprehensive due to the recursive definition of the transformation function/rule~$\mathcal{T}$.

For instance, consider the single variable declaration node $\mathcal{N}_\mathcal{S}\text{ }I \{=E\}$ that is used in a number of places, e.g., formal parameter lists and catch clauses. Building \blank-annotated sequences for this node, in turn, identifies type inference points in statements where constructors are defined, and where exceptions are caught, due to the recursive call~$\mathcal{T}(E)$.

    \item \textit{AST Unparsing}: 
    Following the application of 
    %the 
    such
    transformation rules on the corresponding node types,
    %as described above,
    we obtain a modified version of the AST. This modified AST can be unparsed to generate a code string with additional \blank tokens, inserted at the type inference points. Let us refer to this as \blank-code.
    Note that the \blank-code is equivalent to the original code snippet, but with the additional \blank tokens.
    In the case of complete code with all dependencies, the compiler can easily resolve all bindings and link the \blank tokens to their corresponding FQNs. In cases where the given code is incomplete, the FQNs can be inferred via the next stage of our approach.
\end{enumerate}  

In Fig.~\ref{fig:approach} (Step I), the illustration on the left corresponds to the original code snippet, and that on the right corresponds to the \blank-code along with its corresponding FQN prefixes.

%When constructing the training corpus, we have access to complete code with all dependencies.

%In complete code with all dependencies, this is straightforward since the compiler can resolve all bindings and link the simple names in source code with their corresponding qualified names. In cases where the code in use is incomplete, we can utilize tools such as PPA~\cite{dagenais-oopsla08} to build the AST in a best-effort manner.


\subsection{Type Inference with Infilling Language Model}
\subsubsection{Problem Formulation}

\subsubsection{Training Process}

\subsubsection{Inference}



\section{Empirical Evaluation}
\label{sec:evaluation}

%\subsection{Research Questions}

We conducted several experiments to evaluate {\tool}. We aim to answer the following questions:

\vspace{2pt}
\noindent \textbf{RQ\textsubscript{1} 
  [Effectiveness Evaluation]} {\em How accurate is {\tool} in expanding FQNs for complete code from Java projects?}

\vspace{2pt}
\noindent \textbf{RQ\textsubscript{2} 
  [Effectiveness Evaluation]} {\em How accurate is {\tool} in expanding FQNs for incomplete code from StackOverflow?}

\vspace{2pt}
\noindent \textbf{RQ\textsubscript{3}
[API-Usage Relations]} {\em Do the conditioning-driven interactions help our tool learn API usages?}

\vspace{2pt}
\noindent \textbf{RQ\textsubscript{4} 
  [Ablation Study]}  {\em Is there a benefit to leveraging a wider context in comparison to a narrower surrounding context for predicting fully qualified names?}



%{\em Is our tool capable of capturing the API-usage relations between API elements and other program elements in the code snippet?}

% \vspace{2pt}
% \noindent \textbf{RQ\textsubscript{4} 
% [Practicality Evaluation]}  {\em How accurate is {\tool} in 
% inferring FQNs for API elements in incomplete code snippets from online forums such as StackOverflow?}

%\subsection{Empirical Methodology}

\subsubsection{Datasets}

\subsubsection{RQ1.}

{\em Baselines.}

{\em Procedure.}

{\em Tuning.}

{\em Metrics.}

\subsubsection{RQ2.}

{\em Baselines.}

{\em Procedure.}

{\em Tuning.}

{\em Metrics.}



\section{Effectiveness Evaluation}
\label{sec:effectiveness-eval}

\subsection{Data Collection}

\input{data-statistics}

To enable the effectiveness evaluation, we considered six popular Java libraries that developers typically make use of: \code{android}, \code{gwt}, \code{hibernate}, \code{jdk}, \code{joda-time}, and \code{xstream}. This is consistent with prior works on FQN resolution~\cite{icse18, snr-icse22, prompt-ase22}. Among these, StatType\cite{icse18} and SnR\cite{snr-icse22} leverage projects that utilize these libraries. A limitation with this approach is that not all library API elements are covered in their dataset. In contrast, we download these large-scale repositories from GitHub and use their source code itself to build our dataset.

We used Eclipse JDT to parse each project’s source code and resolve all the FQNs. Next, we traversed the ASTs to collect all the methods. Overall, these libraries contain 198,231 methods including 9,764 from \code{android}, 62,782 from \code{gwt}, 105,734 from \code{hibernate-orm}, 5,235 from \code{jdk}, 9,811 from \code{joda-time}, and 4,905 from \code{xstream}. Next, we randomly split these in a 80\%-10\%-10\% ratio, each for training, validation, and testing, respectively. Finally, we applied the TILE algorithm to identify the type inference locations and build the $\langle$\blank-code, FQN$\rangle$ pairs. In Table~\ref{tab:data-stats}, we report the data statistics including the number of APIs in our dataset.


%All the public classes, fields, and methods are counted not including inherited fields and methods. This dataset represents how developers use a wide variety of real Java libraries in practice, and evaluations using this dataset demonstrate that our technique is sound for libraries ranging from small to large. This benchmark which consists of the code snippets and the libraries was obtained from the original benchmark authors Phan et al.

%We used Eclipse JDT to parse each project’s source code and resolve all the FQNs, and built the source and target sentences for all the methods to form a parallel corpus of pairs of sentences.  The projects using Android and JDK have larger numbers of methods using the APIs due to their popularity.





\subsection{Experiment Methodology}
\subsubsection{Baseline-1} Pre-Trained CodeBERT
\subsubsection{Baseline-2} Fine-Tuned CodeBERT
\subsubsection{Baseline-3} Prompt-Tuned CodeBERT

\subsection{Evaluation Metrics}
\subsubsection{Exact Match (EM) Accuracy}
\subsubsection{ROUGE-L}
\subsubsection{BLEU-2}

\subsection{RQ1... }
\label{sec:rq1}



\section{Investigating API-Usage Relation Learning}
\label{sec:eval}

% \subsection{Background, and Data Collection}
% \subsection{Experiment Methodology}
% \subsection{Evaluation Metrics}
%\subsection{Experiment Results (RQ\textsubscript{3})}
\label{sec:rq3}

% Please add the following required packages to your document preamble:
% \usepackage{multirow}
\begin{table}[]
\centering
\begin{tabular}{l|ccc}
\toprule
\multirow{2}{*}{\textbf{Approach}} & \multicolumn{3}{c}{\textbf{Metrics}}                                  \\ \cline{2-4} 
                                   & \multicolumn{1}{c|}{\textit{Accuracy\textsubscript{EM}}} & \multicolumn{1}{c|}{\textit{ROUGE-L}} & \textit{BLEU-2} \\ \hline
\multicolumn{1}{c|}{$B_n$, \textit{w/o dep. context}}      & \multicolumn{1}{c|}{0.28}         & \multicolumn{1}{c|}{0.61}        &  0.57 \\
\multicolumn{1}{c|}{$B_w$, \textit{w/ dep. context}}       & \multicolumn{1}{c|}{0.72}         & \multicolumn{1}{c|}{0.90}        &  0.91 \\ \bottomrule
\end{tabular}
\caption{Ablation Study.}
\label{tab:ablation}
\end{table}

Within the infilling language modeling framework, {\tool} uses GPT-2, a causal langauge model (CLM) behind the scenes. While the main goal of an CLM is to predict the next-token, another mechanism they take advantage of to learn the various interactions between the inputs is {\em conditioning}. In our task, the CLM is conditioned on all the program elements in the input, as well as the \blank tokens representing the API elements. During the prediction, every subsequent prediction is also conditioned on the previously predicted FQN. We hypothesize that these interactions in conjunction with those with the program elements help GPT-2 learn API-usages.

In this regard, we stratified the test set in Section~\ref{sec:effectiveness-eval} based on the number of API elements ($N$) in the input for which FQNs are to be predicted. In Table~\ref{tab:strat-eval}, we list these strata. For comparing the model performance on these data subsets, we consider the same evaluation metrics as in Section~\ref{sec:effectiveness-eval-proc}.

In Table~\ref{tab:strat-eval}, we can see that the model achieves the worst \textit{Accuracy\textsubscript{EM}} for $N=1$. This could possibly be a consequence of the missing API-API interactions in these examples. Also, we achieve the highest performance for $N=3$ and $N=4$, following which the performance slightly drops. This could be the effect of the greedy search (beam-search) algorithm utilized for this purpose, which tends to worsen as the length of the prediction increases. 

\section{Ablation Study (RQ4)}
\label{sec:ablation}

In Section~\ref{sec:key}, we hypothesized that a wider {\em dependency
  context} can provide crucial hints towards overcoming name ambiguity
and identifying the API elements better. Accordingly, we input the
code of the entire method to train {\tool}. Let us denote such a
dataset by $D_w$. In contrast,
MLM\textsubscript{\textit{FIB}}~\cite{prompt-ase22} leverages a
narrower surrounding context (two lines before and after the statement
containing the API element). Let us denote this dataset by $D_n$. The
details for building $D_w$ and $D_n$ were described in
Section~\ref{sec:effectiveness-eval-proc}.

The goal of this experiment is to gauge the impact that the wider
dependency context has on FQN resolution. Thus, we conducted an
ablation on the amount of contextual information, and created two
baselines: (a) \tool w/ dependency context, i.e., when trained on
$D_w$, (b) \tool w/o dependency context, i.e., trained on $D_n$. We
evaluate both ablation baselines against the test set in
Section~\ref{sec:effectiveness-data}.

\subsection*{Results and Analysis (RQ\textsubscript{2})}
\label{sec:rq2}



% \begin{table}[]
% \centering
% %\scriptsize
% \begin{tabular}{l|c|c|c|l}
% \toprule
% \multicolumn{1}{r|}{\textbf{Baselines} ($\rightarrow$)} & \multicolumn{1}{l|}{\multirow{2}{*}{COSTER}} & \multicolumn{1}{l|}{\multirow{2}{*}{SnR}} & \multicolumn{1}{l|}{\multirow{2}{*}{MLM\textsubscript{FIB}}} & \multirow{2}{*}{\tool} \\
% \multicolumn{1}{l|}{\textbf{Libraries} ($\downarrow$)} & \multicolumn{1}{l|}{}                                 & \multicolumn{1}{l|}{}                              & \multicolumn{1}{l|}{}                              &                                   \\ 
% \hline
% \tabcode{android}                            & 43.3                                                  & 93.6                                               & 90.6                                               &                                   \\
% \tabcode{gwt}                                & 90.8                                                  & 75.8                                               & 95.2                                               &                                   \\
% \tabcode{hibernate}                          & 90.4                                                  & 94.8                                               & 75.9                                               &                                   \\
% \tabcode{jdk}                                & 56.2                                                  & 71.1                                               & 98.9                                               &                                   \\
% \tabcode{joda-time}                          & 57.1                                                  & 89.5                                               & 97.5                                               &                                   \\
% \tabcode{xstream}                            & 88.4                                                  & 100.0                                              & 88.0                                               &                                   \\ \hline
% \textbf{Avg. Accuracy}                          & 71.0                                                  & 87.5                                               & 91.0                                               &      \\
% \bottomrule
% \end{tabular}
% \caption{FQN resolution accuracy (in \%) for incomplete code snippets from StackOverflow (StatType-SO).}
% \label{tab:results-practical}
% \end{table}

% \begin{table}[]
% \centering
% %\scriptsize
% \begin{tabular}{l|c|c|c}
% \toprule
% \multicolumn{1}{r|}{\textbf{Baseline} ($\rightarrow$)} & \multicolumn{3}{c}{\tool} \\ \cline{2-4}
% \textbf{\textbf{Libraries} ($\downarrow$)}             & $Accuracy\textsubscript{\textit{EM}}$     & $ROUGE-L$     & $BLEU-2$    \\
% \hline
% \tabcode{android}                                      &           &           &          \\
% \tabcode{gwt}                                          & 69.23          & 83.39          & 41.53         \\
% \tabcode{hibernate}                                    & 40.00           &  58.61         & 0.27          \\
% \tabcode{jdk}                                          &           &           &          \\
% \tabcode{joda-time}                                    &           &           &          \\
% \tabcode{xstream}                                      & 20.83          &   0.41        & 0.32         \\
% \bottomrule
% \end{tabular}
% \caption{Comparison of model performance for incomplete code snippets from StackOverflow (StatType-SO), based on {\em Exact Match Accuracy} ($M_1$), {\em ROUGE-L} ($M_2$), {\em BLEU-2} ($M_3$)}
% \label{tab:results-practical}
% \end{table}
\begin{table}[]
\begin{tabular}{c|ccc}
\toprule
\multirow{2}{*}{\begin{tabular}[c]{@{}c@{}}\textbf{Evaluation}\\ \textbf{Metrics}\end{tabular}} & \multicolumn{3}{c}{\textbf{Libraries}} \\ \cline{2-4} 
                 & \multicolumn{1}{c|}{\code{gwt}}   & \multicolumn{1}{c|}{\code{hibernate}} & \code{xstream} \\ 
\midrule
\textit{EM}      & \multicolumn{1}{c|}{69.23} & \multicolumn{1}{c|}{40.00}     & 20.83   \\
\textit{ROUGE-L} & \multicolumn{1}{c|}{83.29} & \multicolumn{1}{c|}{58.61}     & 0.41    \\
\textit{BLEU-2}  & \multicolumn{1}{c|}{41.53} & \multicolumn{1}{c|}{0.27}      & 0.32    \\ \bottomrule
\end{tabular}
\caption{Comparison of performance for incomplete Java programs (RQ2)}
\label{tab:practicality}
\end{table}

\section{Related Work}
\label{sec:related}

The approaches to recover the FQNs for the API elements in incomplete
code snippets can be broadly classified into the following categories.
The first category is {\em program analysis}. Partial Program Analysis
(PPA)~\cite{dagenais-oopsla08} resolves the types by applying a set of
heuristics on the syntactic constructs to infer the declared types of
expressions. RecoDoc~\cite{dagenais-icse12} uses PPA to infer links
between APIs and documentation. It requires the target library to be
specified. The second category of approaches leverages {\em
  information retrieval}. Baker~\cite{liveapi14} tracks the scopes of
the names and the candidate lists are combined according to the
scoping rules and get smaller. COSTER~\cite{coster-ase19} formulates
the FQN problem as searching. It aims to match the context of the
query API element with the FQNs in the database regarding three
criteria on likelihood, context similarity, and name similarity.
The third category is {\em constraint-based}. Differing from COSTER,
SnR~\cite{snr-icse22} builds the knowledge of the constraints on
the API elements and extracts such constraints in the code snippet
to match and solve them against the knowledge base.



\section{Conclusion}
\label{sec:conclusion}

In this work, we present {\tool}, a deep learning-based approach that
addresses these issues by formulating FQN resolution as a
\textit{fill-in-the-blank} task. For a given code snippet, our tool
systematically identifies all API locations, inserts a blank ahead of
the simple name corresponding to the API, leverages a causal language
model (CLM) to fill the missing blank with the corresponding FQN.  Our
empirical evaluation shows relatively improves over the
state-of-the-art FQN resolution approaches by 9.1\% on real-world code
snippets. Upon further analysis, we were able to tie the performance
improvement to our tool's capabilities to capture dependencies between
the API and other program elements in the code snippet.


\section{Data Availability}

The model and data is available in our website~\cite{deepfqn}.


%\section*{Acknowledgments}
%This work was supported in part by the US National Science Foundation
%(NSF) grants CNS-2120386.

\newpage
\balance

\bibliographystyle{IEEEtran}

\bibliography{references}

\end{document}
