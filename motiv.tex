\section{Motivation}
\label{motiv:sec}

\subsection{Motivating Examples}
\label{examples:sec}

%https://stackoverflow.com/questions/4531396/get-value-of-a-edit-text-field/4531500#4531500: SO post #4531500
\begin{figure}[htbp]
	\centering
	\lstset{
		numbers=left,
		numberstyle= \tiny,
		keywordstyle= \color{blue!70},
		commentstyle= \color{red!50!green!50!blue!50},
		frame=shadowbox,
		rulesepcolor= \color{red!20!green!20!blue!20} ,
		xleftmargin=1.5em,xrightmargin=0em, aboveskip=1em,
		framexleftmargin=1.5em,
                numbersep= 5pt,
		language=C,
    basicstyle=\scriptsize\ttfamily,
    numberstyle=\scriptsize\ttfamily,
    emphstyle=\bfseries,
                moredelim=**[is][\color{red}]{@}{@},
		escapeinside= {(*@}{@*)}
	}
\begin{lstlisting}[]
(*@{\color{blue}{Button}@*)   mButton;
EditText mEdit;

(*@@@*)Override public (*@{\color{black}{void}@*) onCreate(Bundle savedInstanceState) {
    super.onCreate(savedInstanceState);
    setContentView(R.layout.main);

    (*@{\color{blue}mButton = findViewById(R.id.button);@*)
    mEdit   = (EditText)findViewById(R.id.edittext);

    (*@{\color{blue}mButton.setOnClickListener(@*)
        (*@{\color{purple}new View.OnClickListener()@*)
        {
            public (*@{\color{black}{void}@*) onClick(View view)
            {
                Log.v("EditText", mEdit.getText().toString());
            }
        });
}
\end{lstlisting}
        \vspace{-12pt}
        \caption{StackOverflow Post \#4531500 on Android Library}
        \label{fig:example1}
\end{figure}


Let us examine some real-world examples to help motivate our approach. Fig.~\ref{fig:example1} illustrates a code snippet of an answer to S/O question \#4531500 on the Android library. Due to the informal discourse in S/O, code snippets rarely contain the necessary declarations and references to the fully-qualified names (FQNs); and often lack the import statements. Besides, the references to external types are also unqualified since the responder assumes that those FQNs can be implicitly understood from the context of the post. For example, in Fig.~\ref{fig:example1}, the types \code{Button} (line 1), \code{EditText} (line 2), \code{Bundle} (line 4), \code{View} (line 14), or \code{Log} (line 16) are referenced only by their simple names. Thus, the code will not be compilable unless the corresponding import statements are added.

%https://stackoverflow.com/questions/18323473/how-to-implement-gwt-java-button-and-the-clickhandler
\begin{figure}[htbp]
	\centering
	\lstset{
		numbers=left,
		numberstyle= \tiny,
		keywordstyle= \color{blue!70},
		commentstyle= \color{red!50!green!50!blue!50},
		frame=shadowbox,
		rulesepcolor= \color{red!20!green!20!blue!20} ,
		xleftmargin=1.5em,xrightmargin=0em, aboveskip=1em,
		framexleftmargin=1.5em,
                numbersep= 5pt,
		language=C,
    basicstyle=\scriptsize\ttfamily,
    numberstyle=\scriptsize\ttfamily,
    emphstyle=\bfseries,
                moredelim=**[is][\color{red}]{@}{@},
		escapeinside= {(*@}{@*)}
	}
\begin{lstlisting}[]
public class myClass implements EntryPoint {
    final (*@{\color{blue}{Button}@*) myButton = new (*@{\color{blue}{Button}@*)("text");
    (*@{\color{blue}{myButton.addClickHandler(}@*)
        (*@{\color{purple}{new ClickHandler() \{}@*)
            public (*@{\color{black}{void}@*) onClick(ClickEvent event) {
               onClickMyButton(event);
        }
    });
    private (*@{\color{black}{void}@*) onClickMyButton(ClickEvent event) {
            ... 
    }
}
\end{lstlisting}
        \vspace{-12pt}
        \caption{StackOverflow Post \#18323473 on GWT Library}
        \label{fig:example2}
\end{figure}


Moreover, the APIs of external libraries are prone to name ambiguity, meaning that they can be confused with other APIs from different libraries that share the same name and offer a similar functionality. For example, the element \code{Button} on line 1 in Fig.~\ref{fig:example1} is a common unqualified type. Here, it refers to \code{android.widget.Button}. Next, consider Fig.~\ref{fig:example2}, which illustrates a code snippet of an answer to a S/O post on Google Web Toolkit (GWT). Because this code snippet does not contain any import statements and the references to the APIs are also unqualified, the type \code{Button} on line 2 in Fig.~\ref{fig:example2} is ambiguous from that on line 1 in Fig.~\ref{fig:example1}. Such name ambiguity in type names is a common phenomenon, especially among S/O code snippets. For instance, the simple name \code{getId} occurs 27,434 times across various Java libraries~\cite{liveapi14}.

\subsection{Observations}
\label{sec:obs}

We can observe a need for a tool that automatically derives the fully-qualified names of the API elements in code snippets from online forums. This will facilitate the reuse of such incomplete code by enabling the addition of the appropriate import statements. To build one such tool, we draw motivation from the following observations.

\vspace{2pt}
\noindent {\bf Observation 1} [{\em Regularity of API Usages}]. The designers of software libraries intend for developers to use the API elements together (including API classes, method calls, and field accesses) in certain combinations/orders to achieve a task. For example, in the GWT code snippet illustrated in Fig.~\ref{fig:example2}, a variable of the type \code{Button} (FQN: \code{com.google.gwt.user.client.ui.Button}) is instantiated on line 2. Then, on line 3, to set the handler of that GWT button, one needs to invoke the \code{addClickHandler} method (FQN: \code{com.google.gwt.user.client.ui.Button.add\-Click\-Handler})  on the \code{Button} object with an argument of the type \code{ClickHandler} (FQN: \code{com\-.google\-.gwt\-.event\-.dom\-.client\-.ClickHandler}). Thus, they are intended to be used in  such a combination and will appear together frequently.

Next, in Fig.~\ref{fig:example3}, we illustrate a complete code example published on the GWT tutorial website \code{gwtproject.org}. Here, the author provides all necessary \code{import} statements and demonstrates how to use different GUI elements in GWT. Specifically, consider the \code{Button} object declared with the \code{addStockButton} variable name on line 12. It calls the method \code{addClickHandler} on line 23 with an argument of the same type as earlier, \code{ClickHandler}. 
%Though the \code{Button} object is assigned with different variable names in both cases, i.e., \code{myButton} when incomplete, and \code{addStockButton} when complete, we can see that the combination of these API elements can help establish its identity in the form of its FQN. 
Despite being assigned with different variable names in both the complete and incomplete cases, i.e., \code{myButton} and \code{addStockButton} respectively, we can see that the combination of these API elements can help establish \code{Button} object's identity and resolve its FQN.

In brief, the source code in public repositories is a good source for a model to implicitly learn the API usages and derive the FQNs of the API elements in an incomplete snippet.

%https://www.gwtproject.org/doc/latest/tutorial/manageevents
\begin{figure}[htbp]
	\centering
	\lstset{
		numbers=left,
		numberstyle= \tiny,
		keywordstyle= \color{blue!70},
		commentstyle= \color{red!50!green!50!blue!50},
		frame=shadowbox,
		rulesepcolor= \color{red!20!green!20!blue!20} ,
		xleftmargin=1.5em,xrightmargin=0em, aboveskip=1em,
		framexleftmargin=1.5em,
                numbersep= 5pt,
		language=C,
    basicstyle=\scriptsize\ttfamily,
    numberstyle=\scriptsize\ttfamily,
    emphstyle=\bfseries,
                moredelim=**[is][\color{red}]{@}{@},
		escapeinside= {(*@}{@*)}
	}
\begin{lstlisting}[]
import com.google.gwt.core.client.EntryPoint;
import com.google.gwt.event.dom.client.ClickEvent;
import com.google.gwt.event.dom.client.ClickHandler;
import com.google.gwt.user.client.ui.Button;
import com.google.gwt.user.client.ui.HorizontalPanel;
import com.google.gwt.user.client.ui.RootPanel;
import com.google.gwt.user.client.ui.VerticalPanel;
...
public class StockWatcher implements EntryPoint {
  private VerticalPanel mainPanel = new VerticalPanel();
  private HorizontalPanel addPanel = new HorizontalPanel();
  private Button addStockButton = new Button("Add");
  ...
  public void onModuleLoad() {
    ...
    // Assemble Add Stock panel.
    addPanel.add(addStockButton);
    // Assemble Main panel.
    mainPanel.add(addPanel);
    // Associate the Main panel with the HTML host page.
    RootPanel.get("stockList").add(mainPanel);
    // Listen for mouse events on the Add button.
    (*@{\color{blue}{addStockButton.addClickHandler(}@*) (*@{\color{purple}{new ClickHandler() \{}@*)
      public void onClick(ClickEvent event) {
        addStock();
      }
    });
  }
  private void addStock() {
    ...
  }
}
\end{lstlisting}
        \vspace{-12pt}
        \caption{Complete Source Code in gwtproject.org}
        \label{fig:example3}
\end{figure}


\vspace{2pt}
\noindent {\bf Observation 2} [{\em Dependencies/Relations among API Elements in a Usage}].
The API elements used together in an API usage in certain combinations/orders share various program dependencies. These relationships can contribute to identifying the FQNs of the API elements better. For example, in Fig.~\ref{fig:example3}, we can see that to set a handler for a button in GWT, the object of the type \code{Button} needs to be the {\em receiving object of the method call} to \code{addClickHandler}, which in turn needs to accept an object of the type \code{ClickHandler} as an argument. The client code utilizing GWT library for this purpose will demonstrate the relationships shared between these three API elements as well. For example, in Fig.~\ref{fig:example2}, if \code{addClickHandler} on line 3 is determined to be the API element \code{com.\-google.\-gwt.\-user.\-client.\-ui.\-Button.\-add\-Click\-Handler}, the FQN of the element at line 4 must be \code{com\-.google\-.gwt\-.event\-.dom\-.client\-.ClickHandler}. The opposite direction of reasoning is also applicable. In general, if a model can learn the dependencies/relations among API elements in an API usage, it could leverage such knowledge to decide the FQNs of all API elements at the same time.

%the lines 2 and 3 in Fig. 2
As another example, consider the data dependency from the \code{def-use} relationship via the variable \code{myButton} between line~2 and line~3 in Fig.~\ref{fig:example2}. This relationship helps derive the FQNs for the above API elements. For instance, if a model decides the FQN for \code{Button} on line 2 to be \code{com\-.google\-.gwt\-.user\-.client\-.ui\-.Button}, it could consequently derive the FQN of \code{add\-Click\-Handler} on line~3 as \code{com.\-google.\-gwt.\-user.\-client.\-ui.\-Button.\-add\-Click\-Handler}, and vice versa.

\vspace{2pt}
\noindent {\bf State-of-the-Art Approaches.} Several approaches have been proposed to recover the fully-qualified names (FQNs) for the API elements in a code snippet. The {\em program-analysis-based} approaches (e.g., PPA~\cite{dagenais-oopsla08}, RecoDoc~\cite{dagenais-icse12}), {\em information-retrieval-based} approaches (e.g., Baker~\cite{liveapi14}, COSTER~\cite{coster-ase19}), and {\em constraint-based} approaches (e.g., SnR~\cite{snr-icse22}) are not comprehensive, and suffer from out-of-vocabulary failures. %(i.e., cannot derive FQNs that were not seen in the training corpus).

Employing the recent advances in {\em machine} and {\em deep learning} (ML \& DL) for FQN resolution has enabled the generation of new FQNs for API elements. However, the state-of-the-art ML/DL-based approaches (e.g., StatType~\cite{icse18} and Huang {\em et al.}~\cite{prompt-ase22}) {\bf do not yet leverage the regularity in API-usages and the dependencies among the relevant API elements} for FQN prediction. StatType~\cite{icse18} leverages phrase-based statistical machine translation to transform an API type-sequence without FQNs to one with them. However, with only short phrases of lengths between 3-8 tokens, StatType is not capable of capturing long-range dependencies between API elements. For example, in Fig.~\ref{fig:example1}, StatType misses the inter-connections between the statements spanning across 11 lines from \code{Button mButton} on line 1 to \code{mButton}, \code{findViewById}, etc. on line 8; and to \code{mButton}, \code{setOnClickListener} on line 11. Moreover, in some cases, it is possible for two relevant API elements to be even farther in the code snippet.

Huang {\em et al.}~\cite{prompt-ase22} formulated FQN resolution as a fill-in-the-blank problem and leveraged a masked language model (MLM\textsubscript{\textit{FIB}}) for this purpose. They build each training instance by taking the statement at each type inference point (e.g., \code{View} on line 12) and a couple of surrounding lines with unresolved API elements as context. This approach has several limitations. First, the amount of contextual information might not capture all relevant API elements corresponding to the same API-usage. In Fig.~\ref{fig:example1}, \code{mButton} on line 1 is far apart from \code{mButton} on line 8 and \code{mButton} on line 11. Next, individual API elements might be used in a different context in the client code. For example, the code on line 9 in Fig.~\ref{fig:example1} is specific to the method \code{onCreate}. Thus, such limited context might not provide the model with sufficient information for determining the FQN.


%Talk about StatType and fill-in
