\section{Motivation}
\label{motiv:sec}

\subsection{Motivating Examples}
\label{examples:sec}

%https://stackoverflow.com/questions/4531396/get-value-of-a-edit-text-field/4531500#4531500: SO post #4531500
\begin{figure}[htbp]
	\centering
	\lstset{
		numbers=left,
		numberstyle= \tiny,
		keywordstyle= \color{blue!70},
		commentstyle= \color{red!50!green!50!blue!50},
		frame=shadowbox,
		rulesepcolor= \color{red!20!green!20!blue!20} ,
		xleftmargin=1.5em,xrightmargin=0em, aboveskip=1em,
		framexleftmargin=1.5em,
                numbersep= 5pt,
		language=C,
    basicstyle=\scriptsize\ttfamily,
    numberstyle=\scriptsize\ttfamily,
    emphstyle=\bfseries,
                moredelim=**[is][\color{red}]{@}{@},
		escapeinside= {(*@}{@*)}
	}
\begin{lstlisting}[]
(*@{\color{blue}{Button}@*)   mButton;
EditText mEdit;

(*@@@*)Override public (*@{\color{black}{void}@*) onCreate(Bundle savedInstanceState) {
    super.onCreate(savedInstanceState);
    setContentView(R.layout.main);

    (*@{\color{blue}mButton = findViewById(R.id.button);@*)
    mEdit   = (EditText)findViewById(R.id.edittext);

    (*@{\color{blue}mButton.setOnClickListener(@*)
        (*@{\color{purple}new View.OnClickListener()@*)
        {
            public (*@{\color{black}{void}@*) onClick(View view)
            {
                Log.v("EditText", mEdit.getText().toString());
            }
        });
}
\end{lstlisting}
        \vspace{-12pt}
        \caption{StackOverflow Post \#4531500 on Android Library}
        \label{fig:example1}
\end{figure}


Let us use a few real-world examples to motivate our approach.
Figure~\ref{fig:example1} displays a code snippet of an answer to an
SO question on Android library. Due to the discussion context in SO,
and its informal nature, code snippets rarely contain sufficient
declarations and references to the fully-qualified names (FQNs). The
code snippets are often missing necessary import statements and the
references to the external types are also unqualified (without
FQNs) since the responder assumes that those FQNs could be implicitly
understood in the context of the post. For example, in
Figure~\ref{fig:example1}, the types \code{Button} (line 1),
\code{EditText} (line 2), \code{Bundle} (line 4), \code{View} (line
14), and \code{Log} (line 16) are referenced by simple names
only. Thus, the code will not be compilable if the corresponding
import statements are not added.

%https://stackoverflow.com/questions/18323473/how-to-implement-gwt-java-button-and-the-clickhandler
\begin{figure}[htbp]
	\centering
	\lstset{
		numbers=left,
		numberstyle= \tiny,
		keywordstyle= \color{blue!70},
		commentstyle= \color{red!50!green!50!blue!50},
		frame=shadowbox,
		rulesepcolor= \color{red!20!green!20!blue!20} ,
		xleftmargin=1.5em,xrightmargin=0em, aboveskip=1em,
		framexleftmargin=1.5em,
                numbersep= 5pt,
		language=C,
    basicstyle=\scriptsize\ttfamily,
    numberstyle=\scriptsize\ttfamily,
    emphstyle=\bfseries,
                moredelim=**[is][\color{red}]{@}{@},
		escapeinside= {(*@}{@*)}
	}
\begin{lstlisting}[]
public class myClass implements EntryPoint {
    final (*@{\color{blue}{Button}@*) myButton = new (*@{\color{blue}{Button}@*)("text");
    (*@{\color{blue}{myButton.addClickHandler(}@*)
        (*@{\color{purple}{new ClickHandler() \{}@*)
            public (*@{\color{black}{void}@*) onClick(ClickEvent event) {
               onClickMyButton(event);
        }
    });
    private (*@{\color{black}{void}@*) onClickMyButton(ClickEvent event) {
            ... 
    }
}
\end{lstlisting}
        \vspace{-12pt}
        \caption{StackOverflow Post \#18323473 on GWT Library}
        \label{fig:example2}
\end{figure}


Importantly, the name ambiguity occurs in the references to the APIs
of the external libraries. For example, the type \code{Button} at line
1 of Figure~\ref{fig:example1} is a common unqualified type name. In
this snippet, it refers to \code{android.widget.Button}. However, it
is ambiguous with other APIs in different libraries that
contain the same concept. For example, Figure~\ref{fig:example2} shows
a code snippet of an answer for an SO post on Google Web Toolkit
(GWT). Because the code snippet contains no import statement, and the
references to the APIs are unqualified, the type \code{Button} at line
2 of Figure~\ref{fig:example2} is ambiguous with the type of the same
name at line 1 of Figure~\ref{fig:example1} on Android library. The
ambiguity in type names is popular in code snippets in the forums,
e.g., the simple name \code{getId} occurs 27,434 in various Java
libraries~\cite{liveapi14}.

\subsection{Observations}
\label{sec:obs}

To facilitate the reuse of code snippets in a forum, an automated tool
is needed to derive the fully-qualified names of the API elements in
the snippets so that the proper import statements are added in the
code. To build such a tool, we draw the motivation from the following
observations.

\vspace{2pt}
\noindent {\bf Observation 1} [{\em Regularity of API Usages}]. The
designers of software libraries have the intents for developers to use
certain API elements together (including API classes, method calls,
field accesses) in certain combinations and orders to achieve a
programming task. Those API elements do not occur randomly. For
example, in Figure~\ref{fig:example2} at line 2, in GWT, a variable of
the type \code{Button} (FQN:
\code{com.google.gwt.user.client.ui.Button}) is instantiated. Then, at
line 3, to set the handler of that GWT button, one needs to have a
method call to \code{addClickHandler} (FQN:
\code{com.google.gwt.user.client.ui.Button.add\-Click\-Handler}) on
the \code{Button} object with an argument of the type
\code{ClickHandler} (FQN:
\code{com\-.google\-.gwt\-.event\-.dom\-.client\-.ClickHandler}).  The
API elements are provided and intented to be used in such a code,
called {\em an API usage}. Thus, those API elements of API usages
frequently appear together in the client code using the
library. Figure~\ref{fig:example3} shows a complete example published
in the GWT tutorial website \code{gwtproject.org}. Providing all the
proper \code{import} statements, the author shows how to use the GUI
elements in GWT including \code{Button}. Specifically, at line 23,
despite using a different variable name \code{addStockButton}, the
method \code{addClickHandler} is called on a Button object (declared
at line 12) with the argument of the same type \code{ClickHandler}. In
brief, the source code in the public repositories could be a good
source for a model to implicitly learn the API usages to derive the
FQNs of the API elements in an incomplete snippet.

%https://www.gwtproject.org/doc/latest/tutorial/manageevents
\begin{figure}[htbp]
	\centering
	\lstset{
		numbers=left,
		numberstyle= \tiny,
		keywordstyle= \color{blue!70},
		commentstyle= \color{red!50!green!50!blue!50},
		frame=shadowbox,
		rulesepcolor= \color{red!20!green!20!blue!20} ,
		xleftmargin=1.5em,xrightmargin=0em, aboveskip=1em,
		framexleftmargin=1.5em,
                numbersep= 5pt,
		language=C,
    basicstyle=\scriptsize\ttfamily,
    numberstyle=\scriptsize\ttfamily,
    emphstyle=\bfseries,
                moredelim=**[is][\color{red}]{@}{@},
		escapeinside= {(*@}{@*)}
	}
\begin{lstlisting}[]
import com.google.gwt.core.client.EntryPoint;
import com.google.gwt.event.dom.client.ClickEvent;
import com.google.gwt.event.dom.client.ClickHandler;
import com.google.gwt.user.client.ui.Button;
import com.google.gwt.user.client.ui.HorizontalPanel;
import com.google.gwt.user.client.ui.RootPanel;
import com.google.gwt.user.client.ui.VerticalPanel;
...
public class StockWatcher implements EntryPoint {
  private VerticalPanel mainPanel = new VerticalPanel();
  private HorizontalPanel addPanel = new HorizontalPanel();
  private Button addStockButton = new Button("Add");
  ...
  public void onModuleLoad() {
    ...
    // Assemble Add Stock panel.
    addPanel.add(addStockButton);
    // Assemble Main panel.
    mainPanel.add(addPanel);
    // Associate the Main panel with the HTML host page.
    RootPanel.get("stockList").add(mainPanel);
    // Listen for mouse events on the Add button.
    (*@{\color{blue}{addStockButton.addClickHandler(}@*) (*@{\color{purple}{new ClickHandler() \{}@*)
      public void onClick(ClickEvent event) {
        addStock();
      }
    });
  }
  private void addStock() {
    ...
  }
}
\end{lstlisting}
        \vspace{-12pt}
        \caption{Complete Source Code in gwtproject.org}
        \label{fig:example3}
\end{figure}


\vspace{2pt}
\noindent {\bf Observation 2} [{\em Dependencies/Relations among API
    Elements in a Usage}]. The dependencies/relations among the API
elements in a API usage can help a model better identify the FQNs of
the elements.  In Figure~\ref{fig:example3}, the API elements
\code{Button}, \code{addClickHandler}, and \code{ClickHandler} in GWT
have the program dependencies/relations among them. For example, in
GWT, to set a handler for a button, the object of the type
\code{Button} needs to be the {\em receiving object of the method
  call} to \code{addClickHandler}, which in turn needs to accept an
object of the type \code{ClickHandler} as an argument. These relations
exhibit in the client code using the GWT library, and are useful in
deciding the FQNs of its API elements. For example, in
Figure~\ref{fig:example2},~at line 3, if \code{addClickHandler} is
determined to be the API element
\code{com.\-google.\-gwt.\-user.\-client.\-ui.\-Button.\-add\-Click\-Handler}, the
FQN of the element at line 4 must be
\code{com\-.google\-.gwt\-.event\-.dom\-.client\-.ClickHandler}.  The
other direction of reasoning is applicable as well. In general, if a
model can learn the dependencies/relations among API elements, it
could leverage such knowledge to decide the FQNs of all those APIs at once.


%the lines 2 and 3 in Figure 2
As another example, the data dependency from the \code{def-use}
relation via the variable \code{myButton} between line 2 and line 3 in
Figure~\ref{fig:example2} is useful in deriving the FQNs of the
above API elements. If a model decides the FQN for \code{Button} at
line 2 as \code{com\-.google\-.gwt\-.user\-.client\-.ui\-.Button}, it
could derive the FQN of \code{add\-Click\-Handler} at line~3 as
\code{com.\-google.\-gwt.\-user.\-client.\-ui.\-Button.\-add\-Click\-Handler},
and vice versa.

%at the same time: ...

\subsection{Key Ideas}
\label{sec:key}

We propose {\tool}, an approach to identify the FQNs for the simple
names of variables, API classes, method calls, and field accesses in a
code snippet in online forums. From the observations, we have the
following key ideas:

\vspace{2pt}
\noindent {\bf Key Idea 1} [{\em Leveraging Machine Learning to
    Implicitly Learn Co-occurring API Elements in API Usages to Derive
    FQNs}]. {\bf Observation 1} inspires us in the first principle in
our solution, which is the basis of regularity of API usages in a
large training corpus: the API elements with their FQNs regularly
appear together in API usages have higher impact in deciding the FQNs than
the less regular ones. We leverage the complete, compilable code using
the libraries from large code corpus, in which the FQNs of all the API
elements are known. We use the code to train a Masked Language Model
(MLM) in which the FQNs of the API elements are masked. Then, the
model will be fine-tuned to predict the FQNs for the API elements in
any given code snippet.

\vspace{2pt}
\noindent {\bf Key Idea 2} [{\em ``Tell Me Your Friends, I'll Tell You
    Who You Are''}]. We consider the problem of deriving the FQNs as
the identifications of the API elements in a given code snippet.
Instead of trying to identify the FQN of an API element based on its
characteristics, we aim to derive the FQNs of related API elements at
the same time by leveraging the dependencies/relations among them.  We
use a graph representation, called Augmented Usage Graph
(AUG)~\cite{msr19}, to represent the program dependencies and
relations among program entities and API elements. We enhance the AUG
with all the FQNs because the training code is compilable. From that
AU, we mask the FQNs of all the API elements to train the Masked
Language Model (MLM).

\vspace{2pt}
\noindent {\bf Key Idea 3} [{\em Span-based Masked Language Model}]. ...

%Span-based + Length, Embedding

