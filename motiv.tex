\section{Motivation}
\label{motiv:sec}

\subsection{Motivating Examples}
\label{examples:sec}

%https://stackoverflow.com/questions/4531396/get-value-of-a-edit-text-field/4531500#4531500: SO post #4531500
\begin{figure}[htbp]
	\centering
	\lstset{
		numbers=left,
		numberstyle= \tiny,
		keywordstyle= \color{blue!70},
		commentstyle= \color{red!50!green!50!blue!50},
		frame=shadowbox,
		rulesepcolor= \color{red!20!green!20!blue!20} ,
		xleftmargin=1.5em,xrightmargin=0em, aboveskip=1em,
		framexleftmargin=1.5em,
                numbersep= 5pt,
		language=C,
    basicstyle=\scriptsize\ttfamily,
    numberstyle=\scriptsize\ttfamily,
    emphstyle=\bfseries,
                moredelim=**[is][\color{red}]{@}{@},
		escapeinside= {(*@}{@*)}
	}
\begin{lstlisting}[]
(*@{\color{blue}{Button}@*)   mButton;
EditText mEdit;

(*@@@*)Override public (*@{\color{black}{void}@*) onCreate(Bundle savedInstanceState) {
    super.onCreate(savedInstanceState);
    setContentView(R.layout.main);

    (*@{\color{blue}mButton = findViewById(R.id.button);@*)
    mEdit   = (EditText)findViewById(R.id.edittext);

    (*@{\color{blue}mButton.setOnClickListener(@*)
        (*@{\color{purple}new View.OnClickListener()@*)
        {
            public (*@{\color{black}{void}@*) onClick(View view)
            {
                Log.v("EditText", mEdit.getText().toString());
            }
        });
}
\end{lstlisting}
        \vspace{-12pt}
        \caption{StackOverflow Post \#4531500 on Android Library}
        \label{fig:example1}
\end{figure}


Let us use a few real-world examples to motivate our approach.
Figure~\ref{fig:example1} displays a code snippet of an answer to an
SO question on Android library. Due to the discussion context in SO,
and its informal nature, code snippets rarely contain sufficient
declarations and references to the fully-qualified names (FQNs). The
code snippets are often missing neccessary import statements and the
references to external types are also often unqualified (without FQNs)
since the responder assumes that those FQNs could be implicitly
understood in the context of the post. In Figure~\ref{fig:example1},
the types \code{Button}, \code{EditText}, \code{Bundle}, \code{View},
and \code{Log} are referenced by simple names only. Thus, the code
will not be compilable if the corresponding import statements are not
added.

%https://stackoverflow.com/questions/18323473/how-to-implement-gwt-java-button-and-the-clickhandler
\begin{figure}[htbp]
	\centering
	\lstset{
		numbers=left,
		numberstyle= \tiny,
		keywordstyle= \color{blue!70},
		commentstyle= \color{red!50!green!50!blue!50},
		frame=shadowbox,
		rulesepcolor= \color{red!20!green!20!blue!20} ,
		xleftmargin=1.5em,xrightmargin=0em, aboveskip=1em,
		framexleftmargin=1.5em,
                numbersep= 5pt,
		language=C,
    basicstyle=\scriptsize\ttfamily,
    numberstyle=\scriptsize\ttfamily,
    emphstyle=\bfseries,
                moredelim=**[is][\color{red}]{@}{@},
		escapeinside= {(*@}{@*)}
	}
\begin{lstlisting}[]
public class myClass implements EntryPoint {
    final (*@{\color{blue}{Button}@*) myButton = new (*@{\color{blue}{Button}@*)("text");
    (*@{\color{blue}{myButton.addClickHandler(}@*)
        (*@{\color{purple}{new ClickHandler() \{}@*)
            public (*@{\color{black}{void}@*) onClick(ClickEvent event) {
               onClickMyButton(event);
        }
    });
    private (*@{\color{black}{void}@*) onClickMyButton(ClickEvent event) {
            ... 
    }
}
\end{lstlisting}
        \vspace{-12pt}
        \caption{StackOverflow Post \#18323473 on GWT Library}
        \label{fig:example2}
\end{figure}


Importantly, the name ambiguity occurs in the references to the APIs
of the external libraries. For example, the type \code{Button} at line
1 of Figure~\ref{fig:example1} is a common unqualified type name. In
this snippet, it refers to \code{android.widget.Button}. However, it
is easily ambiguous with other APIs in different libraries that
contain the same concept. For example, Figure~\ref{fig:example2} shows
a code snippet of an answer for an SO post on the Google Web Toolkit
(GWT). Because the code snippet contains no import statement, and the
references to the APIs are unqualified, the type \code{Button} at line
2 is ambiguous with the type of the same name at line 1 of
Figure~\ref{fig:example1} on Android library. The ambiguity in type
names is popular in code snippets in the forums, e.g., the
simple names \code{getId} occur 27,434 in various Java
libraries~\cite{liveapi14}.






%https://www.gwtproject.org/doc/latest/tutorial/manageevents
\begin{figure}[htbp]
	\centering
	\lstset{
		numbers=left,
		numberstyle= \tiny,
		keywordstyle= \color{blue!70},
		commentstyle= \color{red!50!green!50!blue!50},
		frame=shadowbox,
		rulesepcolor= \color{red!20!green!20!blue!20} ,
		xleftmargin=1.5em,xrightmargin=0em, aboveskip=1em,
		framexleftmargin=1.5em,
                numbersep= 5pt,
		language=C,
    basicstyle=\scriptsize\ttfamily,
    numberstyle=\scriptsize\ttfamily,
    emphstyle=\bfseries,
                moredelim=**[is][\color{red}]{@}{@},
		escapeinside= {(*@}{@*)}
	}
\begin{lstlisting}[]
import com.google.gwt.core.client.EntryPoint;
import com.google.gwt.event.dom.client.ClickEvent;
import com.google.gwt.event.dom.client.ClickHandler;
import com.google.gwt.user.client.ui.Button;
import com.google.gwt.user.client.ui.HorizontalPanel;
import com.google.gwt.user.client.ui.RootPanel;
import com.google.gwt.user.client.ui.VerticalPanel;
...
public class StockWatcher implements EntryPoint {
  private VerticalPanel mainPanel = new VerticalPanel();
  private HorizontalPanel addPanel = new HorizontalPanel();
  private Button addStockButton = new Button("Add");
  ...
  public void onModuleLoad() {
    ...
    // Assemble Add Stock panel.
    addPanel.add(addStockButton);
    // Assemble Main panel.
    mainPanel.add(addPanel);
    // Associate the Main panel with the HTML host page.
    RootPanel.get("stockList").add(mainPanel);
    // Listen for mouse events on the Add button.
    (*@{\color{blue}{addStockButton.addClickHandler(}@*) (*@{\color{purple}{new ClickHandler() \{}@*)
      public void onClick(ClickEvent event) {
        addStock();
      }
    });
  }
  private void addStock() {
    ...
  }
}
\end{lstlisting}
        \vspace{-12pt}
        \caption{Complete Source Code in gwtproject.org}
        \label{fig:example3}
\end{figure}

