\section{Related Work}
\label{sec:related}

The approaches to recover the FQNs for the API elements in incomplete
code snippets can be broadly classified into the following categories.
The first category is {\em program analysis}. Partial Program Analysis
(PPA)~\cite{dagenais-oopsla08} resolves the types by applying a set of
heuristics on the syntactic constructs to infer the declared types of
expressions. RecoDoc~\cite{dagenais-icse12} uses PPA to infer links
between APIs and documentation. It requires the target library to be
specified. The second category of approaches leverages {\em
  information retrieval}. Baker~\cite{liveapi14} tracks the scopes of
the names and the candidate lists are combined according to the
scoping rules and get smaller. COSTER~\cite{coster-ase19} formulates
the FQN problem as searching. It aims to match the context of the
query API element with the FQNs in the database regarding three
criteria on likelihood, context similarity, and name similarity.
The third category is {\em constraint-based}. Differing from COSTER,
SnR~\cite{snr-icse22} builds the knowledge of the constraints on
the API elements and extracts such constraints in the code snippet
to match and solve them against the knowledge base.

